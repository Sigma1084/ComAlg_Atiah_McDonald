\chapter{Rings and Ideals} \label{ch:rings-and-ideals}

\begin{exercise}{}{}
	Let \( x \) be a nilpotent element of a ring \( A \).
	Show that \( 1 + x \) is a unit of A\@.
	Deduce that the sum of a nilpotent element and a unit is a unit.
\end{exercise}

Suppose \( x^n = 0 \) for some \( n \in \NN \). \\
Clearly, \( \brak{1 + x}\brak{1 - x + x^2 \ldots + \brak{-1}^{n-1} x^{n-1}}
= 1 - x^n = 1 \). \\
Hence, \( 1 + x \) is a unit. \\
Consider \( u + x \) for some unit \( u \).
Write \( u + x = u \brak{1 + xu\inv} \). \\
Now, we know that \( xu\inv \) is nilpotent with the same \( n \)
and hence, \brak{1 + xu\inv} is a unit. \\
Since product of units is a unit, we can conclude \( u + x \)
is a unit.



\begin{exercise}{}{}
	Let \( A \) be a ring and \( A\sbrak{x} \) be the ring of
	polynomials in an intermediate \( x \) with coefficients in \( A \).
	Let \( f = a_0 + a_1 x + \ldots + a_n x^n \in A\sbrak{x} \).
	Prove that
	\begin{enumerate}
		\item \( f \) is a unit \( \iff a_0 \) is a unit in \( A \) and
			\( a_1, \ldots, a_n \) are nilpotent.
		\item \( f \) is nilpotent \( \iff a_0, a_1, \ldots, a_n \)
			are nilpotent.
		\item \( f \) is a zero-divisor \( \iff \) there exists \( a \neq 0 \)
			 in \( A \) such that \( af = 0 \).
		\item \( f \) is said to be primitive if \brak{a_0, a_1, \ldots, a_n}
			\( = \brak{1} \).
			Prove that if \( f, g \in A\sbrak{x} \), then
			\( fg \) is primitive \( \iff f \AND g \) are primitive.
	\end{enumerate}
\end{exercise}

\begin{enumerate}

	\item
	We prove the result using induction on the degree of the polynomial.
	Base case \( n = 0 \) is trivially correct since \( a_0 \) is a unit
	in \( A\sbrak{x} \iff a_0 \) is a unit in \( A \).
	Let us assume the result for all polynomials of degree \( n-1 \) or lower.

	\( \implies \) \\
	Suppose the inverse of \( f \) be
	\( g(x) = b_0 + b_1 x + \ldots + b_m x^m \). \\
	Observe that \( a_n^{r+1} b_{m-r} = 0 \ \forall\ r \). \\
	\( r = 0 : \) Co-efficient of \( x^{n+m} \) is 0.
	Thus, \( a_n b_m = 0 \implies a_n^{0+1} b_{m-0} = 0 \). \\
	\( r = 1 : \) Co-efficient of \( x^{n+m-1} \) is 0.
	Thus, \( a_n b_{m-1} + a_{n-1} b_m = 0 \implies a_n^{1+1} b_{m-1} = 0 \).
	after multiplying by \( a_n \). \\
	\( \ldots \) \\
	\( r = m : a_n^{m+1} b_0 = 0 \) \\
	Since we also know that \( a_0 b_0 = 1 \), we know that \( b_0 \) is
	a unit.
	Multiplying \( a_0 \) in the above equation, we can see that
	\( a_n^{m+1} = 0 \) and hence \( a_n \) is nilpotent. \\
	We also know that the element \( a_n x^n \) is nilpotent in the
	polynomial ring and hence \( f - a_n x^n \) is a unit using the
	result that the sum of a unit and a nilpotent element is a unit. \\
	But \( f - a_n x^n \) is an \( n-1 \) or a lower degree polynomial and
	hence, the result holds, that is \( a_1, \ldots, a_{n-1} \) are also
	nilpotent along with \( a_0 \) being a unit.

	\( \Longleftarrow \) \\
	Using \( a_0, a_1, \ldots, a_{n-1} \), we construct \( f - a_n x^n \)
	which is a unit using the induction hypothesis and we know that
	\( a_n x^n \) is a nilpotent element and hence,
	\( f - a_n x^n + a_n x^n = f \) being the sum of unit and
	a nilpotent element, is a unit.


	\item
	We prove the result by induction on the degree of the polynomial.
	\( n = 0 \) is trivially correct since
	\( a_0 \) is a nilpotent element of \( A \) if and only if \( a_0 \) is
	a nilpotent element of \( A\sbrak{x} \). \\
	Assume the result for all polynomials of degree \( n-1 \) or lower.

	\( \implies \) \\
	\( f = a_0 + a_1 x + \ldots + a_n x^n \) is nilpotent.
	Thus, for some \( r \in \NN \), \( f^r = 0 \).
	Considering the \( rn^{\text{th}} \) coefficient and setting it to zero,
	we get \( a_n^r = 0 \) and thus, \( a_n x^n \) is a nilpotent
	element of \( A\sbrak{x} \).
	Since we know that the sum of 2 nilpotent elements is nilpotent,
	\( f - a_n x^n \) being a polynomial of degree less than \( n \)
	is nilpotent and we get \( a_0, \ldots, a_{n-1} \) are nilpotent.

	\( \Longleftarrow \) \\
	\( a_0, \ldots, a_n \) are nilpotent elements of \( A \). \\
	Clearly, \( a_n x^n \) is nilpotent and the polynomial
	\( a_0 + a_1 x + \ldots + a_{n-1} x^{n-1} \) is nilpotent using
	the induction hypothesis.
	Adding the above two, we get \( f \) is nilpotent.


	\item
	The only if part is trivial since \( a \in A\sbrak{x} \).
	Proof for the \( \implies \) part. \\
	Suppose \( f \) is a zero-divisor.
	Then, there exists \( g \in A\sbrak{x} \) with the least degree
	such that \( fg = 0 \).
	We have \( a_n b_m = 0 \) since the coefficient of \( x^{n+m} \) is 0. \\
	This implies \( a_n g = 0 \) since \( a_n g f = 0 \) and
	if \( a_n g \neq 0 \), it is a polynomial of degree less than n and
	annihilates \( f \) which contradicts the minimality of the degree of
	\( g \). \\

	We show by induction that \( a_{n-r} g = 0 \) for \( 0 \leq r \leq n \).
	The base case being \( r = 0, a_{n} g = 0 \) is shown above. \\
	Let us just go 1 level for better underdstanding. \\
	Since we proved \( a_n g = 0 \), we showed that \( a_n b_i = 0 \)
	for all \( i \). \\
	Since \( f g = 0 \), considering the coefficient of \( x^{n+m-1} \),
	we have \( a_n b_{m-1} + a_{n-1} b_m = 0 \) but \( a_n b_{m-1} = 0 \). \\
	Thus, \( a_{n-1} b_m = 0 \) and \( a_{n-1} \) annihilates the leading
	co-efficient of \( g \). \\
	Using the same argument as before, we can see that \( a_{n-1} g = 0 \). \\

	Now, we can use the induction hypothesis to show that
	\( a_{n-i} g = 0 \) for \( 0 \leq i \leq n \) and hence,
	\( a_0 g = 0 \). \\
	Thus, \( a_i b_j = 0 \) for all \( 0 \leq i \leq n \) and
	\( 0 \leq j \leq m \). \\
	Also, \( b_m \neq 0 \) and \( a_i b_m = 0 \) for all \( i \),
	we can conclude that \( b_m f = 0 \). \\


	\item
	Suppose \( f = a_0 + a_1 x + \ldots + a_n x^n \) \\
	\( g = b_0 + b_1 x + \ldots + b_m x^m \) and \\
	\( fg = c_0 + c_1 x + \ldots + c_{n+m} x^{n+m} \) where
	\( c_i = \sum_{j=0}^i a_j b_{i-j} \) for \( 0 \leq i \leq n+m \).

	\( \implies \) \\
	Suppose \( fg \) is primitive. \\
	Then, \( \exists\ \gamma_1, \gamma_2, \ldots, \gamma_n \in A \)
	such that \( \gamma_1 c_1 + \gamma_2 c_2 + \ldots +
	\gamma_{m+n} c_{m+n}  = 1 \). \\
	Thus,
	\[ \sum_{i=0}^n \sum_{j=0}^m a_i b_j \gamma_{i+j} = 1 \]
	Putting \( \alpha_i \coloneqq \sum_{j=0}^m b_j \gamma_{i+j} \),
	we get \( \sum_{i=1}^n a_i \alpha_i = 1 \) and hence,
	\( f \) is primitive. \\
	Similarly, we get \( g \) is primitive.

	\( \Longleftarrow \) \\
	We prove the result using the contrapositive. \\
	Suppose \( fg \) is not primitive. \\
	Then, \( \brak{c_0, c_1, \ldots, c_{m+n}} \) is a non-trivial
	ideal of \( A \), and is contained in some maximal ideal,
	say \( \mfc \) and \( 1 \notin \mfc \). \\
	Clearly, \( \bigslant{A}{\mfc} \) is a field. \\
	We have \( fg \equiv 0 \pmod{\mfc} \).
	Since \( \bigslant{A}{\mfc} \) is a field, this implies
	\( f \equiv 0 \pmod{\mfc} \OR g \equiv 0 \pmod{\mfc} \). \\
	This means, either \( f \OR g \) is contained in \( \mfc \) which implies
	\( 1 \) does not belong to it and hence is not primitive.
\end{enumerate}


\begin{exercise}{}{}
	Generalize the results of Exercise 2 to a polynomial ring
	\( A \sbrak{x_1, \ldots, x_r} \) in several indeterminates.
\end{exercise}

Suppose we consider the polynomial ring \( A \sbrak{x_1, \ldots, x_r} \).
An element of the above ring looks like,
\[
	f = \sum_{i_1 + i_2 + \ldots + i_r \leq n}
	a_{i_1, i_2, \ldots, i_r} x_1^{i_1} x_2^{i_2} \cdots x_r^{i_r}
\]

\begin{enumerate}
	\item \( f \) is a unit in \( A \sbrak{x_1, \ldots, x_r} \)
	\( \iff a_{0, 0, \ldots, 0} \) is a unit in \( A \) and
	every other co-efficient is nilpotent.

	\item \( f \) is a nilpotent element in \( A \sbrak{x_1, \ldots, x_r} \)
	\( \iff a_{i_1, i_2, \ldots, i_r} \) is a nilpotent element in \( A \)
	for any combination of \( \brak{i_1, i_2, \ldots, i_r} \).

	\item \( f \) is a zero-divisor in \( A \sbrak{x_1, \ldots, x_r} \)
	\( \iff \ \exists\ a \neq 0 \in A \) such that \( a f = 0 \).

	\item Suppose \( f, g \in A \sbrak{x_1, \ldots, x_r} \). \\
	\( fg \) is primitive in \( A \sbrak{x_1, \ldots, x_r} \)
	\( \iff f \) and \( g \) are primitive in \( A \sbrak{x_1, \ldots, x_r} \).
\end{enumerate}



\begin{exercise}{}{}
	In the ring \( A \sbrak{x} \), the Jacobson radical is equal to
	nilradical.
\end{exercise}

We know that every maximal ideal is a prime ideal and hence, the nilradical
is contained in the Jacobson radical.

We need to prove that Jacobson radical is contained in nilradical.

Let \( f(x) \in \mfR \implies 1 - fg \) is a unit in \( A[x] \ \forall\
g \in A[x] \).
Suppose
\[
	f(x) = a_0 + a_1 x \cdots a_n x^n
\]
We use the fact that if \( h(x) = h_0 + \cdots + h_k x^k \) is unit,
\( h_0 \) is a unit and \( h_1, \cdots, h_k \) are nilpotent.

Putting \( g = x \), we see that \( 1 - fx \) is a unit and hence
\( a_0, a_1, \cdots a_n \) are nilpotent.

Since coefficients of \( f \) are nilpotent, we can conclude
that \( f \) is nilpotent.
\[
	f \in \mfR \implies f \in \mfN
\]
Hence, Jacobson radical is contained in nilradical and we proved the result.


\begin{exercise}{}{}
	Let \( A \) be a ring and let \( A[[x]] \) be a ring of formal
	power series \( f = \sum_{i=0}^{\infty} a_n x^n \) with coefficients
	in \( A \).
	Show that
	\begin{enumerate}
		\item \( f \) is a unit in \( A[[x]] \iff a_o \) is a unit in \( A \).
		\item \( f \) is a nilpotent element in \( A[[x]] \implies a_n \) is
		nilpotent for any \( n \geq 0 \).
		Is the converse true?
		\item \( f \) belongs to the Jacobson radical of \( A[[x]] \iff
		a_0 \) belongs to the Jacobson radical of \( A \).
		\item The contraction of a maximal ideal \( \mfm \) of \( A[[x]] \)
		is a maximal ideal of \( A \) and \( \mfm \) is generated by
		\( \mfm^c \AND x \).
		\item Every prime ideal of \( A \) is a contraction of a prime ideal
		of \( A[[x]] \).
	\end{enumerate}
\end{exercise}

\begin{enumerate}
	\item
	\( \brak{\implies} \) \\
	Now, \( f \) is a unit in \( A[[x]] \) \\
	\( \implies \exists\ g = \sum_{i=0}^{\infty} b_i x^i \in A[[x]] \)
	such that \( fg = 1 \) \\
	\( \implies a_0 b_0 = 1 \implies a_0 \) is a unit. \\

	\( \brak{\Longleftarrow} \) \\
	Suppose \( a_0 \) is a unit in \( A \). \\
	Let us try to construct \( g = \sum_{i \geq 0} b_i \)
	such that \( fg = 1 \). \\
	Since \( a_0 \) is a unit, there is a \( b_0 \) such that
	\( a_0 b_0 = 1 \).
	Since the coefficient of \( x^n \) is \( 0\ \forall\ n \in \NN \)
	in \( fg \), we have
	\[
		0 = \sum_{i=0}^n a_{n-i} b_{i}
		\implies a_0 b_n = - \sum_{i=0}^{n-1} a_{n-i} b_{i}
		\implies b_n = - b_0 \sum_{i=0}^{n-1} a_{n-i} b_{i}
	\]
	We just expressed \( b_n \) in terms of \( b_0, \cdots, b_{n-1} \)
	such that coefficient of \( x^n \) in \( fg = 0 \). \\
	\( \exists\ g \in A[[x]] \) such that \( fg = 1 \).


	\item
	Suppose \( f \) be a nilpotent element in \( A[[x]] \).
	Then, \( f^n = 0 \) for some \( n \in \NN \). \\
	Coefficient of \( 1 \) in \( f^n \) is \( a_0^n = 0 \)
	\( \implies a_0 \) is nilpotent. \\
	Now, consider the element \( f_1 = f - a_0 \).
	Since \(f' = a_1 x + a_2 x^2 + \cdots\), being the sum of
	nilpotent elements, is nilpotent, and \(f' = xf_1\).
	Thus, \(f_1\) is nilpotent. \\
	\(a_1\) being the coefficient of \(1\) in \(f_1\), is nilpotent
	and using the same argument, we can show that \(a_2\) is nilpotent.
	Thus, \(a_n\) is nilpotent for any \(n \geq 0\).

	\item
	\( \brak{\implies} \) \\
	\(f\) belongs to the Jacobson radical of \(A[[x]]\)
	\(\implies 1 - fg\) is a unit \(\forall\ g \in A[[x]]\). \\
	Putting \(g = y\) for some \(y \in A\), we get
	\(1 - a_0 y\) is a unit in \(A \ \forall\ y \in A\) using the
	previous proposition. \\
	This means \(a_0 \in \mfR_A\).

	\(\brak{\Longleftarrow}\) \\
	Suppose \(a_0 \in \mfR \implies 1-a_0 y\) is a unit in
	\(A \ \forall\ y \in A\). \\
	Cosider some \(g = \sum_{i \geq 0} b_i x^i \in A[[x]]\). \\
	\(1-a_0 b_0\) is a unit in \(A \implies 1 - fg\) is a unit
	in \(A[[x]]\) for any \(g \in A[[x]]\). \\
	This proves the claim.
\end{enumerate}


\begin{exercise}{}{}
	A ring \( A \) is such that every ideal not contained in the Nilradical
	contains a non-zero idempotent (that is, an element \( e \) such that
	\( e^2 = e \) and \( e \neq 0 \)).
	Prove that the Nilradical and the Jacobson radical of \( A \) are equal.
\end{exercise}

It is trvial to note that \(\mfN \subseteq \mfR\).

If \(\mfR \nsubseteq \mfN\), then there exists an idempotent \(e \in \mfR\)
such that \(e^2 \neq e, e \neq 0\).
We show that this is not possible.

Since \(e \in \mfR\), \(1 - ey\) is a unit in \(A\) for every \(y \in A\).
Putting \(y = 1\), we get \(1 - e\) is a unit in \(A\).

Since \(e, 1 \in \mfm\), we have \(1-e \in \mfm\) for some maximal ideal
\(\mfm\).

Also, note that \(\brak{1-e}e = e - e^2 = 0 \implies e = 0\)
since \(1-e\) is a unit which is a contradiction.

Hence, \(\mfN = \mfR\).


\begin{exercise}{}{}
	Let \( A \) be a ring in which every element \( x \) satisfies
	\( x^n = x \) for some \( n \in \NN \AND n > 1 \) depending on \( x \).
	Show that every prime ideal of \( A \) is maximal.
\end{exercise}

Suppose \( \mfp \) be a prime ideal of \( A \).

Consider a non zero element \( x \in \bigslant{A}{\mfp} \)
\( \implies x^n \in \bigslant{A}{\mfp} \) for some \( n>1 \in \NN \).

Therefore, \( x^n - x \in \bigslant{A}{\mfp} \implies x\brak{x^{n-1} - 1}
\in \bigslant{A}{\mfp} \).

Since \( \bigslant{A}{\mfp} \) is an integral domain and
\( x \) is chosen to be a non zero element, \\
\( x\brak{x^{n-1} - 1} = 0 \implies x^{n-1} - 1 = 0
\implies x\inv = x^{n-2} \).

We therefore proved that any non-zero element in \( \bigslant{A}{\mfp} \)
is a unit and hence \( \bigslant{A}{\mfp} \) is a field.

Thus, we conclude \( \mfp \) is a maximal ideal.


\begin{exercise}{}{}
	Let A be a ring \(\neq\) 0.
	Show that the set of prime ideals of \(A\) has minimal elements
	with respect to inclusion.
\end{exercise}

Suppose we take the set of prime ideals of \(A\) and consider
the inclusion relation on it as a partial order defined by
\[
	\mfp_1 < \mfp_2 \iff \mfp_1 \supset \mfp_2
\]
This forms a chain in the set of prime ideals of \(A\).

Using Zorn's lemma, we can show that there exists a maximal element
\( \mfp \) in the chain which is the minimal prime ideal of \(A\).


\begin{exercise}{}{}
	Let \(\mfa \neq (1)\) be an ideal in a ring \(A\).
	Show that \(\mfa = r(\mfa) \iff \mfa\) is an intersection
	of prime ideals.
\end{exercise}

One side is trivial since \(r(\mfa)\) is the intersection of all prime ideals
containing \(\mfa\).




\begin{exercise}{}{}
	Let \(A\) be a ring, \(\mfN\) its nilradical.
	Show that the following are equivalent.
	\begin{enumerate}
		\item \(A\) has exactly one prime ideal.
		\item \(A\) Every element of \(A\) is a unit or nilpotent.
		\item \(\bigslant{A}{\mfN}\) is a field.
	\end{enumerate}
\end{exercise}

\(\brak{1 \implies 3}\) \\
Suppose \(A\) has exactly one prime ideal, \(\mfp\).
Then, \(\mfN = \mfp\).

Since any maximal ideal is a prime ideal, and we only have one prime ideal,
the prime ideal must be a maximal ideal and since \(\mfp = \mfN\),
\(\mfN\) is a maximal ideal and \(A\) is a local ring.

We directly get the result that \(\bigslant{A}{\mfN}\) is a field.


\(\brak{3 \implies 2}\) \\
Suppose \(\bigslant{A}{\mfN}\) is a field.

Consider some \(x \in A\).
If \(x \in \mfN\), it is nilpotent and if \(x \notin \mfN\),
\[
	\exists\ y \in A, y \notin \mfN \st \brak{x + \mfN}\brak{y + \mfN}
	= 1 + \mfN \quad \implies xy - 1 \in \mfN
\]

Clearly, since the sum of a unit and a nilpotent unit is a unit, we have
\( xy - 1 + 1 = xy\) is a unit in \(A\).
\[
	\implies \exists\ z \in A \st xyz = 1
\]

\(yz\) is the inverse of \(x\) and hence \(x\) is a unit in \(A\).

Therefore, every element of \(A\) is either nilpotent or a unit. \\


\(\brak{2 \implies 3}\) \\
We need to prove that \(\bigslant{A}{\mfN}\) is a field.

Suppose we consider \(x \in A, x \notin \mfN\)
\[
	\implies \exists\ y \in A, y \notin \mfN \st xy = 1
\]
Clearly,
\[
	\implies (x + \mfN)(y + \mfN) = 1 + \mfN
\]
and hence, we can see that \(\bigslant{A}{\mfN}\) is a field. \\


\(\brak{3 \implies 1}\) \\
We can see that \(\mfN\) is a maximal ideal.

Suppose for contradiction, let there be at least 2 distinct prime ideals
\( \mfp_1, \mfp_2 \) in \(A\).

Clearly, this means that
\[
	\mfN \subseteq \mfp_1 \AND \mfN \subseteq \mfp_2
\]

We claim that
\[
	\mfN \subset \mfp_1 \OR \mfN \subset \mfp_2
\]

since if that does not hold, then we have
\[
	\mfN = \mfp_1 = \mfp_2
\]
which contradicts the fact that \( \mfp_1 \) and \( \mfp_2 \) are distinct
and if it holds, it contradicts the maximality of \(\mfN\).

And hence, we can see that there is only a single prime ideal in \(A\).


\begin{exercise}{}{}
	A ring \(A\) is called a Boolean ring if \(x^2 = x\) for every
	\(x \in A\).
	In a Boolean ring, show that
	\begin{enumerate}
		\item \(2x = 0 \ \forall\ x \in A\)
		\item Every prime ideal \(\mfp\) is maximal and
		\(\bigslant{A}{\mfp}\) is a field with 2 elements.
		\item Every finitely generated ideal in \(A\) is principal.
	\end{enumerate}
\end{exercise}

\begin{enumerate}
	\item
	Since \(x^2 = x \ \forall\ x \in A\), we have
	\begin{align*}
		(1+x)^2 &= (1+x) \\
		\implies \cancel{1} + 2x + \cancel{x^2} &= \cancel{1} + \cancel{x} \\
		\implies 2x &= 0 \quad \forall\ x \in A
	\end{align*}

	\item
	Consider some prime ideal \(\mfp\).
	We know that every element of \(\bigslant{A}{\mfp}\) satisfies
	\(x^2 = x\).
	Since \(\mfp\) is a prime ideal, \(\bigslant{A}{\mfp}\) is an
	integral domain, all the elements of \(\bigslant{A}{\mfp}\) are
	the elements that satisfy \(x = x^2\).
	\[
		x = x^2 \implies x(1-x) = 0 \implies x = 0 \OR x = 1
	\]
	Hence, \(A\) is a field with 2 elements which implies that
	\(\mfp\) is a maximal ideal.

	\item
	We need to show that any ideal generated by 2 elements
	can be generated by a single element. \\
	In other words, we need to show that
	\[
		\brak{x, y} = \brak{z} \quad\quad
		\text{for some } z \in A
		\numberthis \label{eq:q11_to-prove}
	\]
	We claim that
	\[
		z = x + y - xy
	\]
	since \(xz = x \AND yz = y\) and it satisfies~\eqref{eq:q11_to-prove}.

	Hence, we can merge any finite number of generators into a single
	generator which implies every finitely generated ideal is principal.
\end{enumerate}


\begin{exercise}{}{}
	A local ring contains no idempotents \(\neq 0, 1\).
\end{exercise}

Recall that a ring \(A\) is called a local ring if it contains a unique
maximal ideal \(\mfm\).

%Now, if there is some element \(x \in A\) such that \(x^2 = x\),
%then, we must have \(x^2 + \mfm = x + \mfm\).
%\[
%	\implies x \equiv 0 \pmod{\mfm} \qorq x \equiv 1 \pmod{\mfm}
%\]
%since \(\bigslant{A}{\mfm}\) is a field.

Notice that if \(e\) is an idempotent in \(A\), then \(e(e-1) = e^2 - e = 0\)
shows that \(e\) and \(e-1\) are zero divisors and hence, not units.
Thus, \(e, 1-e \in \mfm\) but
\begin{gather*}
	\implies e + (1-e) = 1 \in \mfm \\
	\Longrightarrow \Longleftarrow
\end{gather*}
