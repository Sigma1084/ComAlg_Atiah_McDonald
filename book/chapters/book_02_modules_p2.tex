\chapter*{Modules 2}  \label{ch:modules2}

\section{Exact Sequences}
A sequence of \(A\)-modules and \(A\)-module homomorphisms
\[
	\cdots \xrightarrow[\quad\quad]{} M_{i-1} \xrightarrow[\quad\quad]{f_i}
	M_i \xrightarrow[\quad\quad]{f_{i+1}} M_{i+1}
	\xrightarrow[\quad\quad]{} \cdots
\]
is said to be exact at \(M_i\) if
\[
	\Img(f_i) = \Ker(f_{i+1})
\]

The sequence is said to be exact if it is exact at every \(M_i\).

In particular,
\begin{enumerate}
	\item \(0 \xrightarrow[\quad]{} M' \xrightarrow[\quad]{f} M\)
	is exact \(\iff f\) is injective.
	\item \(M \xrightarrow[\quad]{g} M'' \xrightarrow[\quad]{} 0\)
	is exact \(\iff g\) is surjective.
	\item \(0 \xrightarrow[\quad]{} M' \xrightarrow[\quad]{f} M
	\xrightarrow[\quad]{g} M'' \xrightarrow[\quad]{} 0\)
	is exact \(\iff f\) is injective and \(g\) is surjective.
\end{enumerate}
In the last example, \(g\) induces a homomorphism
\(\bigslant{M}{f(M)} = \Coker(f)\) onto \(M''\).

A sequence of type 3 is called a \emph{short exact sequence}.

Any long exact sequence of \(A\)-modules can be split into
short exact sequences.


\subsection{Right exactness of the functor \(\Hom_A(-, N)\)}

\begin{proposition}{}{}
	Let
	\[
		M' \xrightarrow[\quad\quad]{u} M \xrightarrow[\quad\quad]{v} M''
		\xrightarrow[\quad\quad]{} 0
	\]
	be a sequence of \(A\)-modules and homomorphisms.
	Then, the sequence is exact if and only if

	for all \(A\)-modules \(N\), the sequence
	\[
		\Hom_A(M', N) \xleftarrow[\quad\quad]{\bar{u}} \Hom_A(M, N)
		\xleftarrow[\quad\quad]{\bar{v}} \Hom_A(M'', N)
		\xleftarrow[\quad\quad]{} 0
	\]
	is exact.
\end{proposition}

\begin{proof} \ \\
	\(\brak{\Longrightarrow}\)

	For the if part, suppose that the sequence is exact.

	Consider the sequence
	\[
		\Hom_A(M', N) \xleftarrow[\quad\quad]{\bar{u}} \Hom_A(M, N)
		\xleftarrow[\quad\quad]{\bar{v}} \Hom_A(M'', N)
		\xleftarrow[\quad\quad]{} 0
	\]
	for some \(A\)-module \(N\).

	The maps \(\bar{u}\) and \(\bar{v}\) are defined as
	\[
		\bar{u}(f) = f \circ u \colon M' \to M \to N
		\qandq \bar{v}(f'') = f'' \circ v \colon M \to M'' \to N
	\]

	We need to prove 2 things (exactness at 2 places):
	\begin{enumerate}
		\item \(\bar{v}\) is injective \(\OR \Ker\bar{v} = 0\).
		\item \(\Img\bar{v} = \Ker\bar{u}\).
	\end{enumerate}

	First, we prove that \(\Ker\bar{v} = 0\).
	\begin{align*}
		f'' \in \Ker\bar{v} &\implies f'' \circ v (m) = 0 \ \forall\ m \in M \\
		&\implies f'' \circ v(M) = 0 \\
		&\implies f''(M'') = 0 \implies f'' = 0
	\end{align*}

	Now, we prove that
	\[
		\Img\bar{v} = \Ker\bar{u}
	\]

	Consider some \(f \in \Img\bar{v}\).
	We need to show that \(f \in \Ker\bar{u}\).

	Then, there exists \(f'' \in \Hom_A(M'', N)\) such that
	\(f = f'' \circ v\).


	This is equivalent to showing that \(\bar{u}(f) = 0\)
	or \(f \circ u (M') = 0\).
	\[
		f \circ u (M') = f'' \circ v \circ u (M')
		= f'' \circ v \circ \Img(u) = f'' \circ v \circ \Ker(v)
		= f'' \circ 0 = 0
	\]
	Hence, we showed that
	\[
		\Img\bar{v} \subseteq \Ker\bar{u}
	\]

	Now, suppose that \(f \in \Ker\bar{u}\).
	We need to show that \(f \in \Img\bar{v}\).

	To show that \(f \in \Img\bar{v}\), we need to construct a
	function \(f'' \in \Hom_A(M'', N)\) such that
	\(f = \bar{v}(f'') = f'' \circ v\).

	We know that \(v\) is surjective.
	So, for every \(m'' \in M''\), there exists \(m \in M\) such that
	\(v(m) = m''\) and hence, we can define a function \(f''\)
	given by \(f''(m'') = f''(v(m)) = f(m)\).

	We need to show that this map is well-defined.
	Suppose
	\begin{align*}
		m'' = v(m_1) = v(m_2)
	\end{align*}
	Then, we need to prove that \(f(m_1) = f(m_2)\).

	Since
	\[
		v(m_1) = v(m_2) \implies m_1 - m_2 \in \Ker(v) \subseteq \Ker(f)
		\quad\quad\quad f = f'' \circ v
	\]
	This forces \(f(m_1) = f(m_2)\) and hence, \(f''\) is well-defined.

	\vspace{3em}

	\(\brak{\Longleftarrow}\)

	Now, given the exactness of the sequence
	\[
		\Hom_A(M', N) \xleftarrow[\quad\quad]{\bar{u}} \Hom_A(M, N)
		\xleftarrow[\quad\quad]{\bar{v}} \Hom_A(M'', N)
		\xleftarrow[\quad\quad]{} 0
	\]
	for all \(A\)-modules \(N\), we need to show that the sequence
	\[
		M' \xrightarrow[\quad\quad]{u}
		M \xrightarrow[\quad\quad]{v} M'' \xrightarrow[\quad\quad]{} 0
	\]
	is exact.

	Similar to the previous proof, we need to show 2 things:
	\begin{enumerate}
		\item \(v\) is surjective \(\OR \Img v = M''\).
		\item \(\Img u = \Ker v\).
	\end{enumerate}

	Notice that \(\bar{v}\) is injective for all \(A\)-modules \(N\)
	and hence, let us put
	\[
		N = \bigslant{M''}{\Img(v)}
	\]
	We need to show that \(N = 0\).

	Consider some \(f'' \in \Hom_A \brak{M'', \bigslant{M''}{\Img(v)}}\).

	Notice that \(\bar{v}(f'') = f'' \circ v = 0\).

	Since \(\bar{v}\) is injective, we have \(f'' = 0\).
	\[
		\implies \Hom_A \brak{M'', \bigslant{M''}{\Img(v)}} = 0
		\implies \bigslant{M''}{\Img(v)} = 0
		\implies M'' = \Img(v)
	\]
	Now, since we know that
	\[
		\Ker(\bar{u}) = \Img(\bar{v})
		\implies f'' \circ v \circ u = 0 \ \forall\ f'' \in \Hom_A(M'', N)
	\]
	Putting \(f'' = 1\), the identity function, we get
	\[
		v \circ u = 0 \implies \Img(u) \subseteq \Ker(v)
	\]
	Finally, consider
	\[
		N = \bigslant{M}{\Img(u)}
	\]
	The natural map
	\[
		\phi \colon M \to \bigslant{M}{\Img(u)}
		\quad \in \quad \Hom_A \brak{M, \bigslant{M}{\Img(u)}}
	\]
	Notice that
	\[
		\bar{u}(\phi) = \phi \circ u = 0 \implies
		\phi \in \Ker(\bar{u}) = \Img(\bar{v})
	\]
	Hence, there is a function
	\[
		\psi \colon M'' \to \bigslant{M}{\Img(u)}
		\qstq \phi = \psi \circ v
	\]
	And so, notice that
	\[
		\Ker(v) \subseteq \Ker(\phi) = \Img(u)
	\]
	And thus,
	\[
		\Ker(v) = \Img(u)
	\]
	and we are done.
\end{proof}

\subsection{Left exactness of the functor \(\Hom_A(M, -)\)}

\begin{proposition}{}{}
	Let
	\[
		0 \xrightarrow[\quad\quad]{} N' \xrightarrow[\quad\quad]{u}
		N \xrightarrow[\quad\quad]{v} N''
	\]
	be a sequence of \(A\)-modules and homomorphisms.
	Then, the sequence is exact if and only if

	for all \(A\)-modules \(M\), the sequence
	\[
		0 \xrightarrow[\quad\quad]{} \Hom_A(M, N')
		\xrightarrow[\quad\quad]{\bar{u}} \Hom_A(M, N)
		\xrightarrow[\quad\quad]{\bar{v}} \Hom_A(M, N'')
	\]
	is exact.
\end{proposition}

\begin{proof} \ \\
	The functions \(\bar{u}\) and \(\bar{v}\) are defined as
	\[
		\bar{u}(f) = u \circ f \quad\quad\quad
		\bar{v}(f) = v \circ f
	\]

	\( \brak{\Longrightarrow} \)

	We need to show 2 things:
	\begin{enumerate}
		\item \(\bar{u}\) is injective or \(\Ker(\bar{u}) = 0\).
		\item \(\Img(\bar{u}) = \Ker(\bar{v})\).
	\end{enumerate}

	Consider some \(f' \in \Hom_A(M, N')\)
	\begin{align*}
		f' \in \Ker(\bar{u})
		&\implies \bar{u}(f') = u \circ f' = 0 \\
		&\implies \Img(f') \subseteq \Ker(u) = 0 \\
		&\implies f' = 0 \implies \Ker(\bar{u}) = 0
	\end{align*}

	\(\Img(\bar{u}) \subseteq \Ker(\bar{v})\) can be shown by proving
	\[
		\bar{v} \circ \bar{u} = 0
	\]

	For some \(f' \in \Hom_A(M, N')\),
	\[
		\bar{v} \circ \bar{u}(f') = v \circ u \circ f'
	\]
	We know that
	\[
		v \circ u(n) = 0 \ \forall\ n' \in N'
		\implies v \circ u \circ f' = 0
	\]
	Now, to prove \(\Ker(\bar{v}) \subseteq \Img(\bar{u})\),
	consider some \(f \in \Ker(\bar{v})\).
	We have
	\begin{align*}
		&v \circ f (n) = 0 \ \forall\ n \in N \\
		\implies &\Img(f) \subseteq \Ker(v) = \Img(u) \\
		\implies &\exists\ n' \in N' \st u(n') = f(m) \ \forall\ m \in M \\
		&\text{Define } f' \colon M \to N' \text{ by } f'(m) = n' \\
	\end{align*}
	Hence, we have
	\[
		\bar{u}(f') = u \circ f' = f
		\implies f \in \Img(\bar{u})
		\implies \Ker(\bar{v}) \subseteq \Img(\bar{u})
	\]

	\vspace{3em}

	\( \brak{\Longleftarrow} \)

	We need to show 2 things:
	\begin{enumerate}
		\item \(u\) is injective or \(\Ker(u) = 0\).
		\item \(\Img(u) = \Ker(v)\).
	\end{enumerate}
\end{proof}


\begin{proposition}{}{}
	Let
	\begin{align*}
		\xymatrix{
			0 \ar[r] & M' \ar[d]_-{f'} \ar[r]^-{u} & M
			\ar[d]_-{f} \ar[r]^-{v} & M''
			\ar[d]^-{f''} \ar[r] & 0 \\
			0 \ar[r] & N' \ar[r]_-{u'} & N
			\ar[r]_-{v'} & N'' \ar[r] & 0
		}
	\end{align*}
	be a commutative diagram of \(A\)-modules and homomorphisms,
	with the rows exact.

	Then, there exists an exact sequence
	\begin{align*}
		0 \xrightarrow[\quad]{} \Ker(f')
		\xrightarrow[\quad]{\bar{u}} \Ker(f)
		\xrightarrow[\quad]{\bar{v}} \Ker(f'')
		\xrightarrow[\quad]{d} \Coker(f')
		\xrightarrow[\quad]{\bar{u'}} \Coker(f)
		\xrightarrow[\quad]{\bar{v'}} \Coker(f'')
		\xrightarrow[\quad]{} 0
	\end{align*}
	in which \(\bar{u}, \bar{v}\) are restrictions of \(u, v\)
	and \(\bar{u'}, \bar{v'}\) are induced by \(u', v'\).

	The boundary homomorphism \(d\) is defined as
	\[
		\text{If } x'' \in \Ker(f'')
	\]
\end{proposition}


\subsection{Additive Functions over Modules}

\begin{defn}{Additive Functions}{}
	Let \(\sC\) be a class of \(A\)-modules and let \(\lambda\) be a function
	on \(\sC\) with values in \(\ZZ\).

	The function \(\lambda\) is said to be \emph{additive} if for
	every short exact sequence
	\[
		0 \xrightarrow[\quad\quad]{}
		M' \xrightarrow[\quad\quad]{u}
		M \xrightarrow[\quad\quad]{v}
		M'' \xrightarrow[\quad\quad]{} 0
	\]
	in \(\sC\), we have
	\[
		\lambda(M') - \lambda(M) + \lambda(M'') = 0
	\]
\end{defn}

\begin{proposition}{}{}
	Let
	\[
		0 \xrightarrow[\quad\quad]{} M_0 \xrightarrow[\quad\quad]{}
		M_1 \xrightarrow[\quad\quad]{} \cdots
		\xrightarrow[\quad\quad]{} M_n \xrightarrow[\quad\quad]{} 0
	\]
	be an exact sequence of \(A\)-modules in which all the modules
	and the kernels of the homomorphisms belong to \(\sC\).

	Then, for any additive function \(\lambda\) on \(\sC\), we have
	\[
		\sum_{i=0}^n (-1)^i \lambda(M_i)
	\]
\end{proposition}


\pagebreak
\section{Tensor Product of Modules}

\begin{defn}{\(A\)-bilinear}{}
	Let \(M, N, P\) be three \(A\)-modules.
	A mapping
	\[
	f \colon M \times N \to P
	\]
	is said to be \(A\)-\emph{bilinear} if

	for each \(x \in M\), the mapping
	\(
	y \mapsto f(x, y)
	\)
	of \(N\) into \(P\) is \(A\)-linear and

	for each \(y \in N\), the mapping
	\(
	x \mapsto f(x, y)
	\)
	of \(M\) into \(P\) is \(A\)-linear.
\end{defn}

We will construct an \(A\)-module \(T\), called the Tensor product of
\(M\) and \(N\), with the property that the \(A\)-bilinear mappings
\(M \times N \to P\) are in a natural one-to-one correspondence with
the \(A\)-linear mappings \(T \to P\), for all \(A\)-modules \(P\).


\begin{proposition}{Existence of Tensor Product}{}
	Let \(M, N\) be \(A\)-modules.
	Then, there exists a pair \(\brak{T, g}\) consisting of an \(A\)-module
	\(T\) and an \(A\)-bilinear mapping \(g \colon M \times N \to T\),
	with the following property:

	Given any \(A\)-module \(P\) and any \(A\)-bilinear mapping
	\[
		f \colon M \times N \to P
	\]
	there exists a unique \(A\)-linear mapping
	\[
		f' \colon T \to P
	\]
	such that
	\[
		f = f' \circ g
	\]
	(in other words, every bilinear function on \(M \times N\) factors
	through \(T\)).
\end{proposition}

\begin{proof}{}{}
	Let \(C\) denote the free \(A\)-module \(A^{\brak{M \times N}}\).

	The elements of \(C\) are formal linear combinations of elements
	of \(M \times N\) with coefficients in \(A\).
	That is, they are expressions of the form
	\[
		c \in C \implies c =
		\sum_{i=1}^n a_i \cdot (m_i, n_i)
		\quad\quad \Bigbrak{a_i \in A, (x_i, y_i) \in M \times N}
	\]
	Let \(D\) be a submodule of \(C\) generated by all elements of \(C\)
	of the following types
	\begin{gather*}
		\brak{x + x', y} - \brak{x, y} - \brak{x', y} \\
		\brak{x, y + y'} - \brak{x, y} - \brak{x, y'} \\
		\brak{ax, y} - a \cdot \brak{x, y} \\
		\brak{x, ay} - a \cdot \brak{x, y}
	\end{gather*}

	Now, let us define
	\[
		T = \bigslant{C}{D}
	\]
	For each basis element \(\brak{x, y}\) of \(C\), let
	\(x \otimes y\) denote its image in \(T\).

	Then, \(T\) is an \(A\)-module generated by the elements of the form
	\(x \otimes y\), for \(x \in M\) and \(y \in N\).

	Also, from our definitions, we have
	\begin{gather*}
		\brak{x + x'} \otimes y' = x \otimes y + x' \otimes y \\
		x \otimes \brak{y + y'} = x \otimes y + x \otimes y' \\
		\brak{ax} \otimes y = a \brak{x \otimes y} = x \otimes \brak{ay}
	\end{gather*}

	Equivalently, the mapping
	\[
		g \colon M \times N \to T
	\]
	denoted by
	\[
		g(x, y) = x \otimes y
	\]
	is \(A\)-bilinear.

	Any map \(f\) of \(M \times N\) into an \(A\)-module \(P\) extends
	by linearity to an \(A\)-module homomorphism
	\[
		f' \colon C \to P
	\]
	Suppose in particular that \(f\) is \(A\)-bilinear.
	Then, from the definitions, \(f'\) vanishes on all the generators
	of \(D\) and hence, on the whole of \(D\), and therefore induces
	a well-defined \(A\)-linear mapping \(f' \colon T \to P\)
	such that \(f'(x \otimes y) = f(x, y)\).

	The map \(f'\) is uniquely determined by the proposition and hence,
	satisfies the conditions of the proposition.
\end{proof}

\begin{proposition}{Uniqueness of Tensor Product}{}
	The tensor product of \(M\) and \(N\) is unique up to isomorphism.
	That is, if \(\brak{T, g}\) and \(\brak{T', g'}\) are two pairs
	as in the proposition, then there exists a unique isomorphism
	\[
		j \colon T \to T' \qstq j \circ g = g'
	\]
\end{proposition}

\begin{proof}
	Replacing \((P, f)\) with \((T', g')\), we get a unique
	\[
		j' \colon T' \to T
	\]
	such that
	\[
		g' = j' \circ g
	\]
	Each of the compositions \(j' \circ j\) and \(j \circ j'\) must
	be identity and hence, \(j\) is an isomorphism.
\end{proof}


\begin{note}
	The notation \(x \otimes y\) is inherently ambiguous unless we
	specify the tensor product to which it belongs.

	Suppose \(M', N'\) be submodules of \(M, N\) respectively.
	Then, it can happen that \(x \otimes y\) as an element of
	\(M \times N\) is zero whilst \(x \otimes y\) as an element of
	\(M' \times N'\) is non-zero.

	An example is the case where
	\[
		A = \ZZ \quad M = \ZZ, N = \bigslant{\ZZ}{2\ZZ}
		\quad M' = 2\ZZ, N' = \ZZ
	\]
	Here, the element \(2 \otimes 1\) is zero as an element of
	\(\ZZ \times \bigslant{\ZZ}{2\ZZ}\)
	but non-zero as an element of \(2\ZZ \times \bigslant{\ZZ}{2\ZZ}\).
\end{note}

\begin{defn}{Tensor Product}{}
	The module thus constructed is called the Tensor product of \(M\)
	and \(N\), and is denoted by \(M \otimes_A N\) or
	\(M \otimes N\) when there is no ambiguity.
\end{defn}

If \((x_i)_{i \in I}\) and \((y_j)_{j \in J}\) are two families
of generators of \(M\) and \(N\) respectively, then
\[
	(x_i \otimes y_j)_{i \in I, j \in J}
\]
is a family of generators of \(M \otimes N\).


\begin{corollary}{}{}
	Let \(x_i \in M\) and \(y_i \in N\) be such that
	\[
		\sum x_i \otimes y_i = 0 \quad\quad \text{in } M \otimes N
	\]
	Then, there exist finitely generated submodules \(M_0 \normsg M\)
	and \(N_0 \normsg N\) such that
	\[
		\sum x_i \otimes y_i = 0 \quad\quad \text{in } M_0 \otimes N_0
	\]
\end{corollary}

\begin{proof}
	In the existence proof, if we had \(\sum x_i \otimes y_i = 0\)
	in \(M \otimes N\), then we would have
\end{proof}

We will never need to use the construction of the tensor product but
only use the Bilinear property of the tensor product.


\subsection{Tensor Product of Multilinear Mappings}

Instead of starting with bilinear mappings, we could have started with
multilinear mappings and then constructed the tensor product.
\[
	f \colon M_1 \times M_2 \times \cdots \times M_r \to P
\]
defined in the same way and construct the tensor product
\[
	T \coloneqq M_1 \otimes M_2 \otimes \cdots \otimes M_r
\]
generated by the elements
\[
	x_1 \otimes x_1 \otimes \cdots \otimes x_r
	\quad \brak{x_i \in M_i, 1 \leq i \leq r}
\]


\begin{defn}{Tensor Product of Multilinear Mappings}{}
	Let \(M_1, \cdots, M_r\) be \(A\)-modules.
	Then there exists a pair \(\brak{T, g}\) as in the proposition
	consisting of an \(A\)-module \(T\) and an \(A\)-multilinear mapping
	\[
		g \colon M_1 \times M_2 \times \cdots \times M_r \to T
	\]
	with the following property:

	Given any \(A\)-module \(P\) and any \(A\)-multilinear mapping
	\[
		f \colon M_1 \times M_2 \times \cdots \times M_r \to P
	\]
	there exists a unique \(A\)-linear mapping
	\[
		f' \colon T \to P
	\]
	such that
	\[
		f' \circ g = f
	\]
	Moreover, if \(\brak{T, g}\) and \(\brak{T', g'}\) are two such
	pairs, then there exists a unique isomorphism
	\[
		j \colon T \to T' \qstq j \circ g = g'
	\]
	The module \(T\) is called the Tensor product of \(M_1, \cdots, M_r\)
	and is denoted by
	\[
		T = M_1 \otimes_A M_2 \otimes_A \cdots \otimes_A M_r
	\]
\end{defn}


\subsection{Properties of Tensor Product}

\begin{proposition}{}{}
	Let \(M, N, P\) be \(A\)-modules.
	Then, there exist unique isomorphisms
	\begin{enumerate}
		\item \(M \otimes N \to N \otimes M\)
		\item \(\brak{M \otimes N} \otimes P \to M \otimes (N \otimes P)
			\to M \otimes N \otimes P\)
		\item \(\brak{M \oplus N} \otimes P \to \brak{M \otimes P}
			\oplus \brak{N \otimes P}\)
		\item \(A \otimes M \to M\)
	\end{enumerate}
\end{proposition}

\begin{proof} \ \\
	The point in each case is to show that the obvious isomorphisms
	are well-defined using the definition of the tensor product.
	\begin{enumerate}
		\item Let us define our homomorphism as
		\begin{align*}
			f \colon M \otimes N &\to N \otimes M \\
			x \otimes y &\mapsto y \otimes x
		\end{align*}
		Consider the \(R\)-(multi)linear map
		\begin{align*}
			g \colon M \times N &\to P = N \otimes M \\
			(x, y) &\mapsto y \otimes x
		\end{align*}
		Now, \(g\) induces an \(R\)-(multi)linear maps
		\[
			M \times N \xrightarrow[\quad]{g_1} M \otimes N
			\xrightarrow[\quad]{f} P = N \otimes M
		\]
		where
		\[
			f(x \otimes y) = y \otimes x
		\]
		and hence, \(f\) is well-defined.

		To prove that \(f\) is an isomorphism, let us construct \(f\inv\)
		from \(N \otimes M \to M \otimes N\) as the natural inverse of
		\(f\) and it is well-defined by symmetry.


		\item
	\end{enumerate}
\end{proof}


\begin{exercise}{}{}
	Let \(A, B\) be rings and let \(M\) be an \(A\)-module, \(P\) be
	a \(B\)-module and \(N\) an \(\brak{A, B}\)-bimodule.

	That is, \(N\) is simultaneously an \(A\) and \(B\) module and the two
	structures are compatible in the sense that
	\[
		a(xb) = (ax)b \ \forall\ a \in A, b \in B, x \in N
	\]

	Then, \(M \otimes_A N\) is naturally a \(B\)-module and
	\(N \otimes_B P\) is naturally an \(A\)-module and we have
	\[
		\brak{M \otimes_A N} \otimes_B P \cong M \otimes_A
			\brak{N \otimes_B P}
	\]
\end{exercise}

\subsection{Tensors of Homomorphisms}

Suppose we have \(A\)-linear maps
\[
	f \colon M \to M', \quad g \colon N \to N'
\]
Define
\[
	h \colon M \times N \to M' \otimes N'
	\quad\quad h(x, y) = f(x) \otimes g(y)
\]
It can be easily checked that \(h\) is an \(A\)-bilinear map
and hence, induces a homomorphism
\[
	f \otimes g \colon M \otimes N \to M' \otimes N'
	\quad\quad (f \otimes g)(x \otimes y) = f(x) \otimes g(y)
\]
If we have
\[
	f' \colon M' \to M'', \quad g' \colon N' \to N''
\]
then clearly, the homomorphisms
\((f' \circ f) \otimes (g' \circ g)\) and
\((f' \otimes g') \circ (f \otimes g)\) agree on all elements
of the form \(x \otimes y\) in \(M \otimes N\).
Since these elements generate \(M \otimes N\) as an \(A\)-module,
\[
	(f' \circ f) \otimes (g' \circ g) = (f' \otimes g') \circ (f \otimes g)
\]


\pagebreak

\section{Restriction and Extension of scalars}

Let
\[
	f \colon A \to B
\]
be a ring homomorphism.

\subsection{Restriction of Scalars}
If we have a \(B\)-module \(N\), we can view it as an \(A\)-module
by defining the action of \(a \in A\) on \(x \in N\) as
\[
	a \cdot x \coloneqq f(a) \cdot x
\]
The \(A\)-module formed is said to be obtained from \(N\) by
\emph{restricting the scalars}.

\begin{proposition}{Restriction of Scalars preserves finitely generatedness}{}
	Suppose \(N\) is finitely generated as a \(B\)-module and let
	\[
		f \colon A \to B
	\]
	be a ring homomorphism.
	Then, \(N\) is finitely generated as an \(A\)-module.
\end{proposition}
\begin{proof}
	Let \(y_1, \cdots, y_n\) generate \(N\) over \(B\) and let \(x_1,
	\cdots, x_n\) generate \(B\) over \(A\).
	Then, \(x_i y_j\) generate \(N\) over \(A\).
\end{proof}

\subsection{Extension of Scalars}
Now, let \(M\) be an \(A\)-module.
Since we have seen that \(B\) can be regarded as an \(A\)-module,
we can form an \(A\)-module
\[
	M_B \coloneqq B \otimes_A M
\]
In fact, \(M_B\) has a natural \(B\)-module structure given by
\[
	b \cdot (b' \otimes x) = b b' \otimes x
	\quad \forall\ b, b' \in B, x \in M
\]
The \(B\)-module \(M_B\) is said to be obtained from \(M\) by
\emph{extension of scalars}.

\begin{proposition}{Extension of Scalars preserves finitely generatedness}{}
	Suppose \(M\) is finitely generated as an \(A\)-module and let
	\[
		f \colon A \to B
	\]
	be a ring homomorphism.
	Then, \(M_B\) is finitely generated as a \(B\)-module.
\end{proposition}

\begin{proof}
	If \(x_1, \cdots, x_n\) generate \(M\) over \(A\), then
	\(1 \otimes x_i\) generate \(M_B\) over \(B\).
\end{proof}

\pagebreak

\section{Exactness properties of tensor product}

Let
\[
	f \colon M \times N \to P
\]
be a \(A\)-bilinear map.

For each \(x \in M\), the mapping
\begin{align*}
	\phi_x \colon N &\to P \\
	y &\mapsto f(x, y)
\end{align*}
is a \(A\)-linear map from \(N\) to \(P\) and hence, \(f\)
gives rise to a \(A\)-linear map from \(M \to \Hom_A(N, P)\).
\begin{align*}
	\phi \colon M &\to \Hom_A(N, P) \\
	x &\mapsto \phi_x \\
	(x, y) &\mapsto \phi(x)(y)
\end{align*}
and is bilinear in \(x \in M\) and \(y \in N\).

Hence, the set \(S\) of all \(A\)-bilinear mappings from \(M \times N\)
to \(P\) is in natural one-to-one correspondence with
\[
	\Hom_A(M, \Hom_A(N, P))
\]
On the other hand, \(S\) is in one-to-one correspondence with
\[
	\Hom_A(M \otimes_A N, P)
\]
by the universal property of the tensor product.

Hence, we have a canonical isomorphism
\[
	\boxed{\Hom_A(M \otimes_A N, P) \cong \Hom_A(M, \Hom_A(N, P))}
\]

\begin{proposition}{Right Exactness of Tensor Product}{}
	Let
	\[
		M' \xrightarrow[\quad\quad]{f} M \xrightarrow[\quad\quad]{g} M''
		\xrightarrow[\quad\quad]{} 0
	\]
	be an exact sequence of \(A\)-modules and \(f\) and \(g\) are
	\(A\)-linear maps.
	Let \(N\) be an \(A\)-module.

	Then, the sequence
	\[
		M' \otimes_A N \xrightarrow[\quad\quad]{f \otimes_A 1} M \otimes_A N
		\xrightarrow[\quad\quad]{g \otimes_A 1} M'' \otimes_A N
		\xrightarrow[\quad\quad]{} 0
	\]
	where \(1\) denotes the identity map on \(N\) is exact.
\end{proposition}