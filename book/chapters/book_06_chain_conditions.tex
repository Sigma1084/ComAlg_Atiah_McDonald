\chapter{Chain Conditions}  \label{ch:chain_conditions}

This chapter mainly focuses on imposing finiteness-conditions.
The most convinient way is in the form of chain conditions.
These apply to both rings and modules.
In this chapter, we consider the case of modules. \\


\section{Definitions}

\begin{defn}{Ascending Chain Condition(ACC)}{}
	Let \(\Sigma\) be a set partially ordered by a relation \(\leq\).

	We say that \(\Sigma\) satisfies the ascending chain condition w.r.t.
	\(\leq\) if every increasing sequence
	\[
		x_1 \leq x_2 \leq \cdots \leq x_n \leq \cdots
	\]
	in \(\Sigma\) is stationary.
	That is, \(\exists\ n \in \NN\) such that
	\[
		x_m = x_n \ \forall\ m \geq n
	\]
\end{defn}

\begin{defn}{Descending Chain Condition(DCC)}{}
	Let \(\Sigma\) be a set partially ordered by a relation \(\leq\).

	We say that \(\Sigma\) satisfies the descending chain condition w.r.t.
	\(\leq\) if every decreasing sequence
	\[
		x_1 \geq x_2 \geq \cdots \geq x_n \geq \cdots
	\]
	in \(\Sigma\) is stationary.
	That is, \(\exists\ n \in \NN\) such that
	\[
		x_m = x_n \ \forall\ m \geq n
	\]
\end{defn}

\begin{defn}{Maximal and Minimal conditions}{}
	The Maximal condition is equivalent to the ACC and states that
	every non-empty subset of \(\Sigma\) has a maximal element.

	The Minimal condition is equivalent to the DCC and states that
	every non-empty subset of \(\Sigma\) has a minimal element.
\end{defn}

\begin{definition}{Noetherian (after Emmy Noether)}{}{}
	A ring \(A\) (or a module \(M\) over a ring \(A\)) is said to be
	\emph{Noetherian} if it satisfies the ascending chain condition
	w.r.t.\ the inclusion relation over the set of all ideals of \(A\)
	(or the set of all submodules of \(M\)).
\end{definition}

\begin{definition}{Artinian (after Emil Artin)}{}{}
	A ring \(A\) (or a module \(M\) over a ring \(A\)) is said to be
	\emph{Artinian} if it satisfies the descending chain condition
	w.r.t.\ the inclusion relation over the set of all ideals of \(A\)
	(or the set of all submodules of \(M\)).
\end{definition}


\section{Implications}

\begin{proposition}{}{}
	The following conditions on \(\Sigma\) are equivalent:
	\begin{enumerate}
		\item Every increasing sequence
		\[
			x_1 \leq x_2 \leq \cdots \leq x_n
		\]
		in \(\Sigma\) is stationary.
		\item Every non-empty subset of \(\Sigma\) has a maximal element.
	\end{enumerate}
\end{proposition}

\begin{proof} \

	\(\brak{1 \implies 2}\)
	Proof by contradiction.
	We can construct an ascending chain of elements if 2 is false.

	\(\brak{2 \implies 1}\)
	The set \((x_m)_{m \geq 1}\) has a maximal element, say \(x_n\).
\end{proof}

\begin{example}{}{}
	\begin{enumerate}
		\item A finite abelian group (as a \(Z\)-module) is both Noetherian
		and Artinian.
		\item The ring \(Z\) as a \(Z\)-module is Noetherian but not Artinian.
	\end{enumerate}
\end{example}

\begin{proposition}{}{}
	\(M\) is a Noetherian \(A\)-module \(\iff\) Every submodule
	of \(M\) is finitely generated.
\end{proposition}

\begin{proof} \

	\(\brak{\Longrightarrow}\)
	Consider some submodule \(N\) of \(M\).
	We use the maximal condition to prove the result.

	Suppose \(\Sigma\) be the set of all finitely generated submodules
	of \(N\).
	\(\Sigma\) is non-empty since \(0 \in \Sigma\).

	Then \(\Sigma\) is partially ordered by the inclusion relation.
	By the maximal condition, \(\Sigma\) has a maximal element.

	Suppose \(N_0\) be a maximal element of \(\Sigma\).
	We claim that \(N_0 = N\).
	Otherwise, \(\exists\ x \in N \st x \notin N_0\).

	Then \(N_0 + Ax\) is a finitely generated submodule of \(N\)
	which strictly contains \(N_0\) and hence, contradicts the maximality
	of \(N_0\).
	Thus, \(N_0 = N\) and hence, \(N\) is finitely generated. \\

	\(\brak{\Longleftarrow}\)
	Consider some ascending chain of submodules of \(M\).
	\[
		N_1 \subseteq N_2 \subseteq \cdots \subseteq N_n \subseteq \cdots
	\]
	Consider \(N = \bigcup_{n=1}^\infty N_i\) which is finitely generated.

	Suppose \(N\) is generated by \(\{x_1, \ldots, x_k\}\).
	Say \(x_i \in N_{n_i}\).

	Let \(n = \max_{i=1}^r n_i\) and we can notice that the chain is
	stationary after \(n\) since
	\[
		N_n = \bigcup_{i=1}^{\infty} N_i \qedhere
	\]
\end{proof}


\begin{theorem}{}{}
	Consider the SES of \(A\)-modules
	\[
		0 \xrightarrow[\quad\quad]{} M_1 \xrightarrow[\quad\quad]{\alpha}
		M_2 \xrightarrow[\quad\quad]{\beta} M_3 \xrightarrow[\quad\quad]{} 0
	\]
	Then,
	\begin{align*}
		M_2 \text{ is Noetherian} &\iff M_1, M_3 \text{ are Noetherian} \\
		M_2 \text{ is Artinian} &\iff M_1, M_3 \text{ are Artinian}
	\end{align*}
\end{theorem}

\begin{proof}
	We will prove the Neotherian case.
	The Artinian case is similar.

	\(\brak{\Longrightarrow}\)

	Suppose \(M_2\) is Noetherian.

	Let
	\[
		N_1 \subsetneq N_2 \subsetneq \dots \subsetneq N_n \subsetneq \dots
	\]
	be some strictly ascending chain of submodules of \(M_1\).

	Then, we get
	\[
		\alpha(N_1) \subsetneq \alpha(N_2) \subsetneq \dots
		\subsetneq \alpha(N_n) \subsetneq \dots
	\]
	which is a strictly ascending chain of submodules of \(M_2\).
	We have a strict containment since \(\alpha\) is an inclusion.

	This can not happen since \(M_2\) is satisfies ACC.\

	Thus, \(M_1\) is Noetherian.

	Now, consider a strictly ascending chain of submodules of \(M_3\)
	\[
		L_1 \subsetneq L_2 \subsetneq \dots \subsetneq L_n \subsetneq \dots
	\]

	Then, we get
	\[
		\beta\inv(L_1) \subsetneq \beta\inv(L_2) \subsetneq \dots
		\subsetneq \beta\inv(L_n) \subsetneq \dots
	\]
	which is a strictly ascending chain of submodules of \(M_2\).
	We have a strict containment since \(\beta\) is a surjection.

	This can not happen since \(M_2\) is satisfies ACC.\
	Thus, \(M_3\) is Noetherian. \\

	\(\brak{\Longleftarrow}\)

	Suppose \(M_1\) and \(M_3\) are Noetherian.

	Consider a strictly ascending chain of submodules of \(M_2\)
	\[
		N_1 \subsetneq N_2 \subsetneq \dots \subsetneq N_n \subsetneq \dots
		\qedhere
	\]
	% TODO: Finish this proof
\end{proof}


\begin{corollary}{}{}
	If \(M_i \ (1 \leq i \leq n)\) are Neotherian (resp.\ Artinian), then
	so is \(M_1 \oplus M_2 \oplus \cdots \oplus M_n\).
\end{corollary}

\begin{proof}
	We apply induction and the proposition on the exact sequence
	\[
		0 \xrightarrow[\quad\quad]{} M_n \xrightarrow[\quad\quad]{}
		\bigoplus_{i=1}^n M_i \xrightarrow[\quad\quad]{}
		\bigoplus_{i=1}^{n-1} M_i \xrightarrow[\quad\quad]{} 0 \qedhere
	\]
\end{proof}

\begin{lemma}{}{}
	Every ideal of a Noetherian (resp.\ Artinian) ring \(A\) is
	Noetherian (resp.\ Artinian).
\end{lemma}

\begin{proof}
	Consider the ideal\(\mfa\) of \(A\) as a \(A\)-module.
	Construct the SES
	\[
		0 \xrightarrow[\quad\quad]{} \mfa \xrightarrow[\quad\quad]{}
		A \xrightarrow[\quad\quad]{} \bigslant{A}{\mfa}
		\xrightarrow[\quad\quad]{} 0
	\]
	Then, by the proposition, \(\mfa\) is Noetherian (resp.\ Artinian).
\end{proof}


\begin{proposition}{}{}
	Let \(A\) be a Noetherian (resp.\ Artinian)
	ring and \(M\) be a finitely generated \(A\)-module.

	Then, \(M\) is Noetherian (resp.\ Artinian).
\end{proposition}

\begin{proof}
	Using the structure theorem of finitely generated \(A\)-modules,
	we can write \(M = \bigoplus_{i=1}^n \mfa_i\) where \(\mfa_i\) are
	ideals of \(A\) for some \(n \in \NN\).

	Then, by the lemma, \(\mfa_i\) are Noetherian (resp.\ Artinian)
	and using the corollary, we get that \(M\) is Noetherian.
\end{proof}


\section{Chain}

\begin{defn}{Chain or Composition series}{}
	A \emph{chain} or \emph{composition series} of submodules
	of a module \(M\) is a sequence
	\(M_i \ (0 \leq i \leq n)\) of submodules of \(M\) such that
	\[
		M = M_0 \supsetneq M_1 \supsetneq \cdots \supsetneq M_n = 0
	\]
	\(n\) is called the \emph{length} of the chain denoted by \(l(M)\).
\end{defn}

\begin{defn}{Maximal Chain}{}
	A composition series of \(M\) is a maximal chain, (one in which
	no extra submodules can be inserted) if
	\[
		\bigslant{M_{i-1}}{M_i} \text{ is simple} \quad \forall\
		i \in \{2, \dots, n\}
	\]
\end{defn}


\begin{proposition}{}{}
	Suppose \(M\) being an \(A\)-module has a composition series of
	length \(n\).

	Then every composition series of \(M\) has length \(n\) and every
	chain in \(M\) can be extended to a composition series.
\end{proposition}

\begin{proof}
	Let \(l(M)\) denote the least length of a composition series
	of a module \(M\).
	(We say \(l(M) = \infty\) if \(M\) has no composition series.) \\

	\begin{claim}{}{}
		\[
			N \subsetneq M \implies l(N) < l(M)
		\]
	\end{claim}
	Let \(M_i\) be a composition series of \(M\) of minimal length.
	Consider the submodules \(N_i = N \cap M_i\) of \(N\).

	We have
	\[
		\bigslant{N_{i-1}}{N_i} \subseteq \bigslant{M_{i-1}}{M_i}
	\]
	and the latter is a simple module and hence,
	\[
		\bigslant{N_{i-1}}{N_i} = 0 \implies N_{i-1} = N_i
		\OR \bigslant{M_{i-1}}{M_i}
	\]

	Hence, removing the repeated submodules, we get a composition series
	of \(N\) of length \(l(N) \leq l(M) \).

	If \(l(N) = l(M)\), we have \(N_i = M_i, N_{i-1} = M_{i-1}\) and so
	on till \(N_0 = M_0\) and hence \(N = M\) which is a contradiction. \\

	\begin{claim}{}{}
		Any chain in \(M\) has length at most \(l(M)\).
	\end{claim}
	Consider a chain
	\[
		M = M_0 \subsetneq M_1 \subsetneq \dots \subsetneq M_k = 0
	\]
	Using the previous claim, we have
	\[
		l(M_0) > l(M_1) > \dots > l(M_k) = 0
	\]
	and hence \(k \leq l(M)\)

	Hence, we can say that any composition series
	of \(M\) has length \(l(M)\). \\

	Now, consider a chain.
	If the length of the chain is less than \(l(M)\), it is not a maximal
	chain and hence, we can extend it to a composition series.

	Otherwise, we have a composition series of length \(l(M)\).
\end{proof}


\begin{proposition}{}{}
	\(M\) has a composition series \(\iff\) \(M\) satisfies both
	ACC and DCC.\
\end{proposition}

\begin{proof} \

	\(\brak{\Longrightarrow}\)
	Since \(M\) has a composition series, all chains in \(M\) are
	of finite length and hence, \(M\) satisfies ACC and DCC. \\

	\(\brak{\Longleftarrow}\)
	Construct a composition series of \(M\) as follows.

	Since \(M = M_0\) satisfies the maximum condition, it has a maximal
	submodule \(M_1 \subsetneq M_0\).

	Similarly, \(M_1\) has a maximal submodule \(M_2 \subsetneq M_1\)
	and so on.

	Thus, we get a descending chain of submodules of \(M\) and hence,
	must terminate at some point giving us a composition series.
\end{proof}


\begin{proposition}{}{}
	The length \(l(M)\) is an additive function on the class of all
	\(A\)-modules of finite length.

	That is,
	\[
		0 \xrightarrow[\quad\quad]{} N
		\xrightarrow[\quad\quad]{\alpha} M
		\xrightarrow[\quad\quad]{\beta} L
		\xrightarrow[\quad\quad]{} 0
	\]
	is an exact sequence of \(A\)-modules, then
	\begin{enumerate}
		\item \(l(M) < \infty \iff l(N) < \infty \AND l(L) < \infty\)
		\item \(l(M) = l(N) + l(L)\)
	\end{enumerate}
\end{proposition}

\begin{proof}
	1 is obvious from the fact that
	\[
		M \text{ is Noetherian (resp. Artinian)}
		\iff N \AND L \text{ is Noetherian (resp. Artinian)}
	\]

	Suppose
	\[
		N = N_0 \subsetneq N_1 \subsetneq \dots \subsetneq N_n = 0
	\]
	and
	\[
		L = L_0 \subsetneq L_1 \subsetneq \dots \subsetneq L_l = 0
	\]

	Since \(\alpha\) is an inclusion, we will have
	\[
		\img(\alpha) = \alpha(N_0) \subsetneq \alpha(N_1) \subsetneq \dots
		\subsetneq \alpha(N_n) = 0
	\]

	Also, since \(\beta\) is a surjection, we will have
	\[
		M = \beta\inv(L_0) \subsetneq \beta\inv(L_1) \subsetneq \dots
		\subsetneq \beta\inv(L_l) = \beta\inv(0) = \ker(\beta)
	\]

	Since \(\img(\alpha) = \ker(\beta)\), we have
	\[
		M = \beta\inv(L_0) \subsetneq \beta\inv(L_1) \subsetneq \dots
		\subsetneq \beta\inv(L_l) = \alpha(N_0) \subsetneq \alpha(N_1)
		\subsetneq \dots \subsetneq \alpha(N_n) = 0
	\]

	The length of this composition series is \(l(M) = l(N) + l(L)\).
\end{proof}


\begin{proposition}{}{}
	For \(k\)-vector spaces \(V\), the following conditions are equivalent.
	\begin{enumerate}
		\item Finite dimension
		\item Finite length
		\item Satisfy ACC
		\item Satisfy DCC
	\end{enumerate}
	Moreover, if any of the conditions is satisfied, then the length
	is equal to its dimension.
\end{proposition}

\begin{proof}
	Trivial.
\end{proof}


\begin{corollary}{}{}
	Let \(A\) be a ring in which the zero ideal is a product
	\[
		\mfm_1 \cdot \mfm_2 \cdots \mfm_n = 0
	\]
	(not necessarily distinct) finite number of maximal ideals.

	Then,
	\[
		A \text{ is Noetherian} \iff A \text{ is Artinian}
	\]
\end{corollary}

\begin{proof}
	Consider the chain of ideals of \(A\).
	\[
		A \supsetneq \mfm_1 \supseteq \mfm_1 \mfm_2 \supseteq \dots
		\supseteq \mfm_1 \mfm_2 \cdots \mfm_n = 0
	\]
	Now, each factor \(\bigslant{\mfm_1 \cdots \mfm_{i-1}}
	{\mfm_1 \cdots \mfm_i}\) is a vector space over field
	\(\bigslant{A}{\mfm_i}\).

	Hence, ACC \(\iff\) DCC for each factor and thus, for \(A\).
\end{proof}