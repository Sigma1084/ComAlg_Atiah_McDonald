\chapter{Discrete Valuation Rings and Dededekind Domains}
\label{ch:dvr_dd}

As it has been indicated before, algebraic number theory is one of the
historical sources of commutative algebra. \\

In this chapter, we specialize down to the case of Dedekind domains.
We deduce the unique factorization of ideals in Dedekind domains from
the general primary decomposition theorems.
Although a direct approach is of course possible one obtains more insight
this way into the precise context of number theory in commutative algebra. \\

Another important class of Dedekind domains occurs in connection with
non-singular algebraic curves.
In fact, the geometrical picture of the Dedekind condition is
non-singular of dimension one.

% TODO Understand
(Editors notes: Unclear paragraph) \\

The last chapter dealt with noetherian rings of dimension zero.

Here, we start by considering the next simplest case, namely Noetherian
\textit{integral domains} of dimension one.

That is, we consider Noetherian domains in which every non-zero prime
ideal is maximal.

\begin{proposition}{}{}
\label{prop:9.1}
	Let \(A\) be a Noetherian domain of dimension 1.
	Then, every non-zero ideal \(\mfa\) in \(A\) can be uniquely
	expressed as a product of primary ideals whose radicals are
	all distinct.
\end{proposition}
\begin{proof}
	Since \(A\) is Noetherian, \(\mfa\) has a minimal primary
	decomposition by 7.13,
	\[
		\mfa = \bigcap_{i=1}^n \mfq_i \quad
		\text{ where } \mfq_i \text{ are } \mfp_i\text{-primary and }
		\quad \mfp_i \neq \mfp_j \text{ for } i \neq j
	\]
	We have \(\dim A = 1\) and \(A\) is an integral domain.
	Then, each \(\mfp_i\) is maximal and hence, \(\mfp_i\) are distinct
	maximal ideals and are hence, pairwise coprime and therefore using
	1.10, we get
	\[
		\prod_{i=1}^n{\mfq_i} = \bigcap_{i=1}^n \mfq_i = \mfa
	\]
	For the uniqueness part, we use corollary~\ref{corollary:4_11} to
	conclude that \(\mfq_i\), being the isolated primary components
	of \(\mfa\), are unique.
\end{proof}


Let \(A\) be a Noetherian domain of dimension one in which every primary
ideal is a prime power.
Then, in such a ring, we will have unique factorization of ideals into
products of prime ideals.

Now, if we localize \(A\) with respect to a non-zero prime ideal \(\mfp\),
we get a local ring \(A_{\mfp}\) satisfying the same conditions as \(A\),
and therefore in \(A_{\mfp}\), every non-zero ideal is a power of a
maximal ideal.
Such rings can be characterized in other ways.


\section{Discrete Valuation Rings}

\begin{definition}{Discrete Valuation}{}{}
	Let \(K\) be a field.
	A \textbf{discrete valuation} on \(K\) is a mapping
	\[
		v \colon K^{\times} \to \ZZ  \quad\quad v \text{ is onto}
	\]
	where \(K^{\times} = K \setminus \{0\}\), the multiplicative group
	of \(K\), such that
	\begin{enumerate}
		\item \(v\) is a homomorphism.
		That is, \(v(xy) = v(x) + v(y)\) for all \(x, y \in K^{\times}\).
		\item \(v(x + y) \geq \min(v(x), v(y))\)

	\end{enumerate}
\end{definition}

We define discrete valuation rings as follows.
\begin{definition}{Discrete Valuation Ring}{}{}
	Let \(K\) be a field and \(v\) be a discrete valuation on \(K\).
	The set consisting of \(0\) together with all elements \(x \in K\)
	such that \(v(x) \geq 0\) is a subring of \(K\) called the
	\textbf{discrete valuation ring} of \(v\).

	It is sometimes convenient to extend \(v\) to all of \(K\) by
	defining \(v(0) = \infty\).
\end{definition}

The two standard examples are
\begin{example}{}{}
	\(K = \QQ\).
	Take a fixed prime \(p\) then any non-zero \(x \in \QQ\) can be
	written uniquely in the form \(p^a y\) where \(a \in \ZZ\) and
	both numerator and denominator of \(y\) are not divisible by \(p\).

	Define \(v_p(x) = a\).
	Then, \(v_p\) is a discrete valuation on \(\QQ\) and the discrete
	valuation ring of \(v_p\) is the local ring \(\ZZ_{(p)}\).
	\[
		\ZZ_{(p)} = \left\{ \frac{a}{b} \in \QQ \mid p \nmid b \right\}
	\]
\end{example}
\begin{proof}
	The above holds clearly since,
	\[
		v(xy) = v(p^a y p^b z) = v(p^{a+b} yz) = a + b = v(x) + v(y)
	\]
	and
	\[
		v(x) \geq 0 \implies x = p^a \frac{c}{d} \quad\quad
		\text{s.t. } c, d \nmid p \AND a \geq 0
		\implies x = \frac{c'}{d} \in \ZZ_{(p)}
	\]
	Clearly,
	\[
		x \in \ZZ_{(p)} \implies v(x) \geq 0
	\]
	since any factor of \(p\) can only occur in the numerator.
\end{proof}

\begin{example}{}{}
	\(K = k(x)\), where \(k\) is a field and \(x\) an indeterminate.

	Take a fixed irreducible polynomial \(f(x) \in k[x]\).
	Consider some \(g(x) \in k[x]\) and write
	\[
		g(x) = f(x)^a h(x) \quad\quad \text{where } h(x) \text{ is not
		divisible by } f(x)
	\]
	and define \(v_f\) on \(k(x)\) by
	\[
		v_f \colon k(x)^{\times} \to \ZZ
	\]
	defined by
	\[
		v_f (g(x)) = a
	\]
	where \(g(x) = f(x)^a h(x)\) as above.
	Then, \(v_f\) is a discrete valuation on \(k(x)\) and the discrete
	valuation ring of \(v_f\) is the local ring \(k[x]_{(f(x))}\).
\end{example}