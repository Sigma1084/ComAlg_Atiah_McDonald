\chapter{Rings and modules of fractions}
\label{ch:book_03_rings_and_modules_of_fractions}


\section{Introduction and Definition}
The formation of rings of fractions and the localization of rings
is a very important tool in commutative algebra.

They correspond to the algebro-geometric picture to concentrating attention
on an open set or near a point.

\subsection{Field of fractions over an integral domain}

The procedure by which one constructs \(\QQ\) from \(\ZZ\) can
be extended to any integral domain \(A\) and
produces the \emph{field} of fractions of \(A\).

Construct the equivalence relation \(\sim\) on the pairs of the
integral domain \(A\) by declaring
\[
	\brak{a, s} \sim \brak{b, t} \iff a \cdot t - b \cdot s = 0
\]

Reflexivity and symmetry are easy to check.

\begin{proof}[Proof for Transitivity] \

	Suppose we have \(\brak{a, s} \sim \brak{b, t}\) and
	\(\brak{b, t} \sim \brak{c, u}\).

	Then we need to prove that \(\brak{a, s} \sim \brak{c, u}\).

	We have
	\begin{align*}
		at - bs = 0 \AND bu - ct = 0 \\
		\implies atu - bsu = 0 \\
		\implies atu - bct = 0 \\
		\implies t \brak{au - bc} = 0
	\end{align*}
	Now, we know that \(t \neq 0\) and \(A\) is an integral domain.
	Therefore, \(au - bc = 0\) and hence
	\[
		\brak{a, s} \sim \brak{c, u}
	\]
\end{proof}

This works only if \(A\) is an integral domain since proving
that the relation is transitive involves cancelling.

That is, the fact that \(A\) has no non-trivial zero divisors is crucial.

Hence, we define the field of fractions of \(A\) as
\[
	\bigslant{A \times A^{\times}}{\sim}
\]
where \(A^{\times}\) is the set of non-zero elements of \(A\).

It is easy to check that this is a field.

\subsection{Localization of a ring at a multiplicative subset}

\begin{defn}{Multiplicatively Closed Subset}{}
	Let \(A\) be a ring and \(S\) be a subset of \(A\).

	Then \(S\) is said to be \emph{multiplicatively closed} if
	\(1 \in S\) and \(S\) is closed under multiplication.
\end{defn}

We can now define the localization of a ring at a multiplicative subset.
Let \(S\) be a multiplicatively closed subset of \(A\).

Define a relation
\[
	\sim \colon A \times S
\]
as follows
\[
	\brak{a, s} \sim \brak{b, t} \iff \brak{at - bs}u = 0
	\text{ for some } u \in S
\]
Clearly, this relation is reflexive and symmetric.

\begin{proof}[Proof for Transitivity] \

	Suppose we have \(\brak{a, s} \sim \brak{b, t}\) and
	\(\brak{b, t} \sim \brak{c, u}\).
	Then we need to prove that \(\brak{a, s} \sim \brak{c, u}\).

	For some \(v, w \in S\), we have
	\begin{align*}
		\brak{at - bs}v = 0 \AND \brak{bu - ct}w = 0 \\
		\brak{at - bs}vuw = 0 \AND \brak{bu - ct}wsv = 0 \\
		\implies \brak{atuvw - cstvw} = 0
		\implies \brak{au - cs}tvw = 0
	\end{align*}
	Since \(S\) is multiplicatively closed, \(tvw \in S\).
\end{proof}
Thus, \(\sim\) is an equivalence relation.

\begin{defn}{Localization of a ring}{}
	We define the localization of \(A\) at a multiplicatively closed
	subset \(S\) as
	\[
		S^{-1}A \coloneqq \bigslant{A \times S}{\sim}
	\]
	where \(\sim\) is the equivalence relation defined as
	\[
		\brak{a, s} \sim \brak{b, t} \iff \brak{at - bs}u = 0
		\text{ for some } u \in S
	\]
	The elements of \(S^{-1}A\) are denoted by \(\bigslant{a}{s}\)
	where \(a \in A\) and \(s \in S\).
\end{defn}


\subsection{Localization as a ring of fractions}

We can define addition and multiplication on \(S^{-1}A\) as follows
\begin{align*}
	\frac{a}{s} + \frac{b}{t} & = \frac{at + bs}{st} \\
	\frac{a}{s} \cdot \frac{b}{t} & = \frac{ab}{st}
\end{align*}

The identity element of \(S^{-1}A\) is \(\frac{1}{1}\).

\begin{claim}{}{}
	The addition and multiplication
	of the elements of \(S^{-1} A\) is well-defined.
\end{claim}

\begin{proof}
	We need to prove
	\[
		\frac{a}{s} + \frac{b}{t} = \frac{at + bs}{st}
		=\frac{a_1 t_1 + b_1 s_1}{s_1 t_1}
		\quad \text{ where } \frac{a_1}{s_1} = \frac{a}{s}
		\quad \text{ and } \frac{b_1}{t_1} = \frac{b}{t}
	\]
	We know that for some \(u, v \in S\), we have
	\[
		\brak{a_1 s - a s_1}u = 0 \AND
		\brak{b_1 t - b t_1}v = 0
	\]
	Simplifying, we have
	\begin{align*}
		\brak{a_1 s - a s_1} t t_1 u v = 0
		\AND \brak{b_1 t - b t_1} s s_1 u v &= 0 \\
		\brak{a_1 t_1 s t - a t s_1 t_1} u v = 0
		\AND \brak{b_1 s_1 s t - b s s_1 t_1} u v &= 0 \\
		\implies \Bigbrak{ \brak{a_1 t_1 + b_1 s_1} st -
		\brak{a t + b s} s_1 t_1} uv &= 0 \\
		\implies \boxed{
			\frac{a_1 t_1 + b_1 s_1}{s_1 t_1} = \frac{a t + b s}{s t}
		}
	\end{align*}

	Now, we need to prove that
	\[
		\frac{a}{s} \cdot \frac{b}{t} = \frac{ab}{st}
		= \frac{a_1 b_1}{s_1 t_1}
	\]

	Similarly,
	\begin{align*}
		\brak{a_1 s - a s_1}u = 0 \AND
		\brak{b_1 t - b t_1}v &= 0 \\
		\brak{a_1 s - a s_1} b_1 t u v = 0 \AND
		\brak{b_1 t - b t_1} a s_1 u v &= 0 \\
		\brak{a_1 b_1 s t - a b_1  s_1 t} u v = 0 \AND
		\brak{a b_1 s_1 t - a b s_1 t_1} u v &= 0 \\
		\implies \brak{a_1 b_1 st - a b s_1 t_1} uv &= 0 \\
		\implies \boxed{
			\frac{a_1 b_1}{s_1 t_1} = \frac{ab}{st}
		}
	\end{align*}
\end{proof}


\section{Some properties of localizations}

We will also have a ring homomorphism
\[
	f \colon A \to S^{-1}A \quad \text{ defined as }
	f(a) = \frac{a}{1}
\]
Note that this is not injective in general.

For example, consider \(A = \ZZ_6\) and \(S = \fbrak{1, 3}\).

We will have \(f(0) = \frac{0}{1}\) and
\(f(2) = \frac{2}{1} = \frac{6}{3} = \frac{0}{1} = f(0)\).

\begin{proposition}{Universal Property of Localization}{}
	Let \(g \colon A \to B\) be a ring homomorphism such that
	\(g(s)\) is a unit for all \(s \in S\).
	Then there exists a unique ring homomorphism
	\[
		h \colon S^{-1}A \to B
	\]
	such that \(g = h \circ f\).
\end{proposition}

\begin{proof}
	For the uniqueness, suppose \(h\) satisfies the above property.
	Then, we have
	\[
		g(a) = h \circ f(a) \quad \forall\ a \in A
		\implies h\brak{\frac{a}{1}} = g(a)
	\]
	For any \(s \in S\), we have
	\[
		h(s) \cdot h \brak{\frac{1}{s}} =
		h \brak{\frac{s}{s}} = h \brak{\frac{1}{1}} = g(1)
		\implies h \brak{\frac{1}{s}} = g(s)^{-1}
	\]
	Hence, we will have
	\[
		h\brak{\frac{a}{s}} = g(a) \cdot g(s)^{-1}
	\]
	which is uniquely determined by \(g\).


	Consider the map
	\[
		h \colon S^{-1}A \to B \quad \text{ defined as }
		h\left(\frac{a}{s}\right) = g(a) \cdot g(s)^{-1}
	\]
	We need to prove that this is well-defined.

	For the existence, consider \(\frac{a}{s} = \frac{a_1}{s_1}\).
	Then, we have
	\[
		\brak{a_1 s - a s_1}u = 0 \text{ for some } u \in S
	\]
	Applying \(g\), we have
	\[
		\brak{g(a_1) g(s) - g(a) g(s_1)} g(u) = 0
	\]
	Now, since \(g(u)\) is a unit, we have
	\[
		g(a_1) g(s) = g(a) g(s_1)
		\implies g(a) \cdot g(s)^{-1} = g(a_1) \cdot g(s_1)^{-1}
	\]
	and hence, \(h\) is well-defined.
\end{proof}


\begin{proposition}{}{}
	The ring \(S^{-1}A\) and the homomorphism
	\[
		f \colon A \to S^{-1}A \quad \text{ defined as }
		f(a) = \frac{a}{1}
	\]
	have the following properties:
	\begin{enumerate}
		\item \(s \in S \implies f(s)\) is a unit in \(S^{-1}A\)
		\item \(f(a) = 0 \implies as = 0\) for some \(s \in S\)
		\item Every element of \(S^{-1}A\) is of the form
			\(f(a) \cdot f(s)^{-1}\) for some \(a \in A\) and
			\(s \in S\)
	\end{enumerate}
\end{proposition}

\begin{proof} \
	\begin{enumerate}
		\item \(f(s) = \frac{s}{1}\) and when multiplied by
		\(\frac{1}{s}\), we get \(1\) which proves that
		\(f(s)\) is a unit.

		\item \(f(a) = 0 \implies f(a) = \frac{a}{1} = \frac{0}{1}
		\implies \brak{a \cdot 1 - 0 \cdot 1} s = 0\) for some \(s \in S\)
		which implies \(as = 0\).

		\item Clearly, any element of \(S\) is of the form \(\frac{a}{s}\)
		which is precisely \(f(a) \cdot f(s)^{-1}\).
	\end{enumerate}
\end{proof}

Conversely, these three conditions determine the ring \(S^{-1} A\)
up to isomorphism.

\begin{corollary}{}{}
	If \(S\) is a multiplicative subset of \(A\) and
	\(g \colon A \to B\) is a ring homomorphism such that
	\begin{enumerate}
		\item \(s \in S \implies g(s)\) is a unit in \(B\)
		\item \(g(a) = 0 \implies as = 0\) for some \(s \in S\)
		\item Every element of \(B\) is of the form
			\(g(a) \cdot g(s)^{-1}\) for some \(a \in A\) and \(s \in S\)
	\end{enumerate}
	Then, there exists a unique isomorphism
	\[
		h \colon S^{-1}A \to B
		\quad \text{ such that } \quad
		h \circ f = g
	\]
\end{corollary}

\begin{proof}
	Using 1 and the universal property of localization, we have
	a unique homomorphism
	\[
		h \colon S^{-1}A \to B
		\quad \text{ such that } \quad
		h \circ f = g
		\quad \text{ defined by } \quad
		h\brak{\frac{a}{s}} = g(a) \cdot g(s)^{-1}
	\]
	We now need to prove that this is an isomorphism.

	Using 3, notice that \(h\) is surjective.
	Consider the kernel of \(h\).
	\begin{align*}
		\frac{a}{s} \in \Ker h \implies g(a) \cdot g(s)^{-1} = 0
		\implies g(a) = 0
	\end{align*}
	Using 2, we have \(as = 0\) for some \(s \in S\)
	which implies \(\frac{a}{s} = \frac{0}{1}\).
	Hence, \(\Ker h = \fbrak{0}\) and \(h\) is injective.
\end{proof}


\pagebreak


\subsection{Localization at a prime ideal and an element}

Let \(\mfp\) be a prime ideal of \(A\).
Then \(S = A \setminus \mfp\) is a multiplicatively closed
subset of \(A\).
In fact,

\begin{claim}{}{}
	For a ring \(A\),
	\[
		A \setminus \mfp \text{ is multiplicatively closed }
		\iff \mfp \text{ is a prime ideal }
	\]
\end{claim}

\begin{proof}
	Both the statements imply and are implied by
	\[
		a \notin \mfp \text{ and } b \notin \mfp
		\implies ab \notin \mfp \quad
		\forall\ a, b \in A
		\qedhere
	\]
\end{proof}

\begin{defn}{Localization at a prime ideal}{}
	Let \(\mfp\) be a prime ideal of \(A\).
	We define
	\[
		\boxed{
			A_{\mfp} \coloneqq S^{-1}A
			\quad \text{where} \quad
			S = A \setminus \mfp
		}
	\]
\end{defn}


\begin{claim}{}{}
	The set of the elements of \(A_{\mfp}\) of the form
	\[
		\mfm \coloneqq \fbrak{\textstyle{\frac{a}{s}} \mid a \in \mfp, s \in S}
	\]
	is a maximal ideal of \(A_{\mfp}\).
\end{claim}
\begin{proof}
	Suppose we have an element \(\frac{b}{t}\) such that
	\(\frac{b}{t} \notin \mfm\).
	\[
		\textstyle{\frac{b}{t}} \notin \mfm
		\implies b \notin \mfp
		\implies b \in S
	\]
	Now, this implies that \(\frac{b}{t}\) is a unit and hence,
	\(\mfm\) is a maximal ideal.
\end{proof}

It proves that if we have an ideal \(\mfa\) that is not contained in
\(\mfm\), then \(\mfa\) contains a unit.

Now, this means \(\mfm\) is the only maximal ideal of \(A_{\mfp}\).

\[
	\boxed{
		A_{\mfp} \text{ is a local ring}
	}
\]
The process of passing from \(A\) to \(A_{\mfp}\) is called
\emph{localization at a prime ideal}.

In case of \(\ZZ\), \(\ZZ_{(p)}\) is the ring of rational numbers
with denominators co-prime to \(p\).

\begin{defn}{Localization at an element}{}
	Let \(f \in A\).
	We define
	\[
		\boxed{
			A_f \coloneqq S^{-1}A
			\quad \text{where} \quad
			S = \fbrak{f^n}_{n \geq 0}
		}
	\]
\end{defn}
The elements of \(A_f\) are of the form
\[
	\frac{a}{f^n}
	\quad \text{ where } \quad
	a \in A \text{ and } n \geq 0
\]

\section{Localization using Modules}
The construction of \(S^{-1}A\) can be carried out through an \(A\)-module
\(M\) in place of the ring \(A\).

Define the relation
\[
	\sim \colon M \times S
\]
as follows
\[
	\brak{m, s} \sim \brak{m', s'} \iff \exists\ t \in S \text{ such that }
	\brak{sm' - s'm}t = 0
\]

As before, this is an equivalence relation and we define
\[
	S^{-1}M \coloneqq \bigslant{\brak{M \times S}}{\sim}
\]
The elements of \(S^{-1}M\) are denoted by \(\frac{m}{s}\) and
\[
	S^{-1}M \text{ is a } S^{-1}A\text{-module}
\]
with the action
\[
	\frac{a}{s} \cdot \frac{m}{t} = \frac{am}{st}
\]
When we have a prime ideal \(\mfp\) of \(A\), we can consider
\[
	M_{\mfp} \coloneqq S^{-1}M \quad \text{ where } \quad
	S = A \setminus \mfp
\]
and
\[
	M_f \coloneqq S^{-1}M \quad \text{ where } \quad
	S = \fbrak{f^n}_{n \geq 0}
\]

Also, when we have a \(A\)-module homomorphism
\[
	u \colon M \to N
\]
This gives rise to an \(S^{-1}A\)-module homomorphism
\[
	S^{-1}u \colon S^{-1}M \to S^{-1}N
\]
defined by
\[
	S^{-1}u \brak{\frac{m}{s}} = \frac{u(m)}{s}
\]
We also have
\[
	S^{-1} \brak{v \circ u} = \brak{S^{-1}v} \circ \brak{S^{-1}u}
\]


\begin{proposition}{}{}
	The operation \(S^{-1}\) is exact.
	That is,
	\[
		M' \xrightarrow[\quad\quad]{f} M \xrightarrow[\quad\quad]{g} M''
		\text{ is exact at } M \qquad\implies\qquad
		S^{-1}M' \xrightarrow[\quad\quad]{S^{-1}f} S^{-1}M
		\xrightarrow[\quad\quad]{S^{-1}g} S^{-1}M'' \text{ is exact at }
		S^{-1}M
	\]
\end{proposition}
\begin{proof}
	We have
	\[
		g \circ f = 0 \implies S^{-1}g \circ S^{-1}f
		= S^{-1}\brak{g \circ f} = 0
	\]
	Hence, we have
	\[
		\Img \brak{S^{-1}f} \subseteq \Ker \brak{S^{-1}g}
	\]
	To prove the reserve inclusion, suppose we have an element
	\(\frac{m}{s} \in \Ker \brak{S^{-1}g}\).
	Then, we have
	\[
		S^{-1}g \brak{\frac{m}{s}} = \frac{g(m)}{s} = 0
		\implies \exists\ t \in S \text{ such that }
		tg(m) = 0 \implies tg(m) = g(tm) = 0
	\]
	Hence, we have \(tm \in \Ker g = \Img f\).
	\[
		\therefore \exists\ m' \in M' \text{ such that }
		f(m') = tm \implies f\brak{\frac{m'}{st}} = \frac{tm}{st}
		= \frac{m}{s}
	\]
	This proves
	\[
		\Ker \brak{S^{-1}g} \subseteq \Img \brak{S^{-1}f}
	\]
	and we are done.
\end{proof}

The above proposition proves that if \(M' \normsg M\) is a submodule,
then \(S^{-1}M' \normsg S^{-1}M\).

\begin{corollary}{}{}
	Formation of fractions commutes with formation of finite
	sums, finite intersections and quotients.
	Precisely, if \(N, P\) are submodules of \(M\), then
	\begin{enumerate}
		\item \(S^{-1}\brak{N + P} = S^{-1}N + S^{-1}P\)
		\item \(S^{-1}\brak{N \cap P} = S^{-1}N \cap S^{-1}P\)
		\item The \(S^{-1}A\)-modules are isomorphic
		\[
			S^{-1}\brak{\bigslant{M}{N}} \cong
			\bigslant{\brak{S^{-1}M}}{\brak{S^{-1}N}}
		\]
	\end{enumerate}
\end{corollary}

\begin{proof} \

	\begin{enumerate}
		\item It is easy to check that
		\[
			\frac{n + p}{s} \in S^{-1}\brak{N + P} \text{ where }
			n \in N, p \in P \iff \frac{n}{s} + \frac{p}{s}
			\in S^{-1}N + S^{-1}P
		\]

		\item One way is easy.
		It can be checked that
		\[
			m \in N \cap P \implies \frac{m}{s} \in S^{-1}N \AND
			\frac{m}{s} \in S^{-1}P \implies \frac{m}{s} \in
			S^{-1}N \cap S^{-1}P
		\]
		For the other way,
		\[
			\frac{m}{s} \in S^{-1}N \cap S^{-1}P
			\implies \exists\ n \in N, p \in P \AND s_n, s_p \in S
			\text{ such that } \frac{m}{s} = \frac{n}{s_n} = \frac{p}{s_p}
		\]
		This implies that
		\[
			\exists\ t \in S \text{ such that }
			t\brak{s_p n - s_n p} = 0 \implies
			w \coloneqq t s_p n = t s_n p \in N \cap P
		\]
		We now have
		\[
			\frac{m}{s} = \frac{n}{s_n} =
			\frac{w}{ts_n s_p} \in S^{-1}\brak{N \cap P}
		\]

		\item Consider the exact sequence
		\[
			N \xrightarrow[\quad\quad]{} M \xrightarrow[\quad\quad]{}
			\bigslant{M}{N}
		\]
		Since the operation \(S^{-1}\) is exact, we have
		\[
			S^{-1}N \xrightarrow[\quad\quad]{}
			S^{-1}M \xrightarrow[\quad\quad]{}
			S^{-1}\brak{\bigslant{M}{N}}
		\]
		is exact and hence,
		\[
			S^{-1}\brak{\bigslant{M}{N}} \cong
			\bigslant{\brak{S^{-1}M}}{\brak{S^{-1}N}}
		\]
	\end{enumerate}
\end{proof}

\begin{proposition}{}{}
	Let \(M\) be an \(A\)-module.
	Then, the \(S^{-1}A\) modules are isomorphic
	\[
		S^{-1}M \cong S^{-1}A \otimes_A M
	\]
	More precisely, there is a unique isomorphism
	\[
		f \colon S^{-1}A \otimes_A M \to S^{-1}M
	\]
	such that
	\[
		f\brak{\frac{a}{s} \otimes m} = \frac{am}{s}
		\qquad \forall\ a \in A, m \in M, s \in S
	\]
\end{proposition}
\begin{proof}
	The mapping
	\[
		S^{-1}A \times M \to S^{-1}M
	\]
	given by
	\[
		\brak{\frac{a}{s}, m} \mapsto \frac{am}{s}
	\]
	is \(A\)-bilinear and hence, by the universal property of
	tensor products, there is a unique \(A\)-module homomorphism
	\[
		f \colon S^{-1}A \otimes_A M \to S^{-1}M
	\]
	which proves the uniqueness of \(f\).

	Clearly, \(f\) is surjective.
	To prove injectivity, suppose we have
	\[
		\sum_{i} \frac{a_i}{s_i} \otimes m_i \in \Ker f
	\]
	Let us define
	\[
		s \coloneqq \prod_{i} s_i \qandq t_i \coloneqq \prod_{j \neq i} s_j
	\]
	Now, we have
	\[
		\sum_{i} \brak{\frac{a_i}{s_i} \otimes m_i}
		= \sum_{i} \brak{\frac{a_i t_i}{s} \otimes m_i}
		= \sum_{i} \brak{\frac{1}{s} \otimes a_i t_i m_i}
		= \frac{1}{s} \otimes \brak{\sum_{i} a_i t_i m_i}
		= \frac{1}{s} \otimes m
	\]
	where \(m \coloneqq \sum_{i} a_i t_i m_i\) is an element of \(M\).

	We have,
	\[
		f\brak{\frac{1}{s} \otimes m} = 0 \implies \frac{m}{s} = 0
		\implies \exists\ t \in S \text{ such that } tm = 0
	\]
	Hence,
	\[
		\frac{1}{s} \otimes m = \frac{t}{st} \otimes m
		= \frac{1}{st} \otimes tm = \frac{1}{st} \otimes 0 = 0
	\]
	which proves that \(f\) is injective.
	Hence, \(f\) is an isomorphism.
\end{proof}


\begin{corollary}{}{}
	\(S^{-1}A\) is a flat \(A\)-module.
\end{corollary}
\begin{proof}
	TODO % TODO
\end{proof}


\begin{proposition}{}{}
	If \(M, N\) are \(A\)-modules, there is a unique isomorphism
	of \(S^{-1}A\)-modules
	\[
		f \colon S^{-1}M \otimes_{S^{-1}A} S^{-1}N
		\to S^{-1}\brak{M \otimes_A N}
	\]
	such that
	\[
		f\brak{\frac{m}{s} \otimes \frac{n}{t}} =
		\frac{m \otimes n}{st} \qquad \forall\ m \in M, n \in N, s, t \in S
	\]
	In particular, if \(\mfp\) is any prime ideal, then
	\[
		M_{\mfp} \otimes_{A_{\mfp}} N_{\mfp} \cong
		\brak{M \otimes_A N}_{\mfp}
	\]
	as \(A_\mfp\)-modules.
\end{proposition}
\begin{proof}
	TODO Tensors % TODO
\end{proof}


\pagebreak


\section{Local Properties}

\begin{definition}{Local Property}{}
	A property \(P\) of a ring \(A\) (or of an \(A\)-module \(M\))
	is called a \textbf{local property} if
	\[
		A \text{ (or } M \text{)} \text{ has } P \iff
		A_\mathfrak{p} \text{ (or } M_\mathfrak{p} \text{)} \text{ has } P
		\qquad \forall\ \mathfrak{p} \in \Spec A
	\]
	where \(A_\mathfrak{p} \coloneqq S^{-1}A\) and
	\(S \coloneqq A \setminus \mathfrak{p}\) and
	\(\Spec A\) is the set of prime ideals of \(A\).
\end{definition}

\begin{proposition}{}{}
	Let \(M\) be an \(A\)-module.
	Then, the following are equivalent:
	\begin{enumerate}
		\item \(M = 0\)
		\item \(M_\mathfrak{p} = 0\) for all prime ideals
			\(\mathfrak{p}\) of \(A\)
		\item \(M_\mathfrak{m} = 0\) for all maximal ideals
			\(\mathfrak{m}\) of \(A\)
	\end{enumerate}
\end{proposition}

\begin{proof}
	\(\brak{1 \implies 2 \implies 3}\) is clear.
	We prove \(\brak{3 \implies 1}\).

	Suppose \(3\) holds and \(M \neq 0\).
	Then, there exists \(m \in M\) such that \(m \neq 0\).

	Consider \(\mfa = \Ann(m)\).
	Clearly, \(\mfa \neq A\) and hence, \(\mfa \subseteq \mfm\)
	for some maximal ideal \(\mfm\) of \(A\).

	Consider the element \(\frac{m}{1} \in M_\mfm\).
	This element is zero and hence is killed by an element,
	\(a \in A \setminus \mfm\).

	This is impossible since \(a \notin \mfm \implies a \notin \mfa\)
	but annihilates \((x)\).
\end{proof}


\begin{proposition}{}{}
	Let \(\phi \colon M \to N\) be an \(A\)-module homomorphism.
	Then, the following are equivalent:
	\begin{enumerate}
		\item \(\phi\) is injective
		\item \(\phi_\mathfrak{p} \colon M_\mathfrak{p} \to N_\mathfrak{p}\)
			is injective for all prime ideals \(\mathfrak{p}\) of \(A\)
		\item \(\phi_\mathfrak{m} \colon M_\mathfrak{m} \to N_\mathfrak{m}\)
			is injective for all maximal ideals \(\mathfrak{m}\) of \(A\)
	\end{enumerate}
\end{proposition}

\begin{proof}
	\(\brak{1 \implies 2}\)
	\[
		0 \to M \to N \text{ is exact}
		\implies 0 \to M_\mathfrak{p} \to N_\mathfrak{p} \text{ is exact}
	\]
	and hence, \(\phi_\mathfrak{p}\) is injective for all \(\mathfrak{p}\).

	\(\brak{2 \implies 3}\) is clear.

	\(\brak{3 \implies 1}\)
	Notice that the sequence
	\[
		0 \xrightarrow[\quad\quad]{} \ker \phi
		\xrightarrow[\quad\quad]{i} M
		\xrightarrow[\quad\quad]{\phi} N
	\]
	is exact and hence,
	\[
		0 \xrightarrow[\quad\quad]{} \ker \phi_\mathfrak{m}
		\xrightarrow[\quad\quad]{i_{\mfm}} M_\mathfrak{m}
		\xrightarrow[\quad\quad]{\phi_\mathfrak{m}} N_\mathfrak{m}
	\]
	is exact for all maximal ideals \(\mathfrak{m}\) of \(A\).

	Since \(\Img(i_{\mfm}) = \Ker(\phi_{\mfm}) = 0\), and using the
	fact that \(i_{\mfm}\) is an inclusion, we can conclude that
	\(\Ker(\phi_{\mfm}) = 0\) for all maximal ideals \(\mfm\) of \(A\).

	Hence, we have \(\brak{\ker \phi}_\mathfrak{m} = 0\) for
	all maximal ideals \(\mathfrak{m}\) of \(A\).

	Using the previous proposition, we can conclude that
	\(\ker \phi = 0\) and hence, \(\phi\) is injective.
\end{proof}


\subsection{Flatness as a Local Property}

\begin{proposition}{}{}
	Let \(M\) be an \(A\)-module.
	Then, the following are equivalent:
	\begin{enumerate}
		\item \(M\) is a flat \(A\)-module
		\item \(M_\mathfrak{p}\) is a flat \(A_\mathfrak{p}\)-module
			for all prime ideals \(\mathfrak{p}\) of \(A\)
		\item \(M_\mathfrak{m}\) is a flat \(A_\mathfrak{m}\)-module
			for all maximal ideals \(\mathfrak{m}\) of \(A\)
	\end{enumerate}
\end{proposition}

\begin{proof}
	TODO  % TODO Flatness as a Local Property proof
\end{proof}


\pagebreak


\section{Extended and Contracted Ideals in Rings of Fractions}

Let \(A\) be a ring and \(S \subseteq A\) be a multiplicative
subset of \(A\) and let
\[
	\boxed{f \colon A \to S^{-1}A \qquad a \mapsto \frac{a}{1}}
\]
be the canonical map.

Let
\begin{align*}
	\sC &\coloneqq \fbrak{\mfa \normsg A \mid \mfa \text{ is a contraction
		of an ideal of } S^{-1}A} \\
	\sE &\coloneqq \fbrak{\mfb \normsg S^{-1}A \mid \mfb \text{ is an
		extension of an ideal of } A}
\end{align*}

If \(\mfa\) is an ideal in \(A\), then the extension of \(\mfa\) is
\(S^{-1} \mfa\) where the elements of \(S^{-1} \mfa\) are of the form
\(\frac{a}{s}\) where \(a \in \mfa\) and \(s \in S\).

\begin{proposition}{}{}
\label{prop:3_11}
	\begin{enumerate}
		\item Every ideal in \(S^{-1}A\) is an extended ideal.
		\item If \(\mfa\) is an ideal in \(A\), then
			\(\mfa^{ec} = \cup_{s \in S} \brak{\mfa : s}\).
			Hence, \(\mfa^e = (1) \iff \mfa\) meets \(S\).
		\item \(\mfa \in \sC \iff \) No element of \(S\) is a zero divisor
			on \(\bigslant{A}{\mfa}\).
		\item The prime ideals of \(S^{-1}A\) are in one-to-one
			correspondence with the prime ideals of \(A\) that do not
			meet S\@.
		\item The operation \(S^{-1}\) commutes with formation of
			finite sums, products, intersections and radicals.
	\end{enumerate}
\end{proposition}

\begin{proof} \
	\begin{enumerate}
		\item
		Let \(\mfb\) be an ideal in \(S^{-1}A\).
		Consider some element in \(\mfb\), say \(\frac{x}{s}\).
		Then, \(\frac{x}{s} \cdot \frac{s}{1} = \frac{x}{1} \in \mfb\).

		Hence, \(x \in \mfb^c\) and \(\frac{x}{s} \in \mfb^{ce}\).
		For this ideal, we have \(\mfb \subseteq \mfb^{ce}\).

		Since for any general ideal, we have
		\(\mfb \supseteq \mfb^{ce}\), we can conclude that
		\(\mfb = \mfb^{ce}\) and hence, \(\mfb \in \sE\).

		\item
		Notice that any element of \(\mfa^{e}\) is of the form
		\[
			\sum_{i=1}^{n} \frac{a_i}{s_i}
			= \frac{a}{s} \quad \text{for some} \quad a \in \mfa
			\qandq s \in S
		\]
		after taking \(s = s_1 \cdots s_n\).

		Also, notice that
		\[
			xs \in \mfa \iff x \brak{s} \subseteq \mfa
		\]
		where \(\brak{s}\) is the ideal generated by \(s\).

		Now,
		\begin{align*}
			\mfa^{ec} &= \fbrak{x \in A \mid \frac{x}{1} = \frac{a}{s}
				\text{ for some } a \in \mfa, s \in S} \\
			&= \fbrak{x \in A \mid xst \in \mfa \text{ for some } a \in \mfa,
				s, t \in S} \\
			&= \fbrak{x \in A \mid xs \in \mfa \text{ for some } s \in S} \\
			&= \fbrak{x \in A \mid x \brak{s} \subseteq \mfa
				\text{ for some } s \in S} \\
			&= \cup_{s \in S} \fbrak{x \in A \mid x \brak{s} \subseteq \mfa} \\
			\mfa^{ec} &= \cup_{s \in S} \brak{\mfa : s}
		\end{align*}
		For the next part,
		\[
			\mfa^{e} = \fbrak{\frac{a}{s} \mid a \in \mfa, s \in S}
		\]
		The element \(\frac{1}{1} \in \mfa^{e} \iff\)
		there exists some \(a \in \mfa\) and \(s \in S\) such that
		\(\frac{a}{s} = \frac{1}{1}\).

		This corresponds to \(at = st\) for some \(t \in S\)
		which is equivalent to \(st \in \mfa\).

		Thus,
		\[
			\frac{1}{1} \in \mfa^{e} \iff \exists\ t \in S \st t \in \mfa
		\]

		\item Notice that
		\begin{align*}
			& \mfa \in \sC \\
			\iff & \mfa^{ec} \subseteq \mfa \quad\quad (\text{since }
				\mfa^{ec} = \mfa \AND \mfa^{ec} \supseteq \mfa
				\text{ holds for any ideal}) \\
			\iff & \brak{xs \in \mfa \text{ for some } s \in S
				\implies x \in \mfa} \\
			\iff & \brak{\bar{x} \in \bigslant{A}{\mfa}, \quad
				\bar{x}s = 0 \text{ for some } s \in S
				\implies \bar{x} = 0} \\
			\iff & \text{No element of } S \text{ is a zero divisor on }
				\bigslant{A}{\mfa}
		\end{align*}

		\item
		Consider some prime ideal \(\mfq\) in \(S^{-1}A\).
		Then, \(\mfq = \fbrak{\frac{a}{s} \mid a \in \mfp, s \in S}\)
		for some prime ideal \(\mfp\) of \(A\).

		If there is an element \(\frac{a}{s} \in \mfq\) such that
		\(a \in S\), this implies that \(\frac{1}{s} \in \mfq\) and hence,
		\(\mfq = S^{-1}A\).

		Thus, we can assume that \(a \notin S\) for any \(a \in \mfp\)
		and hence, \(\mfp \cap S = \emptyset\). \\

		Conversely, let \(\mfp\) be a prime ideal in \(A\) such that
		\(\mfp \cap S = \emptyset\).

		We want to show that \(\mfq = \fbrak{\frac{a}{s} \mid a \in \mfp,
		s \in S}\) is a prime ideal in \(S^{-1}A\).

		Consider 2 elements \(\frac{a}{s}, \frac{b}{t} \notin \mfq\).
		We need to show that \(\frac{a}{s} \cdot \frac{b}{t} \notin \mfq\).

		Suppose \(\frac{a}{s} \cdot \frac{b}{t} \in \mfq\).
		Then, \(\frac{ab}{st} \in \mfq \implies abu \in \mfp\)
		for some \(u \in S\).

		But this implies that \(a \in \mfp\) or \(b \in \mfp\) or
		\(u \in \mfp\) neither of which is possible.

		\item
		Using the fact that every ideal in \(S^{-1}A\) is an extended ideal,
		we can deduce that \(S^{-1}\) commutes with formation of finite
		sums and products.
		Intersections has been proven earlier.

		For radicals, we already have \(\rad(\mfa)^e \subseteq
		\rad(\mfa^e)\). \\
		Consider some element \(\frac{a}{s} \in \rad(\mfa^e)\).
		Then, \(\brak{\frac{a}{s}}^n \in \mfa^e\) for some \(n \geq 1\).

		That is, \(\frac{a^n}{s^n} \in \mfa^e\) and hence,
		\(a^n t \in \mfa\) for some \(t \in S\).
		If \(t \notin \mfa\), we are done.

		\(t \in \mfa\) TODO  % TODO
	\end{enumerate}
\end{proof}


\begin{corollary}{}{}
	If \(\mfN\) is the nilradical of \(A\), then \(S^{-1}\mfN\) is the
	nilradical of \(S^{-1}A\).
\end{corollary}
\begin{proof}
	S
\end{proof}

\begin{corollary}{}{}
	If \(\mfp\) is a prime ideal of \(A\), the prime ideals of a local ring
	\(A_{\mfp}\) are in one-to-one correspondence with the prime ideals of
	\(A\) that are contained in \(\mfp\).
\end{corollary}
\begin{proof}
	Consider \(S = A \setminus \mfp\) in the 4th part of the proposition.
\end{proof}

\begin{note}
	Thus, the passage from \(A\) to \(A_{\mfp}\) cuts of all
	the prime ideals that are not contained in \(\mfp\).

	In the other direction, the passage from \(A\) to \(A_{\mfp}\) cuts
	of all the prime ideals except those containing \(\mfp\). \\

	Hence, if \(\mfp\) and \(\mfq\) are prime ideals of \(A\) such that
	\(\mfp \supseteq \mfq\), then localizing at \(\mfp\) and then
	taking quotient mod \(\mfq\) (these operations commute), we are
	restricting our attention to those prime ideals that
	lie between \(\mfq\) and \(\mfp\). \\

	In particular, if \(\mfp = \mfq\), we will be left with a field,
	called the residue field of \(A\) at \(\mfp\), which can be obtained
	either as the field of fractions at \(\bigslant{A}{\mfp}\) or as
	the residue field of the local ring \(A_{\mfp}\).
\end{note}


\begin{proposition}{}{}
	Let \(M\) be a finitely generated \(A\)-module, \(S\) a multiplicatively
	closed subset of \(A\).
	Then,
	\[
		S^{-1} \brak{\Ann(M)} = \Ann(S^{-1}M)
	\]
\end{proposition}
\begin{proof}
	If the proposition is true for \(M\) and \(N\), then it is true for
	\(M + N\).
	\begin{align*}
		S^{-1} \brak{\Ann(M + N)} &= S^{-1} \brak{\Ann(M) \cap \Ann(N)} \\
		&= S^{-1} \brak{\Ann(M)} \cap S^{-1} \brak{\Ann(N)} \\
		&= \Ann(S^{-1}M) \cap \Ann(S^{-1}N) \\
		&= \Ann(S^{-1}M + S^{-1}N) \\
		&= \Ann(S^{-1}(M + N))
	\end{align*}
	Thus, it is enough to prove the proposition for \(M\) generated by
	one element.

	We know that \(M \cong \bigslant{A}{\Ann(M)}\) as an \(A\)-module.
	Hence, we have
	\[
		S^{-1}M \cong S^{-1} \brak{\bigslant{A}{\Ann(M)}}
		\cong \bigslant{S^{-1}A}{S^{-1} \brak{\Ann(M)}}
	\]
	Hence, we have \(\Ann(S^{-1}M) = S^{-1} \brak{\Ann(M)}\).
\end{proof}

\begin{corollary}{}{}
	If \(N, P\) are submodules of an \(A\)-module \(M\), and if \(P\)
	is finitely generated, then
	\[
		S^{-1} \brak{N \colon P} = \brak{S^{-1}N \colon S^{-1}P}
	\]
\end{corollary}
\begin{proof}
	We know that
	\[
		\brak{N \colon P} \cong \Ann \brak{\bigslant{\brak{N + P}}{N}}
	\]
	and now, we can apply the previous proposition.
\end{proof}

\begin{proposition}{}{}
	Let \(A \to B\) be a ring homomorphism and let \(\mfp\) be a prime
	ideal in \(A\).
	Then, \(\mfp\) is a contraction of a prime ideal of \(B\) if and
	only if \(\mfp^{ec} = \mfp\).
\end{proposition}
\begin{proof}
	\(\brak{\Longleftarrow}\) is clear.

	\(\brak{\Longrightarrow}\)
	Let \(\mfp^{ec} = \mfp\).
	Let \(S\) be the image of \(A \setminus \mfp\) in \(B\).
	Then, \(S\) is a multiplicatively closed subset of \(B\).

	Clearly, \(\mfp^e\) does not meet \(S\) and using (3.9), we can
	conclude that \(S^{-1}B \neq B\) and is hence, contained in a maximal
	ideal \(\mfm\) of \(B\).

	If \(\mfq\) is the contraction of \(\mfm\) in \(B\), then \(\mfq\)
	is prime, \(\mfq \cap S = \emptyset\) and \(\mfp \subseteq \mfq\).

	Hence, \(\mfq^c = \mfp\).
\end{proof}
