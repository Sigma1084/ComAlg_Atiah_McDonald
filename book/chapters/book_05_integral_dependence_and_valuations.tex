\chapter{Integral Dependence and Valuations}
\label{ch:book_05_integral_dependence_and_valuations}

\section{Integral Dependence}

\begin{definition}{Integral Element}{}
	Let \(B\) be a ring and \(A \subseteq B\) be a subring \(\brak{1 \in A}\).
	An element \(x \in B\) is said to be \textbf{integral} over \(A\) if \(x\)
	is a root of a \textbf{monic} polynomial with coefficients in \(A\).
	\[
		x^n + a_1 x^{n-1} + \cdots + a_n = 0
	\]
\end{definition}

Consider the case when \(A = \ZZ\) and \(B = \QQ\).
Suppose a rational number \(\frac{r}{s}\) is integral over \(\ZZ\)
where \(\gcd(r, s) = 1\).

Then, \(\frac{r}{s}\) is a root of \(x^m + a_1 x^{m-1} + \cdots + a_m = 0\)
where \(a_i \in \ZZ\).

Multiplying by \(s^m\), we get
\[
	r^m + a_1 r^{m-1} s + \cdots + a_m s^m = 0
\]
We get that \(s \mid r^m\) and hence, \(s = \pm 1\).
Thus, \(\frac{r}{s} \in \ZZ\).

\begin{proposition}{}{}
\label{prop:5.1}
	The following are equivalent
	\begin{enumerate}
		\item \(x \in B\) is integral over \(A\).
		\item \(A[x]\) is a finitely generated \(A\)-module.
		\item \(A[x]\) is contained in a subring \(C\) of \(B\) such that
			\(C\) is a f.g. \(A\)-module.
		\item There exists a faithful \(A[x]\)-module \(M\) which is a
			f.g. \(A\)-module.
	\end{enumerate}
\end{proposition}
\begin{proof}
	\(\brak{1 \Longrightarrow 2}\) From the definition, we have
	\[
		x^{n + r} = - \brak{a_1 x^{n-1} + \cdots + a_n} x^r
		\quad \forall\ r \geq 0
	\]
	Upon induction, we get that
	\(A[x]\) is generated by \(1, x, \ldots, x^{n-1}\).

	\(\brak{2 \Longrightarrow 3}\) Take \(C = A[x]\).

	\(\brak{3 \Longrightarrow 4}\) Take \(M = C\), which is a faithful
	\(A[x]\)-module since \(yC = 0 \implies y \cdot 1 = 0\).

	\(\brak{4 \Longrightarrow 1}\) This follows from proposition 2.4 by
	taking \(\phi\) to be multiplication by \(x\) and \(\mfa\) to be \(A\).

	We have that \(xM \subseteq M = AM\) and hence, x satisfies a monic
	polynomial with coefficients in \(A\) since \(M\) is faithful.
\end{proof}


\begin{corollary}{}{}
\label{cor:5.2}
	Let \(x_i (1 \leq i \leq n)\) be elements of \(B\), integral over \(A\).
	Then, \(A[x_1, \ldots, x_n]\) is a finitely generated \(A\)-module.
\end{corollary}
\begin{proof}
	By induction on \(n\).
	The case \(n = 1\) is covered in proposition~\ref{prop:5.1}.

	Assume \(n > 1\) and let \(A[x_1, x_2, \cdots, x_{n-1}]\) be f.g.\
	as an \(A\) module.

	Now, \(A[x_1, \ldots, x_n-1][x_n]\) is f.g.\ as an
	\(A[x_1, \ldots, x_{n-1}]\) module since \(x_n\) is integral over
	\(A\) and hence \(A[x_1, \ldots, x_{n-1}]\).

	Therefore, \(A[x_1, \ldots, x_n]\) is f.g.\ as an \(A\)-module.
\end{proof}


\begin{corollary}{}{}
\label{cor:5.3}
	The set \(C\) of elements of \(B\) which are integral over \(A\) is
	a subring of \(B\) containing \(A\).
\end{corollary}
\begin{proof}
	Consider some \(x, y \in C\).
	Then, using \(A[x, y]\) is f.g.\ as an \(A\)-module using
	corollary~\ref{cor:5.2}.

	Hence, \(x + y\) and \(xy\) are integral over \(A\) using
	proposition~\ref{prop:5.1}(3).
\end{proof}


\begin{defn}{Integral Closure}{}
	Let \(A \subseteq B\) be rings.
	Then,
	\[
		\bar{A} = C \coloneqq
		\fbrak{x \in B \mid x \text{ is integral over } A}
	\]
	is called the \textbf{integral closure} of \(A\) in \(B\).
\end{defn}

\begin{defn}{Integrally Closed}{}
	Let \(A \subseteq B\) be rings and \(\bar{A} = C\) be the
	integral closure of \(A\) in \(B\).

	Then, \(A\) is said to be \textbf{integrally closed} in \(B\)
	if \(C = A\) or \(\bar{A} = A\).
\end{defn}

\begin{defn}{Integral ring extension}{}
	Let \(A \subseteq B\) be rings and \(\bar{A} = C\) be the
	integral closure of \(A\) in \(B\).

	Then, \(B\) is said to be \textbf{integral over} \(A\) if \(C = B\)
	or \(\bar{A} = B\).
\end{defn}

\begin{note}{}{}
	Let \(f \colon A \to B\) be a ring homomorphism
	so that \(B\) is an \(A\)-algebra.

	Then, \(f\) is said to be \textbf{integral} or \(B\) os said to be
	an \textbf{integral} \(A\)-algebra if \(B\) is integral over \(f(A)\).

	We showed that
	\[
		\text{finite type} + \text{integral} = \text{finite}
	\]
\end{note}


\begin{corollary}{Transitivity of inegral dependence}{}
\label{cor:5.4}
	If \(A \subseteq B \subseteq C\) are rings such that \(B\) is integral
	over \(A\) and \(C\) is integral over \(B\).

	Then, \(C\) is integral over \(A\).
\end{corollary}
\begin{proof}
	Consider some \(x \in C\).
	Then, \(x\) is integral over \(B\) and hence, satisfies a monic
	polynomial with coefficients in \(B\).
	\[
		x^n + b_1 x^{n-1} + \cdots + b_n = 0  \quad b_i \in B
	\]
	The ring \(B' = A[b_1, \ldots, b_n]\) is finitely generated as an
	\(A\)-module using corollary~\ref{cor:5.2}
	(Since \(x\) is integral over \(B'\)).

	Hence, \(B'[x]\) is finitely generated as an \(A\)-module using
	proposition 2.6 and hence \(x\) is integral over \(A\) using
	proposition~\ref{prop:5.1}(3) since \(A[x] \subseteq B'[x]\).
\end{proof}


\begin{corollary}{}{}
\label{cor:5.5}
	Let \(A \subseteq B\) be rings and let \(C = \bar{A}\) be the
	integral closure of \(A\) in \(B\).

	Then, \(C = \bar{A}\) is integrally closed in \(B\).
\end{corollary}
\begin{proof}
	We need to show that \(\bar{\bar{A}} = \bar{A}\) or \(\bar{C} = C\).

	Consider some \(x \in B\) such that \(x\) is integral over \(C\).
	Then, \(x\) is integral over \(A\) using corollary~\ref{cor:5.4}.

	Hence, \(x \in C\) and hence, \(\bar{C} = C\).
\end{proof}


\begin{proposition}{Integral dependence preserved over quotients}{}
\label{prop:5.6.1}
	Let \(A \subseteq B\) be rings and let \(B\) be integral over \(A\).

	Then, if \(\mfb\) is an ideal such that
	\[
		\mfa = \mfb^c = A \cap \mfb \implies \bigslant{B}{\mfb}
		\text{ is integral over } \bigslant{A}{\mfa}
	\]
\end{proposition}
\begin{proof}
	Consider some \(x \in B\).
	Then, \(x\) is integral over \(A\) and hence, satisfies a monic
	polynomial with coefficients in \(A\).
	\[
		x^n + a_1 x^{n-1} + \cdots + a_n = 0 \quad a_i \in A
	\]
	Taking the equation modulo \(\mfb\), we get
	\[
		\brak{x + \mfb}^n + \brak{a_1 + \mfa} \brak{x + \mfb}^{n-1}
		+ \cdots + \brak{a_n + \mfa} = 0
	\]
	since \(\mfa = \mfb^c\) .
\end{proof}


\begin{proposition}{Integral dependence preserved over localizations}{}
\label{prop:5.6.2}
	Let \(A \subseteq B\) be rings and let \(B\) be integral over \(A\).

	If \(S\) is a multiplicative subset of \(A\), then
	\(S^{-1} B\) is integral over \(S^{-1} A\).
\end{proposition}
\begin{proof}
	Consider some \(\frac{x}{s} \in S^{-1} B\).

	Then, \(\frac{x}{s}\) satisfies
	\[
		\brak{\frac{x}{s}}^n + \frac{a_1}{s} \brak{\frac{x}{s}}^{n-1}
		+ \cdots + \frac{a_n}{s^n} = 0 \quad a_i \in A
	\]
	which shows that \(\frac{x}{s}\) is integral over \(S^{-1} A\).
\end{proof}


\pagebreak


\section{The going up theorem}
\begin{proposition}{}{}
\label{prop:5.7}
	Let \(A \subseteq B\) be integral domains and let \(B\) be integral
	over \(A\).
	Then,
	\[
		A \text{ is a field} \iff B \text{ is a field}
	\]
\end{proposition}
\begin{proof}
	\(\brak{\Longrightarrow}\) Consider some \(x \in B \setminus \fbrak{0}\).
	Then, \(x\) is integral over \(A\) and hence, satisfies a monic
	polynomial with coefficients in \(A\).
	Consider the monic polynomial of least degree satisfied by \(x\).
	\[
		x^n + a_1 x^{n-1} + \cdots + a_n = 0 \quad a_i \in A
	\]
	Now, we have \(a_n \neq 0\).
	Otherwise, \(x\) would satisfy a monic polynomial of degree \(< n\).

	Therefore, we have
	\begin{align*}
		-a_n &= x \brak{x^{n-1} + a_1 x^{n-2} + \cdots + a_{n-1}} \\
		\implies x^{-1} &= - \frac{1}{a_n} \brak{x^{n-1} + a_1 x^{n-2}
			+ \cdots + a_{n-1}}
	\end{align*}
	which shows that \(x^{-1} \in B\) and hence, \(B\) is a field.
	
	\vspace{1em}

	\(\brak{\Longleftarrow}\) Conversely, consider some
	\(x \in A \setminus \fbrak{0}\).

	Since \(x \in B\) and \(B\) is a field, we have \(x^{-1} \in B\).

	Therefore, \(x^{-1}\) is integral over \(A\) and hence, satisfies
	a monic polynomial with coefficients in \(A\).
	\[
		x^{-m} + a_1 x^{-m+1} + \cdots + a_n = 0 \quad a_i \in A
	\]
	Multiplying by \(x^{m-1}\), we get
	\[
		x^{-1} = - \brak{a_1 + \cdots + a_n x^{m-1}}
	\]
	which shows that \(x^{-1} \in A\) since \(x \in A\) and
	\(a_i \in A \ \forall\ i\).

	Hence, \(A\) is a field.
\end{proof}


\begin{corollary}{}{}
\label{cor:5.8}
	Let \(A \subseteq B\) be rings and let \(B\) be integral over \(A\).

	Let \(\mfq\) be a prime ideal in \(B\) and let
	\(\mfp = \mfq^c = A \cap \mfq\).

	Then, \(\mfq\) is maximal if and only if \(\mfp\) is maximal.
\end{corollary}
\begin{proof}
	Since we know that a contraction of a prime ideal is prime,
	\(\mfp\) is prime.

	Now, since \(\bigslant{B}{\mfq}\) is integral over \(\bigslant{A}{\mfp}\)
	which are both integral domains, using proposition~\ref{prop:5.7},
	we get that \(\bigslant{A}{\mfp}\) is a field if and only if
	\(\bigslant{B}{\mfq}\) is a field.

	And hence, \(\mfp\) is maximal if and only if \(\mfq\) is maximal.
\end{proof}

\begin{corollary}{}{}
	Let \(A \subseteq B\) be rings and let \(B\) is integral over \(A\).

	Let \(\mfq\) and \(\mfq'\) be prime ideals in \(B\) such that
	\(\mfq \subseteq \mfq'\) and \(\mfq^c = \mfq'^c = \mfp\) say.

	Then, \(\mfq = \mfq'\).
\end{corollary}
\begin{proof}
	Using proposition~\ref{prop:5.6.2}, we know that \(B_{\mfp}\) is
	integral over \(A_{\mfp}\).

	Let \(\mfm\) be the extension of \(\mfp\) in \(B_{\mfp}\).
	and let \(\mfn, \mfn'\) be extensions of \(\mfq, \mfq'\) in
	\(B_{\mfp}\).

	Then, \(\mfm\) is a maximal ideal of \(A_{\mfp}\) and
	\(\mfn^c = \mfn'^c = \mfm\).

	By corollary~\ref{cor:5.8}, it follows that \(\mfn\) and \(\mfn'\)
	are maximal such that \(\mfn \subseteq \mfn'\) which implies
	\(\mfn = \mfn'\) and hence using 3.11 (4), we have
	\(\mfq = \mfq'\).
\end{proof}


\begin{theorem}{}{}
\label{thm:5.10}
	Let \(A \subseteq B\) be rings and let \(B\) be integral over \(A\).

	Let \(\mfp\) be a prime ideal in \(A\).
	Then, there exists a prime ideal \(\mfq\) in \(B\) such that
	\(\mfq \cap A = \mfp\).
\end{theorem}
\begin{proof}
	By proposition~\ref{prop:5.6.2}, we know that \(B_{\mfp}\) is
	integral over \(A_{\mfp}\) and hence the diagram
	\begin{align*}
		\xymatrix{
			A \ar[d]_-{\alpha} \ar[r]^-{} & B \ar[d]_-{\beta} \\
			A_{\mfp} \ar[r]_-{} & B_{\mfp}
		}
	\end{align*}
	is commutative where the horizontal arrows are inclusions and
	the vertical arrows are localizations.

	Let \(\mfn\) be a maximal ideal in \(B_{\mfp}\).
	Then, \(\mfn \cap A_{\mfp}\) is the unique maximal ideal
	of the local ring \(A_{\mfp}\) using corollary~\ref{cor:5.8}.

	If \(\mfq = \beta^{-1}(\mfn)\), then \(\mfq\) is a prime ideal
	since it is a contraction of a prime ideal.

	Hence, we have a prime ideal \(\mfq\) in \(B\) such that
	\(\mfq \cap A = \mfp\).
\end{proof}


\begin{theorem}{Going up theorem}{}
\label{thm:5.11}
	Let \(A \subseteq B\) be rings and let \(B\) be integral over \(A\).
	Let
	\[
		\mfp_1 \subseteq \mfp_2 \subseteq \cdots \subseteq \mfp_n
	\]
	be a chain of prime ideals in \(A\) and
	\[
		\mfq_1 \subseteq \mfq_2 \subseteq \cdots \subseteq \mfq_m
	\]
	be a chain of prime ideals \(\brak{m < n}\) in \(B\) such that
	\[
		\mfq_i \cap A = \mfp_i \quad \forall\ i \in \fbrak{1, \ldots, m}
	\]
	Then, the chain of prime ideals in \(B\) can be extended to a chain
	\[
		\mfq_1 \subseteq \mfq_2 \subseteq \cdots \subseteq \mfq_n
	\]
	such that \(\mfq_i \cap A = \mfp_i\)
	for all \(i \in \fbrak{1, \ldots, n}\).
\end{theorem}
\begin{proof}
	By induction, we can reduce the problem to the case
	when \(n = 2\) and \(m = 1\).

	Consider the diagram
	\begin{align*}
		\xymatrix{
			A \ar[d]_-{\alpha} \ar[r]^-{} & B \ar[d]_-{\beta} \\
			\bigslant{A}{\mfp_1} \ar[r]_-{} & \bigslant{B}{\mfq_1}
		}
	\end{align*}
	where the horizontal arrows are inclusions and the vertical arrows
	are quotients.

	Then, \(\beta\) is integral using proposition~\ref{prop:5.6.1}.

	By theorem~\ref{thm:5.10}, there exists a prime ideal \(\bar{\mfq_2}\),
	say, in \(\bigslant{B}{\mfq_1}\) such that \(\bar{\mfq_2} \cap A
	= \bar{\mfp_2}\), the image of \(\mfp_2\) in \(\bar{A}\).

	Lifting back \(\bar{\mfp_2}\) to \(B\) and we have a prime ideal
	\(\mfq_2\) in \(B\) such that \(\mfq_2 \cap A = \mfp_2\).
\end{proof}