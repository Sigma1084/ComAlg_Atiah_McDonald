\chapter{Integral Dependence and Valuations}
\label{ch:book_05_integral_dependence_and_valuations}

\section{Integral Dependence}

\begin{definition}{Integral Element}{}
	Let \(B\) be a ring and \(A \subseteq B\) be a subring \(\brak{1 \in A}\).
	An element \(x \in B\) is said to be \textbf{integral} over \(A\) if \(x\)
	is a root of a \textbf{monic} polynomial with coefficients in \(A\).
	\[
		x^n + a_1 x^{n-1} + \cdots + a_n = 0
	\]
\end{definition}

Consider the case when \(A = \ZZ\) and \(B = \QQ\).
Suppose a rational number \(\frac{r}{s}\) is integral over \(\ZZ\)
where \(\gcd(r, s) = 1\).

Then, \(\frac{r}{s}\) is a root of \(x^m + a_1 x^{m-1} + \cdots + a_m = 0\)
where \(a_i \in \ZZ\).

Multiplying by \(s^m\), we get
\[
	r^m + a_1 r^{m-1} s + \cdots + a_m s^m = 0
\]
We get that \(s \mid r^m\) and hence, \(s = \pm 1\)
Thus, \(\frac{r}{s} \in \ZZ\).

\begin{proposition}{}{}
\label{prop:5.1}
	The following are equivalent
	\begin{enumerate}
		\item \(x \in B\) is integral over \(A\).
		\item \(A[x]\) is a finitely generated \(A\)-module.
		\item \(A[x]\) is contained in a subring \(C\) of \(B\) such that
			\(C\) is a f.g. \(A\)-module.
		\item There exists a faithful \(A[x]\)-module \(M\) which is a
			f.g. \(A\)-module.
	\end{enumerate}
\end{proposition}
\begin{proof}
	\(\brak{1 \Longrightarrow 2}\) From the definition, we have
	\[
		x^{n + r} = - \brak{a_1 x^{n-1} + \cdots + a_n} x^r
		\quad \forall\ r \geq 0
	\]
	Upon induction, we get that
	\(A[x]\) is generated by \(1, x, \ldots, x^{n-1}\).

	\(\brak{2 \Longrightarrow 3}\) Take \(C = A[x]\).

	\(\brak{3 \Longrightarrow 4}\) Take \(M = C\), which is a faithful
	\(A[x]\)-module since \(yC = 0 \implies y \cdot 1 = 0\).

	\(\brak{4 \Longrightarrow 1}\) This follows from proposition 2.4 by
	taking \(\phi\) to be multiplication by \(x\) and \(\mfa\) to be \(A\).

	We have that \(xM \subseteq M = AM\) and hence, x satisfies a monic
	polynomial with coefficients in \(A\) since \(M\) is faithful.
\end{proof}


\begin{corollary}{}{}
\label{cor:5.2}
	Let \(x_i (1 \leq i \leq n)\) be elements of \(B\), integral over \(A\).
	Then, \(A[x_1, \ldots, x_n]\) is a finitely generated \(A\)-module.
\end{corollary}
\begin{proof}
	By induction on \(n\).
	The case \(n = 1\) is covered in proposition~\ref{prop:5.1}.

	Assume \(n > 1\) and let \(A[x_1, x_2, \cdots, x_{n-1}]\) be f.g.\
	as an \(A\) module.

	Now, \(A[x_1, \ldots, x_n-1][x_n]\) is f.g.\ as an
	\(A[x_1, \ldots, x_{n-1}]\) module since \(x_n\) is integral over
	\(A\) and hence \(A[x_1, \ldots, x_{n-1}]\).

	Therefore, \(A[x_1, \ldots, x_n]\) is f.g.\ as an \(A\)-module.
\end{proof}


\begin{corollary}{}{}
\label{cor:5.3}
	The set \(C\) of elements of \(B\) which are integral over \(A\) is
	a subring of \(B\) containing \(A\).
\end{corollary}
\begin{proof}
	Consider some \(x, y \in C\).
	Then, using \(A[x, y]\) is f.g.\ as an \(A\)-module using
	corollary~\ref{cor:5.2}.

	Hence, \(x + y\) and \(xy\) are integral over \(A\) using
	proposition~\ref{prop:5.1}(3).
\end{proof}


\begin{defn}{Integral Closure}{}
	Let \(A \subseteq B\) be rings.
	Then,
	\[
		\bar{A} = C \coloneqq
		\fbrak{x \in B \mid x \text{ is integral over } A}
	\]
	is called the \textbf{integral closure} of \(A\) in \(B\).
\end{defn}

\begin{defn}{Integrally Closed}{}
	Let \(A \subseteq B\) be rings and \(\bar{A} = C\) be the
	integral closure of \(A\) in \(B\).

	Then, \(A\) is said to be \textbf{integrally closed} in \(B\)
	if \(C = A\) or \(\bar{A} = A\).
\end{defn}

\begin{defn}{Integral ring extension}{}
	Let \(A \subseteq B\) be rings and \(\bar{A} = C\) be the
	integral closure of \(A\) in \(B\).

	Then, \(B\) is said to be \textbf{integral over} \(A\) if \(C = B\)
	or \(\bar{A} = B\).
\end{defn}

\begin{note}{}{}
	Let \(f \colon A \to B\) be a ring homomorphism
	so that \(B\) is an \(A\)-algebra.

	Then, \(f\) is said to be \textbf{integral} or \(B\) os said to be
	an \textbf{integral} \(A\)-algebra if \(B\) is integral over \(f(A)\).

	We showed that
	\[
		\text{finite type} + \text{integral} = \text{finite}
	\]
\end{note}


\begin{corollary}{Transitivity of inegral dependence}{}
\label{cor:5.4}
	If \(A \subseteq B \subseteq C\) are rings such that \(B\) is integral
	over \(A\) and \(C\) is integral over \(B\).

	Then, \(C\) is integral over \(A\).
\end{corollary}
\begin{proof}
	Consider some \(x \in C\).
	Then, \(x\) is integral over \(B\) and hence, satisfies a monic
	polynomial with coefficients in \(B\).
	\[
		x^n + b_1 x^{n-1} + \cdots + b_n = 0  \quad b_i \in B
	\]
	The ring \(B' = A[b_1, \ldots, b_n]\) is finitely generated as an
	\(A\)-module using corollary~\ref{cor:5.2}
	(Since \(x\) is integral over \(B'\)).

	Hence, \(B'[x]\) is finitely generated as an \(A\)-module using
	proposition 2.6 and hence \(x\) is integral over \(A\) using
	proposition~\ref{prop:5.1}(3) since \(A[x] \subseteq B'[x]\).
\end{proof}


\begin{corollary}{}{}
\label{cor:5.5}
	Let \(A \subseteq B\) be rings and let \(C = \bar{A}\) be the
	integral closure of \(A\) in \(B\).

	Then, \(C = \bar{A}\) is integrally closed in \(B\).
\end{corollary}
\begin{proof}
	We need to show that \(\bar{\bar{A}} = \bar{A}\) or \(\bar{C} = C\).

	Consider some \(x \in B\) such that \(x\) is integral over \(C\).
	Then, \(x\) is integral over \(A\) using corollary~\ref{cor:5.4}.

	Hence, \(x \in C\) and hence, \(\bar{C} = C\).
\end{proof}


\begin{proposition}{Integral dependence preserved over quotients}{}
\label{prop:5.6.1}
	Let \(A \subseteq B\) be rings and let \(B\) be integral over \(A\).

	Then, if \(\mfb\) is an ideal such that
	\[
		\mfa = \mfb^c = A \cap \mfb \implies \bigslant{B}{\mfb}
		\text{ is integral over } \bigslant{A}{\mfa}
	\]
\end{proposition}
\begin{proof}
	Consider some \(x \in B\).
	Then, \(x\) is integral over \(A\) and hence, satisfies a monic
	polynomial with coefficients in \(A\).
	\[
		x^n + a_1 x^{n-1} + \cdots + a_n = 0 \quad a_i \in A
	\]
	Taking the equation modulo \(\mfb\), we get
	\[
		\brak{x + \mfb}^n + \brak{a_1 + \mfa} \brak{x + \mfb}^{n-1}
		+ \cdots + \brak{a_n + \mfa} = 0
	\]
	since \(\mfa = \mfb^c\) .
\end{proof}


\begin{proposition}{Integral dependence preserved over localizations}{}
\label{prop:5.6.2}
	Let \(A \subseteq B\) be rings and let \(B\) be integral over \(A\).

	If \(S\) is a multiplicative subset of \(A\), then
	\(S^{-1} B\) is integral over \(S^{-1} A\).
\end{proposition}
\begin{proof}
	Consider some \(\frac{x}{s} \in S^{-1} B\).

	Then, \(\frac{x}{s}\) satisfies
	\[
		\brak{\frac{x}{s}}^n + \frac{a_1}{s} \brak{\frac{x}{s}}^{n-1}
		+ \cdots + \frac{a_n}{s^n} = 0 \quad a_i \in A
	\]
	which shows that \(\frac{x}{s}\) is integral over \(S^{-1} A\).
\end{proof}


\pagebreak


\section{The going up theorem}
\begin{proposition}{}{}
\label{prop:5.7}
	Let \(A \subseteq B\) be integral domains and let \(B\) be integral
	over \(A\).
	Then,
	\[
		A \text{ is a field} \iff B \text{ is a field}
	\]
\end{proposition}
\begin{proof}
	\(\brak{\Longrightarrow}\) Consider some \(x \in B \setminus \fbrak{0}\).
	Then, \(x\) is integral over \(A\) and hence, satisfies a monic
	polynomial with coefficients in \(A\).
	Consider the monic polynomial of least degree satisfied by \(x\).
	\[
		x^n + a_1 x^{n-1} + \cdots + a_n = 0 \quad a_i \in A
	\]
	Now, we have \(a_n \neq 0\).
	Otherwise, \(x\) would satisfy a monic polynomial of degree \(< n\).

	Therefore, we have
	\begin{align*}
		-a_n &= x \brak{x^{n-1} + a_1 x^{n-2} + \cdots + a_{n-1}} \\
		\implies x^{-1} &= - \frac{1}{a_n} \brak{x^{n-1} + a_1 x^{n-2}
			+ \cdots + a_{n-1}}
	\end{align*}
	which shows that \(x^{-1} \in B\) and hence, \(B\) is a field.
	
	\vspace{1em}

	\(\brak{\Longleftarrow}\) Conversely, consider some
	\(x \in A \setminus \fbrak{0}\).

	Since \(x \in B\) and \(B\) is a field, we have \(x^{-1} \in B\).

	Therefore, \(x^{-1}\) is integral over \(A\) and hence, satisfies
	a monic polynomial with coefficients in \(A\).
	\[
		x^{-m} + a_1 x^{-m+1} + \cdots + a_n = 0 \quad a_i \in A
	\]
	Multiplying by \(x^{m-1}\), we get
	\[
		x^{-1} = - \brak{a_1 + \cdots + a_n x^{m-1}}
	\]
	which shows that \(x^{-1} \in A\) since \(x \in A\) and
	\(a_i \in A \ \forall\ i\).

	Hence, \(A\) is a field.
\end{proof}


\begin{corollary}{}{}
\label{cor:5.8}
	Let \(A \subseteq B\) be rings and let \(B\) be integral over \(A\).

	Let \(\mfq\) be a prime ideal in \(B\) and let
	\(\mfp = \mfq^c = A \cap \mfq\).

	Then, \(\mfq\) is maximal if and only if \(\mfp\) is maximal.
\end{corollary}
\begin{proof}
	Since we know that a contraction of a prime ideal is prime,
	\(\mfp\) is prime.

	Now, since \(\bigslant{B}{\mfq}\) is integral over \(\bigslant{A}{\mfp}\)
	which are both integral domains, using proposition~\ref{prop:5.7},
	we get that \(\bigslant{A}{\mfp}\) is a field if and only if
	\(\bigslant{B}{\mfq}\) is a field.

	And hence, \(\mfp\) is maximal if and only if \(\mfq\) is maximal.
\end{proof}

\begin{corollary}{}{}
	Let \(A \subseteq B\) be rings and let \(B\) is integral over \(A\).

	Let \(\mfq\) and \(\mfq'\) be prime ideals in \(B\) such that
	\(\mfq \subseteq \mfq'\) and \(\mfq^c = \mfq'^c = \mfp\) say.

	Then, \(\mfq = \mfq'\).
\end{corollary}
\begin{proof}
	Using proposition~\ref{prop:5.6.2}, we know that \(B_{\mfp}\) is
	integral over \(A_{\mfp}\).

	Let \(\mfm\) be the extension of \(\mfp\) in \(B_{\mfp}\).
	and let \(\mfn, \mfn'\) be extensions of \(\mfq, \mfq'\) in
	\(B_{\mfp}\).

	Then, \(\mfm\) is a maximal ideal of \(A_{\mfp}\) and
	\(\mfn^c = \mfn'^c = \mfm\).

	By corollary~\ref{cor:5.8}, it follows that \(\mfn\) and \(\mfn'\)
	are maximal such that \(\mfn \subseteq \mfn'\) which implies
	\(\mfn = \mfn'\) and hence using 3.11 (4), we have
	\(\mfq = \mfq'\).
\end{proof}


\begin{theorem}{}{}
\label{thm:5.10}
	Let \(A \subseteq B\) be rings and let \(B\) be integral over \(A\).

	Let \(\mfp\) be a prime ideal in \(A\).
	Then, there exists a prime ideal \(\mfq\) in \(B\) such that
	\(\mfq \cap A = \mfp\).
\end{theorem}
\begin{proof}
	By proposition~\ref{prop:5.6.2}, we know that \(B_{\mfp}\) is
	integral over \(A_{\mfp}\) and hence the diagram
	\begin{align*}
		\xymatrix{
			A \ar[d]_-{\alpha} \ar[r]^-{} & B \ar[d]_-{\beta} \\
			A_{\mfp} \ar[r]_-{} & B_{\mfp}
		}
	\end{align*}
	is commutative where the horizontal arrows are inclusions and
	the vertical arrows are localizations.

	Let \(\mfn\) be a maximal ideal in \(B_{\mfp}\).
	Then, \(\mfn \cap A_{\mfp}\) is the unique maximal ideal
	of the local ring \(A_{\mfp}\) using corollary~\ref{cor:5.8}.

	If \(\mfq = \beta^{-1}(\mfn)\), then \(\mfq\) is a prime ideal
	since it is a contraction of a prime ideal.

	Hence, we have a prime ideal \(\mfq\) in \(B\) such that
	\(\mfq \cap A = \mfp\).
\end{proof}


\begin{theorem}{Going up theorem}{}
\label{thm:5.11}
	Let \(A \subseteq B\) be rings and let \(B\) be integral over \(A\).
	Let
	\[
		\mfp_1 \subseteq \mfp_2 \subseteq \cdots \subseteq \mfp_n
	\]
	be a chain of prime ideals in \(A\) and
	\[
		\mfq_1 \subseteq \mfq_2 \subseteq \cdots \subseteq \mfq_m
	\]
	be a chain of prime ideals \(\brak{m < n}\) in \(B\) such that
	\[
		\mfq_i \cap A = \mfp_i \quad \forall\ i \in \fbrak{1, \ldots, m}
	\]
	Then, the chain of prime ideals in \(B\) can be extended to a chain
	\[
		\mfq_1 \subseteq \mfq_2 \subseteq \cdots \subseteq \mfq_n
	\]
	such that \(\mfq_i \cap A = \mfp_i\)
	for all \(i \in \fbrak{1, \ldots, n}\).
\end{theorem}
\begin{proof}
	By induction, we can reduce the problem to the case
	when \(n = 2\) and \(m = 1\).

	Consider the diagram
	\begin{align*}
		\xymatrix{
			A \ar[d]_-{\alpha} \ar[r]^-{} & B \ar[d]_-{\beta} \\
			\bigslant{A}{\mfp_1} \ar[r]_-{} & \bigslant{B}{\mfq_1}
		}
	\end{align*}
	where the horizontal arrows are inclusions and the vertical arrows
	are quotients.

	Then, \(\beta\) is integral using proposition~\ref{prop:5.6.1}.

	By theorem~\ref{thm:5.10}, there exists a prime ideal \(\bar{\mfq_2}\),
	say, in \(\bigslant{B}{\mfq_1}\) such that \(\bar{\mfq_2} \cap A
	= \bar{\mfp_2}\), the image of \(\mfp_2\) in \(\bar{A}\).

	Lifting back \(\bar{\mfp_2}\) to \(B\) and we have a prime ideal
	\(\mfq_2\) in \(B\) such that \(\mfq_2 \cap A = \mfp_2\).
\end{proof}


\pagebreak


\section{Integrally Closed Integral Domains and The Going Down Theorem}

The proposition~\ref{prop:5.6.2} can be sharpened to the following.

\begin{proposition}{}{}
\label{prop:5.12}
	Let \(A \subseteq B\) be rings and
	\(C = \bar{A}\) be the integral closure of \(A\) in \(B\).

	Let \(S\) be a multiplicatively closed subset of \(A\).

	Then, \(S^{-1} C = S^{-1} \bar{A}\) is the integral closure of
	\(S^{-1} A\) in \(S^{-1} B\).
\end{proposition}
\begin{proof}
	Using proposition~\ref{prop:5.6.2}, we know that \(S^{-1} C\) is
	integral over \(S^{-1} A\).

	Conversely, consider some integral element \(\frac{b}{s} \in S^{-1} B\).
	Then, \(\frac{b}{s}\) satisfies
	\[
		\brak{\frac{b}{s}}^n + \frac{a_1}{s_1} \brak{\frac{b}{s}}^{n-1}
		+ \cdots + \frac{a_n}{s_n} = 0 \quad\quad a_i \in A, s_i \in S
	\]
	Let \(t = s_1 s_2 \cdots s_n\).
	Multiplying the above equation by \((st)^n\), we get
	\[
		\brak{bt}^n + \frac{a_1 st}{s_1} \brak{bt}^{n-1}
		+ \frac{a_2 s^2 t^2}{s_2} \brak{bt}^{n-2}
		+ \cdots + \frac{a_n s^n t^n}{s_n} = 0
	\]
	Notice that after cancelling,
	\[
		\frac{a_i s^i t^i}{s_i} = \frac{a_1 s^i t^{i-1} \cdot
		s_1 \ldots s_{i-1} \cancel{s_i} s_{i+1} \ldots s_n}{\cancel{s_i}}
		= a_i' \in A
	\]
	Therefore, we have
	\[
		\brak{bt}^n + a_1' \brak{bt}^{n-1} + \cdots + a_n' = 0
	\]
	which shows that \(bt\) is integral over \(A\) and hence, \(bt \in C\)
	and hence,
	\[
		\frac{b}{s} = \frac{bt}{st} \in S^{-1} C \qedhere
	\]
\end{proof}

\begin{defn}{Integrally Closed Domain}{}
	An integral domain \(A\) is said to be \textbf{integrally closed} if
	it is integrally closed in its field of fractions.
\end{defn}

For example, the ring of integers \(\ZZ\) is integrally closed since it is
integrally closed in \(\QQ\).

Also, notice that the ring \(\ZZ\sbrak{\sqrt{-5}}\) is not integrally closed
since \(\frac{1 + \sqrt{-5}}{2}\) is integral over \(\ZZ\sbrak{\sqrt{-5}}\).

\(\frac{1 + \sqrt{-5}}{2}\) is a root of \(x^2 - x + 2 = 0\).

The same argument holds for anu UFD.\
In particular, a polynomial ring \(k\sbrak{x_1, \ldots, x_n}\) is integrally
closed where \(k\) is a field.

\begin{proposition}{Integral Closure is a local property}{}
\label{prop:5.13}
	Let \(A\) be an integral domain.
	Then, the following are equivalent.
	\begin{enumerate}
		\item \(A\) is integrally closed.
		\item \(A_{\mfp}\) is integrally closed for every prime ideal
			\(\mfp\) in \(A\).
		\item \(A_{\mfm}\) is integrally closed for every maximal ideal
			\(\mfm\) in \(A\).
	\end{enumerate}
\end{proposition}
\begin{proof}
	Let \(K\) be the field of fractions of \(A\) and \(C\) be the integral
	closure of \(A\) in \(K\).

	Let \(f \colon A \to C\) be the identity mapping of \(A\) into \(C\).

	Clearly,
	\[
		f \text{ is surjective} \iff C = A \iff A \text{ is integrally closed}
	\]

	We now argue
	\[
		f \text{ is surjective} \iff \text{all }f_{\mfp} \text{ is surjective}
		\iff \text{all } f_{\mfm} \text{ is surjective}
	\]
	using proposition 3.9.
\end{proof}

\begin{defn}{Integral over an ideal and Integral Closure}{}
	Let \(A \subseteq B\) be rings and let \(\mfa\) be an ideal in \(A\).

	An element \(x \in B\) is said to be \textbf{integral over} \(\mfa\)
	if it satisfies a monic polynomial with coefficients in \(\mfa\).

	The \textbf{integral closure} of an ideal \(\mfa\) in \(B\) is
	\[
		\bar{\mfa} = \fbrak{x \in B \mid x \text{ is integral over } \mfa}
	\]
\end{defn}


\begin{lemma}{}{}
\label{lemma:5.14}
	Let \(A \subseteq B\) be rings and let \(C\) be the integral closure
	of \(A\) in \(B\).

	Let \(\mfa\) be an ideal in \(A\) and let \(\mfa^e\) be the extension
	of \(\mfa\) in \(C\).

	Then, the integral closure of \(\mfa\) in \(B\) is the radical of
	\(\mfa^e\).
	\[
		\rad(\mfa^e) = \fbrak{x \in B \mid x \text{ is integral over } \mfa}
	\]

	(Hence, integral closure of extensions is closed under addition and
	multiplication.)
\end{lemma}
\begin{proof}
	If \(x \in B\) is integral over \(\mfa\), we have an equation of the form
	\[
		x^n + a_1 x^{n-1} + \cdots + a_n = 0 \quad a_i \in \mfa
	\]
	We have
	\[
		x \in C \AND x^n \in \mfa^e \implies x \in \rad(\mfa^e)
	\]
	Conversely, consider some \(x \in \rad(\mfa^e)\).
	Then,
	\[
		x^n = \sum_{i = 1}^m a_i x_i \quad a_i \in \mfa, x_i \in C
	\]
	Since every \(x_i\) is integral over \(A\), the module
	\(M = A\sbrak{x_1, \cdots, x_n}\) is finitely generated over \(A\)
	and we have
	\[
		x^n M \subseteq \mfa M
	\]
	Using proposition 2.4 with \(\phi\) being multiplication by \(x^n\),
	we get that \(x^n\) is integral over \(A\).

	Therefore, \(x\) is integral over \(A\).
\end{proof}


\begin{proposition}{}{}
\label{prop:5.15}
	Let \(A \subseteq B\) be integral domains such that \(A\) is integrally
	closed.
	Let \(x \in B\) be integral over an ideal \(\mfa \normsg A\).

	Then, \(x\) is algebraic over the field of fractions, \(K\) of \(A\),
	and if its minimal polynomial over \(K\) is
	\[
		t^n + a_1 t^{n-1} + \cdots + a_n = 0 \quad a_i \in K
	\]
	then, \(a_i \in \rad(\mfa)\) for all \(i \in \fbrak{1, \ldots, n}\).
\end{proposition}
\begin{proof}
	Clearly, \(x\) is algebraic over \(K\) since it is integral over \(A\).

	Let \(L\) be an extension field of \(K\) which contains all the
	conjugates \(x_1, \ldots, x_n\) of \(x\).

	Each \(x_i\) satisfies the same equation of integral dependence
	as \(x\) and hence, each \(x_i\) is integral over \(\mfa\).

	The coefficients of the minimal polynomial of \(x\) over \(K\)
	are polynomials in \(x_i\) and hence, using lemma~\ref{lemma:5.14},
	are integral over \(\mfa\).

	Since \(A\) is integrally closed, they must lie in \(\rad(\mfa)\).
\end{proof}


\begin{theorem}{Going down theorem}{}
\label{thm:5.16}
	Let \(A \subseteq B\) be integral domains with \(A\) integrally closed
	and \(B\) integral over \(A\).

	Let
	\[
		\mfp_1 \supseteq \mfp_2 \supseteq \cdots \supseteq \mfp_n
	\]
	be a chain of prime ideals in \(A\) and let
	\[
		\mfq_1 \supseteq \mfq_2 \supseteq \cdots \supseteq \mfq_m
		\quad \brak{m < n}
	\]
	be a chain of prime ideals in \(B\) such that \(\mfq_i \cap A = \mfp_i\)
	for all \(i \in \fbrak{1, \ldots, m}\).

	Then, the chain of prime ideals in \(B\) can be extended to a chain
	\[
		\mfq_1 \supseteq \mfq_2 \supseteq \cdots \supseteq \mfq_n
	\]
	such that \(\mfq_i \cap A = \mfp_i\)
	for all \(i \in \fbrak{1, \ldots, n}\).
\end{theorem}
\begin{proof}
	As in theorem~\ref{thm:5.11}, we reduce immediately to the case
	when \(n = 2\) and \(m = 1\).

	From the note~\ref{note:3_localization_of_prime_ideals}, recall that
	after localization of \(B\) at \(\mfq_1\), only the prime ideals
	contained in \(\mfq_1\) survive.

	Now, the problem is reduced to showing that \(\mfp_2\) is the
	contraction of a prime ideal in \(B_{\mfq_1}\) or equivalently,
	using proposition~\ref{prop:3_16},
	\[
		\mfp_2^{ec} = \mfp_2 \quad \text{where }
		\mfp_2^e = B_{\mfq_1} \mfp_2 \quad \text{and }
		\mfp_2^{ec} = \mfp_2^e \cap A
		\quad\quad
		\iff B_{\mfq_1} \mfp_2 \cap A = \mfp_2
	\]
	We just want to show that \(B_{\mfq_1} \mfp_2 \cap A = \mfp_2\).

	Every \(x \in B_{\mfq_1} \mfp_2\) is of the form \(\frac{y}{s}\)
	where \(y \in B\mfp_2\) and \(s \in B \setminus \mfq_1\).
	\[
		y \in B\mfp_2 \implies y \in \mfp_2^e \implies y \in \rad(\mfp_2^e)
		\implies y \text{ is integral over } \mfp_2^{ec} = \mfp_2
	\]
	The last implication follows from lemma~\ref{lemma:5.14}.

	Using proposition~\ref{prop:5.15}, we get that the minimal equation
	satisfied by \(y\) over \(K\) has coefficients in \(\rad\brak{\mfp_2}\).
	\[
		y^n + u_1 y^{n-1} + \cdots + u_r = 0 \quad u_i \in \rad\brak{\mfp_2}
	\]
	TODO % TODO Finish this proof
\end{proof}

The proof of the next theorem assumes some standard facts from
field theory.
\begin{proposition}{}{}
\label{prop:5.17}
	Let \(A\) be an integrally closed domain, \(K\) its field of fractions
	and \(L\) a finite separable (algebraic) extension of \(K\), \(B\)
	the integral closure of \(A\) in \(L\).

	Then, there exists a basis \(v_1, \cdots, v_n\) of \(L\) over \(K\)
	such that \(B\) is contained in the \(A\)-submodule of \(L\) generated
	by \(v_1, \cdots, v_n\).
\end{proposition}




\pagebreak

\section{Valuation Rings}
\begin{definition}{Valuation Ring}{}
	Let \(B\) be an integral domain, \(K\) its field of fractions.

	\(B\) is a \emph{valuation ring} if,
	\[
		x \in K \setminus \fbrak{0}
		\implies x \in B \OR x^{-1} \in B
	\]
\end{definition}

We see a few properties of valuation rings.

\begin{proposition}{}{}
\label{prop:5.18.1}
	Let \(B\) be a valuation ring.
	Then, \(B\) is a local ring.
\end{proposition}
\begin{proof}
	We construct the maximal ideal of \(B\) and show that it is unique.

	Consider the set of non-units in \(B\).
	\[
		\mfm \coloneqq \fbrak{x \in B \mid x = 0 \OR x\inv \notin B}
	\]
	We show that \(\mfm\) is an ideal in \(B\).

	Consider two non-zero elements \(x, y \in \mfm \setminus \fbrak{0}\).

	Closed under addition.
	Clearly, we have
	\[
		xy^{-1} \in B \qorq yx^{-1} \in B
	\]
	In the first case, we have
	\[
		xy^{-1} \in B \implies
		x + y = y + xy^{-1} y = y \brak{1 + xy^{-1}} \in B \mfm
		\implies x + y \in \mfm
	\]
	We get the same result in the other case as well.

	Absorbing property.
	Consider some \(a \in B\) and \(x \in \mfm\).
	Then, we have \(ax \in \mfm\).
	Otherwise,
	\[
		ax \notin \mfm \implies \brak{ax}^{-1} \in B
		\implies x^{-1} a^{-1} \in B \implies x^{-1} a^{-1} a = x^{-1} \in B
	\]
	which is a contradiction and hence,
	\[
		a \in B, x \in \mfm \implies ax \in \mfm
	\]
	This proves that \(\mfm\) is an ideal in \(B\).

	Using the result 1.6, we can conclude that \(\mfm\) is a unique
	maximal ideal in \(B\) and hence, \(B\) is a local ring.
\end{proof}

\begin{proposition}{}{}
\label{prop:5.18.2}
	Let \(B\) be a valuation ring.
	Then,
	\[
		B \subseteq B' \subseteq K
		\implies B' \text{ is a valuation ring}
	\]
\end{proposition}
\begin{proof}
	Consider some \(x \in K \setminus{\fbrak{0}}\).

	Then,
	\[
		x \notin B'
		\implies x \notin B
		\implies x^{-1} \in B
		\implies x^{-1} \in B'
	\]
	and clearly \(0 \in B'\) and hence,
	\(B'\) is a valuation ring.
\end{proof}

\begin{proposition}{}{}
\label{prop:5.18.3}
	Let \(B\) be a valuation ring.
	Then, \(B\) is integrally closed.
	(in \(K\))
\end{proposition}
\begin{proof}
	Consider some element \(x \in K\) which is integral over \(B\).
	\[
		x^n + b_1 x^{n-1} + \cdots + b_n = 0 \quad b_i \in B
	\]
	If \(x \in B\), then we're done.
	Otherwise, \(x^{-1} \in B\) and hence,
	\[
		x = -b_1 - b_2 x^{-1} - \cdots - b_n x^{-(n-1)} \in B \qedhere
	\]
\end{proof}


\subsection{Construction of a valuation ring from a field}

Let \(K\) be a field and \(\Omega\) be an algebraically closed field.
\[
	\Sigma = \fbrak{\brak{A, f} \colon A \text{ is a subring of } K,
		f \colon A \to \Omega \text{ is a ring homomorphism}
	}
\]
We then define a partial order on \(\Sigma\) as follows.
\[
	\brak{A, f} \leq \brak{A', f'} \iff A \subseteq A', \ f' \mid_A = f
\]
The conditions of Zorn's lemma are satisfied and hence, the set
\(\Sigma\) has a maximal element \(\brak{B, g}\).

We claim
\begin{claim}{}{}
	\(B\) is a valuation ring of \(K\).
\end{claim}

To prove this, we start with the following lemma.
\begin{lemma}{}{}
\label{lemma:5.19}
	\(B, g\) as defined above.
	\(B\) is a local ring and \(\mfm = \Ker (g)\) is its maximal ideal.
\end{lemma}
\begin{proof}
	Since \(g(B)\) is a subring of a field, it is an integral domain.

	Hence,
	\(
		\mfm = \Ker (g)
	\)
	is prime.

	We can now extend \(g\) to a homomorphism
	\[
		\bar{g} \colon K_{\mfm} \to \Omega
	\]
	defined by
	\[
		\bar{g}\brak{\frac{b}{s}} = \frac{g(b)}{g(s)}
		\quad \forall\ b \in B, \ s \in B \setminus \mfm
		\quad \text{ since } g(s) \neq 0
	\]
	Since \(B\) is the maximal element,
	we have
	\[
		B = B_{\mfm}
	\]
	which shows that \(B\) is a local ring with \(\mfm\) as its maximal
	ideal.
\end{proof}

\begin{lemma}{}{}
\label{lemma:5.20}
	Let \(x\) be a non-zero element of \(K\).

	Let \(B[x]\) be the subring of \(K\)
	generated by \(x\) over \(B\) and let \(\mfm[x]\) be the
	extension of \(\mfm\) in \(B[x]\).

	Then,
	\[
		\mfm[x] \neq B[x] \qorq
		\mfm\sbrak{x^{-1}} \neq B\sbrak{x^{-1}}
	\]
\end{lemma}
\begin{proof}
	Suppose we have
	\[
		\mfm[x] = B[x] \qandq \mfm\sbrak{x^{-1}} = B\sbrak{x^{-1}}
	\]
	Then, we will have equations
	\[
		1 \in \mfm[x] \implies 1 = u_0 + u_1 x + \cdots + u_n x^m
		\quad u_i \in \mfm
	\]
	and similarly,
	\[
		1 \in \mfm\sbrak{x^{-1}}
		\implies 1 = v_0 + v_1 x^{-1} + \cdots + v_n x^{-n}
		\quad v_j \in \mfm
	\]
	where \(m, n\) are as small as possible.

	Assume \(m \geq n\) and multiply the second equation by \(x^n\).
\end{proof}