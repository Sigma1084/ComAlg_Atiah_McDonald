\chapter{Primary Decomposition}
\label{ch:book_04_primary_decomposition}

The decomposition of an ideal into primary ideals is a traditional
pillar of ideal theory.

From another point of view primary decomposition provides a generalization
of the factorization of an integer as a product of prime-powers.

\section{Definition}

A prime ideal in a ring A is in some sense a generalization of
a prime number.
The corresponding generalization of a power of a prime number is a
primary ideal.

\begin{definition}{}{}
\label{def:primary_ideal}
	An ideal \(\mfq\) of a ring \(A\) is called \textbf{primary} if
	for any \(x, y \in A\)
	\(\mfq \neq A\) and
	\[
		xy \in \mfq \implies x \in \mfq \text{ or } y^n \in \mfq
		\text{ for some } n \geq 1
	\]
\end{definition}

\begin{proposition}{Equivalent Condition for primary}{}
\label{prop:4_0}
	\(\mfq\) is primary \(\iff \bigslant{A}{\mfq} \neq 0\) and
	every zero-divisor in \(\bigslant{A}{\mfq}\) is nilpotent.
\end{proposition}

\begin{proof}
	\(\brak{\Longrightarrow}\)
	Consider some zero-divisor \(a + \mfq\) of \(\bigslant{A}{\mfq}\).
	Then there exists some \(b + \mfq \neq 0\) such that
	\[
		\brak{a + \mfq}\brak{b + \mfq} = ab + \mfq = 0 + \mfq
	\]
	which implies that \(ab \in \mfq \implies ba \in \mfq\).
	By definition of primary ideal, \(a^n \in \mfq\) or \(b \in \mfq\).

	Since \(b \notin \mfq\), \(a^n \in \mfq\) for some \(n \geq 1\).
	Then \(\brak{a + \mfq}^n = a^n + \mfq = 0 + \mfq\).

	\(\brak{\Longleftarrow}\)
	Let \(ab \in \mfq\).
	If \(a \in \mfq\), we are done.
	Suppose \(a \notin \mfq\).

	Then,
	\[
		ab + \mfp = 0 + \mfp \text{ but } a + \mfp \neq 0 + \mfp
		\implies b + \mfq \text{ is a zero-divisor in } \bigslant{A}{\mfq}
	\]

	This implies that \(b + \mfq\) is nilpotent in \(\bigslant{A}{\mfq}\).

	Hence, \(\brak{b + \mfq}^n = b^n + \mfq = 0 + \mfq\) which
	implies \(b^n \in \mfq\).
\end{proof}


\begin{proposition}{Smallest prime ideal containing \(\mfq\)}{}
\label{prop:4_1}
	Let \(\mfq\) be a primary ideal of a ring \(A\).
	Then, \(\rad(\mfq)\) is the smallest prime ideal containing \(\mfq\).
\end{proposition}
\begin{proof}
	Consider some \(x, y \in A\) such that \(xy \in \rad(\mfq)\).
	Then, \(\brak{xy}^n \in \mfq\) for some \(n \geq 1\).

	By definition of primary ideal, \(x^n \in \mfq\) or \({y^n}^m \in \mfq\)
	for some \(m \geq 1\).

	If \(x^n \in \mfq\), then \(x \in \rad(\mfq)\) and if \({y^n}^m \in \mfq\),
	then \(y \in \rad(\mfq)\) thus implying that \(\rad(\mfq)\) is prime. \\

	For the smallest part, let \(\mfp\) be a prime ideal containing \(\mfq\).
	Consider some element \(a \in \rad(\mfq)\).

	\(a^n \in \mfq \subseteq \mfp\) for some \(n \geq 1\) implies that
	\(a \in \mfp\).

	Hence, \(\rad(\mfq) \subseteq \mfp\).
\end{proof}

\begin{defn}{\(\mfp\)-primary}{}
	An ideal \(\mfq\) of a ring \(A\) is called \textbf{\(\mfp\)-primary}
	if \(\mfq \neq A\) and \(\rad(\mfq) = \mfp\).
\end{defn}


\section{Examples}

The primary ideals in \(\ZZ\) are precisely the ideals of the form
\(\brak{p^n}\) where \(p\) is a prime number and \(n \geq 1\) including
the zero ideal \(\brak{0}\).

\begin{example}{}{}
	Consider \(A = k[x, y]\) where \(k\) is a field, \(\mfq = (x, y^2)\).
	Then,
	\[
		\bigslant{A}{\mfq} \quad \cong \quad \bigslant{k[x]}{(y^2)}
	\]
	where the zero-divisors are all the multiples of \(y\) and are
	hence nilpotent.

	Hence, \(\mfq\) is primary and it's radical is \(\mfp = (x, y)\).

	Here, we have
	\[
		\rad(\mfq)^2 = \mfp^2 \subsetneq \mfq \subsetneq \mfp = \rad(\mfq)
	\]
\end{example}
The above example shows that \textbf{a primary ideal
need not be a power of a prime ideal}.

Conversely, we also show that \textbf{a power of a prime ideal} \({\mfp}^n\)
is \textbf{not necessarily primary} although it's radical is the prime \(\mfp\).

\begin{example}{}{}
	Consider \(A = \bigslant{k[x, y, z]}{(xy - z^2)}\).
	Let \(\bar{x}, \bar{y}, \bar{z}\) denote the images of \(x, y, z\)
	respectively in \(A\).

	Consider the ideal \(\mfp = (\bar{x}, \bar{z})\).
	Clearly, \(\mfp\) is a prime ideal since
	\[
		\bigslant{A}{\mfp} \quad \cong \quad k[y]
	\]
	which is an integral domain.

	Now, we have \(\bar{z}^2 = \bar{x}\bar{y} \in \mfp^{2}\).
	Clearly, \(\bar{x} \notin \mfp^2\).

	Also, \(\bar{y}^n \in \mfp^{2} \implies \bar{y} \in \rad(\mfp^2) = \mfp\).
	But, \(\bar{y} \notin \mfp\) and hence, \(\mfp^2\) is not primary.
\end{example}

However, the powers of a maximal ideal are primary.

\begin{proposition}{Powers of a maximal \(\mfm\) are \(\mfm\)-primary}{}
\label{prop:4_2}
	If \(\rad(\mfa)\) is maximal, then \(\mfa\) is primary.
	In particular, the powers of a maximal ideal are primary.
\end{proposition}
\begin{proof}
	Let \(\rad(\mfa) = \mfm\) be maximal.
	The image of \(\mfm\) in  \(\bigslant{A}{\mfa}\) is the
	nilradical of \(\bigslant{A}{\mfa}\).

	Hence, \(\bigslant{A}{\mfa}\) has only one prime ideal, which is the
	nilradical of \(\bigslant{A}{\mfa}\).

	Hence, any element of \(\bigslant{A}{\mfa}\) is either a unit
	or nilpotent.

	This implies that any zero-divisor in \(\bigslant{A}{\mfa}\) is nilpotent.

	By proposition~\ref{prop:4_0}, \(\mfa\) is primary.
\end{proof}


\section{Intersections of primary ideals}
\label{sec:4_intersections}


\begin{lemma}{}{}
\label{lemma:4_3}
	Let \(\mfq_i\) where \(1 \leq i \leq n\) are \(\mfp\)-primary ideals
	of \(A\).
	Then, \(\mfq = \bigcap_{i = 1}^{n} \mfq_i\) is \(\mfp\)-primary.
\end{lemma}
\begin{proof}
	We first show \(\rad(\mfq) = \mfp\).
	\[
		\rad(\mfq) = \rad \brak{\bigcap_{i = 1}^{n} \mfq_i} =
		\bigcap_{i = 1}^{n} \rad(\mfq_i) = \bigcap_{i = 1}^{n} \mfp
		= \mfp
	\]
	Now, we show that \(\mfq\) is primary.
	Consider some \(ab \in \mfq\) such that \(a \notin \mfq\).

	There exists an \(i\) such that \(a \notin \mfq_i\).
	Then, \(b \in \mfp\).

	Since \(\rad(\mfq) = \mfp\), we have \(b^n \in \mfq\)
	for some \(n \geq 1\).
\end{proof}


\begin{lemma}{}{}
\label{lemma:4_4}
	Let \(\mfq\) be a \(\mfp\)-primary ideal of \(A\) and \(x \in A\).
	Then,
	\begin{enumerate}
		\item \(x \in \mfq \implies \brak{\mfq \colon x} = (1)\).
		\item \(x \notin \mfq \implies \brak{\mfq \colon x}\) is
			\(\mfp\)-primary and hence, \(\rad\brak{\mfq \colon x} = \mfp\).
		\item \(x \notin \mfp \implies \brak{\mfq \colon x}  = \mfq\).
	\end{enumerate}
\end{lemma}
\begin{proof}
	\begin{enumerate}
		\item By definition,
		\begin{align*}
			& \brak{\mfq \colon x} = \fbrak{y \in A \colon yx \in \mfq} \\
			x \in \mfq \implies & yx \in \mfq \ \forall\ y \in A \implies
			\brak{\mfq \colon x} = \brak{1}
		\end{align*}

		\item Consider \(xy \in \mfq\).
		As \(x \notin \mfq\), \(y \in \mfp\).
		Hence, we have
		\[
			\mfq \subseteq \brak{\mfq \colon x} \subseteq \mfp
		\]
		Taking radical, we have
		\[
			\rad\brak{\mfq \colon x} = \mfp
		\]
		Consider \(yz \in \brak{\mfq \colon x}\) with \(y \notin \mfp\).
		Then, \(yzx \in \mfq\) and hence, \(zx \in \mfq\) which implies
		that \(z \in \brak{\mfq \colon x}\).

		\item Since \(x \notin \mfp\), \(x^n \notin \mfq\) for any
		\(n \geq 1\).
		\[
			\brak{\mfq \colon x} = \fbrak{y \in A \colon yx \in \mfq} =
			\fbrak{y \in A \colon y \in \mfq} = \mfq
		\]
	\end{enumerate}
\end{proof}

\subsection{Primary decomposition}

\begin{definition}{Primary Decomposition}{}
	A \textbf{primary decomposition} of an ideal \(\mfa\) of a ring \(A\)
	is an expression of \(\mfa\) as an intersection of finitely many
	primary ideals, say \(\mfq_i\)
	\[
		\mfa = \bigcap_{i = 1}^{n} \mfq_i
	\]
	We then say that \(\mfa\) is \textbf{decomposable} or \textbf{reducible}.
\end{definition}

\begin{note}
	In general, a primary decomposition need not exist.
	In this chapter, we restrict ourselves to ideals which
	have a primary decomposition.
\end{note}

\begin{defn}{Minimal Primary Decomposition}{}
	A primary decomposition \(\mfa = \bigcap_{i = 1}^{n} \mfq_i\) is said
	to be \textbf{minimal} (or irredundant, or reduced, or normal) if
	\begin{enumerate}
		\item \(\rad(\mfq_i) \neq \rad(\mfq_j)\) for any \(i \neq j\).
		\item \(\mfq_i \not\subseteq \bigcap_{j \neq i} \mfq_j\)
			for all \(i\).
	\end{enumerate}
\end{defn}

\begin{proposition}{}{}
	Let \(\mfa\) be an ideal of a ring \(A\) with a primary decomposition
	\[
		\mfa = \bigcap_{i = 1}^{n} \mfq_i
	\]
	Then, \(\mfa\) has a minimal primary decomposition.
\end{proposition}
\begin{proof}
	We can achieve the first condition by combining all the
	\(\mfq_i\) with the same radical into one using the
	lemma~\ref{lemma:4_3}.

	The second condition can be achieved by removing all the
	superfluous \(\mfq_i\).
\end{proof}


\subsection{Uniqueness of primary decomposition}

\begin{theorem}{1st uniqueness theorem}{}
\label{theorem:4_5}
	Let \(\mfa\) be a decomposable ideal and let
	\[
		\mfa = \bigcap_{i = 1}^{n} \mfq_i
	\]
	be a minimal primary decomposition of \(\mfa\).
	Let \(\mfp_i = \rad(\mfq_i)\).
	Then, the \(\mfp_i\) are precisely the prime ideals
	\(\rad\brak{\mfa \colon x} \brak{x \in A}\), and hence are
	independent of the particular decomposition of \(\mfa\).
\end{theorem}
\begin{proof}
	Consider some \(x \in A\).
	\[
		\brak{\mfa \colon x} = \brak{\bigcap_{i = 1}^{n} \mfq_i \colon x}
		= \bigcap_{i = 1}^{n} \brak{\mfq_i \colon x}
	\]
	Using lemma~\ref{lemma:4_4}, we have
	\[
		\brak{\mfa \colon x} = \bigcap_{i = 1}^{n} \brak{\mfq_i \colon x}
		= \bigcap_{x \notin \mfq_i}^{n} \mfp_i
	\]
	Suppose \(\rad(\mfa \colon x)\) is prime
	(we assume this in order to show that \(\mfp_i\) occur in this form
	and vice versa).

	Using proposition 1.11, we have
	\[
		\rad\brak{\mfa \colon x} = \mfp_j \text{ for some } j
		\ \st x \notin \mfq_j
	\]
	Hence, every prime ideal of the form \(\rad\brak{\mfa \colon x}\)
	is one of the \(\mfp_i\).

	Now, we need to show that every \(\mfp_i\) occurs in the form
	\(\rad\brak{\mfa \colon x}\) for some \(x \in A\).

	Notice that since the decomposition is minimal, we have
	\[
		\forall\ i, \ \exists\ x_i \notin \mfq_i \ \st
		x_i \in \bigcap_{j \neq i} \mfq_j
	\]

	Hence, \(\rad\brak{\mfa \colon x_i} = \mfp_i\).
\end{proof}

\begin{note}
	The prime ideals \(\mfp_i\) are said to \textbf{belong} to \(\mfa\)
	or \textbf{associated} to \(\mfa\).

	The ideal \(\mfa\) is primary if and only if it has only one
	associated prime ideal.

	The minimal elements of the set of associated prime ideals of \(\mfa\)
	are called the \textbf{minimal/isolated
	associated prime ideals} of \(\mfa\).

	The others are called the \textbf{embedded associated prime ideals}
	of \(\mfa\).
\end{note}

\begin{note}
	Note that in the above proof, coupled with lemma~\ref{lemma:4_4},
	we have shown that
	\[
		\forall\ i, \ \exists\ x_i \in A \ \st
		\brak{\mfa \colon x_i} \text{ is } \mfp_i \text{-primary}
	\]
	Considering \(\bigslant{A}{\mfa}\) as an \(A\)-module,
	theorem~\ref{theorem:4_5} is equivalent to saying that
	\(\mfp_i\) are precisely the prime ideals which occur as
	\(\rad\brak{\Ann(x)}\) for some	\(x \in \bigslant{A}{\mfa}\).
\end{note}


\begin{example}{}{}
	Consider \(A = k[x, y]\) where \(k\) is a field.
	Let \(\mfa = (x^2, xy)\).
	Then, \(\mfa\) is decomposable since
	\[
		\mfa = \mfp_1 \cap \mfp_2^2
		\quad \text{ where } \mfp_1 = (x) \AND \mfp_2 = (x, y)
		\text{ are prime}
	\]
	\(\mfp_2\) being a maximal implies that \(\mfp_2^2\) is primary.

	We have \(\rad(\mfp_1) = \mfp_1\) and \(\rad(\mfp_2^2) = \mfp_2\)
	and hence, \(\rad\brak{\mfa} = \mfp_1 \cap \mfp_2 = \mfp_1\).
\end{example}

In the above example, we have \(\mfp_1 \subseteq \mfp_2\) and hence,
\(\mfp_2\) is an embedded associated prime ideal of \(\mfa\) while
\(\mfp_1\) is an isolated associated prime ideal of \(\mfa\).


\begin{proposition}{}{}
\label{prop:4_6}
	Let \(\mfa\) be a decomposable ideal.
	Then, any prime ideal \(\mfp \supseteq \mfa\) contains a minimal
	prime ideal belonging to / associated with \(\mfa\).

	Thus, the minimal prime ideals belonging to \(\mfa\) are precisely
	the minimal elements of the set of prime ideals containing \(\mfa\).
\end{proposition}

\begin{proof}
	Consider some prime ideal \(\mfp \supseteq \mfa\).
	Then,
	\begin{align*}
		\mfp &\supseteq \bigcap_{i = 1}^{n} \mfq_i \\
		\implies \rad(\mfp) &\supseteq \bigcap_{i = 1}^{n} \rad(\mfq_i) \\
		\implies \mfp &\supseteq \bigcap_{i = 1}^{n} \mfp_i
	\end{align*}
	Hence, by \(1.11\), we have \(\mfp \supseteq \mfp_i\) for some \(i\)
	which shows that \(\mfp\) contains a minimal prime ideal belonging to
	\(\mfa\).
\end{proof}


\begin{proposition}{}{}
\label{prop:4_7}
	Let \(\mfa\) be a decomposable ideal, and let
	\[
		\mfa = \bigcap_{i = 1}^{n} \mfq_i
	\]
	be a minimal primary decomposition and let \(\mfp_i = \rad(\mfq_i)\).
	Then,
	\[
		\bigcup_{i=1}^n \mfp_i =
		\fbrak{x \in A \mid \brak{\mfa \colon x} \neq \mfa}
	\]
	In particular, if the zero ideal is decomposable, then the set \(D\)
	of zero-divisors in \(A\) is the union of the prime ideals
	belonging to / associated with the zero ideal.
\end{proposition}
\begin{proof}
	If \(\mfa\) is decomposable, then 0 is decomposable in
	\(\bigslant{A}{\mfa}\).
	\[
		0 = \bigcap_{i = 1}^{n} \bar{\mfq_i} \quad \text{ where }
		\bar{\mfq_i} \text{ is the image of } \mfq_i \text{ in }
		\bigslant{A}{\mfa}
	\]
	Hence, it is enough to prove the last statement.

	We know that
	\[
		D = \bigcup_{x \in A} \rad\brak{0 \colon x}
	\]
	From the proof of theorem~\ref{theorem:4_5}, we have
	\[
		\rad\brak{0 \colon x} = \bigcap_{x \notin \mfq_j} \mfp_j
		\subseteq \mfp_j \text{ for some } j
	\]
	Hence, we have
	\[
		D \subseteq \bigcup_{j = 1}^{n} \mfp_j
	\]
	Also from the proof of theorem~\ref{theorem:4_5}, we have
	that each \(\mfp_i\) is of the form \(\rad\brak{0 \colon x}\)
	for some \(x \in A\).
	Hence,
	\[
		\bigcup_{j = 1}^{n} \mfp_j \subseteq D
	\]
\end{proof}

Thus, (the zero ideal being decomposable)
\begin{align*}
	D &= \bigcup \mfp_i \text{ where } \mfp_i \text{ is associated to } 0 \\
	\mfN &= \bigcap \mfp_i \text{ where } \mfp_i
	\text{ is minimal and associated to } 0
\end{align*}


\section{Primary ideals under localization}

\begin{proposition}{}{}
\label{prop:4_8}
	Let \(S\) be a multiplicatively closed subset of a ring \(A\),
	and let \(\mfq\) be a \(\mfp\)-primary ideal of \(A\).
	Then,
	\begin{enumerate}
		\item \(S \cap \mfp \neq \phi \implies S^{-1}\mfq = S^{-1} A\)
		\item \(S \cap \mfp = \phi \implies S^{-1}\mfq\) is
			\(S^{-1}\mfp\)-primary and it's contraction is \(\mfq\).
	\end{enumerate}
\end{proposition}
\begin{proof} \
	\begin{enumerate}
		\item Consider some \(s \in S \cap \mfp\).
		Then, \(s \in \mfp \implies s^n \in \mfq\) for some
		\(n \geq 1\).

		Hence,
		\[
			\frac{s^n}{1} \in S^{-1}\mfq \implies
			\frac{s^n}{s^n} = \frac{1}{1} \in S^{-1}\mfq
		\]
		which implies that \(S^{-1}\mfq = S^{-1}A\).

		\item If \(S \cap \mfp = \phi\), then, using
		proposition~\ref{prop:3_11}(4), when \(S \cap \mfp = \phi\), we have
		\[
			s \in S \AND as \in \mfq \implies a \in \mfq
			\implies \mfq^{ec} = \mfq
		\]
		Also from proposition~\ref{prop:3_11}(5), we have
		\[
			\rad\brak{\mfq^e} = \rad\brak{S^{-1}\mfq} = S^{-1}\mfp
		\]
		To verify that \(S^{-1}\mfq\) is primary, consider
		\[
			\frac{a}{s} \cdot \frac{b}{t} \in S^{-1}\mfq
			\implies \frac{ab}{st} \in S^{-1}\mfq
		\]
		This boils down to \(ab \in \mfq\) and the proposition follows.
	\end{enumerate}
\end{proof}

\begin{defn}{Contraction of an ideal in \(S^{-1}A\)}{}
	For an ideal \(\mfa\) and a multiplicatively closed subset \(S\),
	the \textbf{contraction} of \(\mfa\) in \(S^{-1}A\) is denoted by
	\[
		S\brak{\mfa} \coloneqq \text{Contraction of the ideal }
		S^{-1}\mfa \normsg S^{-1}A \text{ where } \mfa \normsg A
	\]
\end{defn}

\begin{proposition}{}{}
\label{prop:4_9}
	Let S be a multiplicatively closed subset of a \(A\) and let \(\mfa\)
	be a decomposable ideal of \(A\).
	Let
	\[
		\mfa = \bigcap_{i = 1}^{n} \mfq_i \qstq \mfp_i = \rad(\mfq_i)
	\]
	be a minimal primary decomposition of \(\mfa\).

	Suppose the \(\mfq_i\) are numbered so that \(S\) meets
	\(\mfp_{m+1}, \ldots, \mfp_n\) but not \(\mfp_1, \ldots, \mfp_m\).
	Then,
	\[
		S^{-1}\mfa = \bigcap_{i = 1}^{m} S^{-1}\mfq_i
		\quad\quad \text{ and } \quad\quad
		S\brak{\mfa} = \bigcap_{i = 1}^{m} \mfq_i
	\]
	and these are minimal primary decompositions.
\end{proposition}
\begin{proof}
	Using proposition~\ref{prop:3_11}(5) and proposition~\ref{prop:4_8},
	we have
	\begin{align*}
		S^{-1}\mfa &= \bigcap_{i = 1}^{n} S^{-1}\mfq_i
			&~\text{\ref{prop:3_11}(5)} \\
		S^{-1}\mfa &= \bigcap_{i = 1}^{m} S^{-1}\mfq_i
			&~\text{\ref{prop:4_8}} \\
	\end{align*}
	and \(S^{-1}\mfq_i\) are \(S^{-1}\mfp_i\)-primary for
	\(i = 1, \ldots, m\).

	Since \(\mfp_i\) are distinct, \(S^{-1}\mfp_i\) are distinct
	and hence, we have a minimal primary decomposition of \(S^{-1}\mfa\).

	Contracting both sides, we get
	\[
		S\brak{\mfa} = \brak{S^{-1}\mfa}^c =
		\bigcap_{i = 1}^{m} \brak{S^{-1}\mfq_i}^c
		= \bigcap_{i = 1}^{m} \mfq_i
	\]
	using proposition~\ref{prop:4_8} again.
\end{proof}

\begin{defn}{Set of associated isolated ideals}{}
	A set \(\Sigma\) of prime ideals belonging to / associated with \(\mfa\)
	is said to be \textbf{isolated} if it satisfies the following condition:
	\[
		\mfp' \text{ is a prime ideal associated with } \mfa \AND
		\mfp' \subseteq \mfp \text{ for some } \mfp \in \Sigma
		\implies \mfp' = \mfp
	\]
\end{defn}

\begin{proposition}{}{}
	Let \(\Sigma\) be an isolated set of prime ideals associated with
	\(\mfa\) and let
	\[
		S = A \setminus \bigcup_{\mfp \in \Sigma} \mfp
	\]
	Then, \(S\) is multiplicatively closed and for any prime ideal
	\(\mfp'\) associated with \(\mfa\), we have
	\begin{align*}
		\mfp' \in \Sigma &\implies \mfp' \cap S = \phi \\
		\mfp' \notin \Sigma &\implies \mfp' \not\subseteq
		\bigcup_{\mfp \in \Sigma} \mfp \implies \mfp' \cap S \neq \phi
		& \text{using proposition 1.11}
	\end{align*}
\end{proposition}

\subsection{2nd uniqueness theorem}
\begin{theorem}{}{}
\label{theorem:4_10}
	Let \(\mfa\) be a decomposable ideal and let
	\[
		\mfa = \bigcap_{i = 1}^{n} \mfq_i
	\]
	be a minimal primary decomposition of \(\mfa\) and let
	\(\fbrak{\mfp_{i_1}, \mfp_{i_2}}, \cdots, \mfp{i_m}\) be an
	isolated set of prime ideals in \(\mfa\).
	Then,
	\[
		\mfq_{i_1} \cap \cdots \cap \mfq_{i_m}
	\]
	is independent of the decomposition.
\end{theorem}
\begin{proof}
	We know that \(\mfp_i\) depend only on \(\mfa\) using
	theorem~\ref{theorem:4_5}.

	Also, from proposition~\ref{prop:4_9}, we have
	\[
		q_{i_1} \cap \cdots \cap \mfq_{i_m} =
		S(\mfa) \quad \text{where } S = A \setminus
		\bigcup_{j = 1}^{m} \mfp_{i_j}
	\]
	and hence depends only on \(\mfa\).
\end{proof}

In particular, we have
\begin{corollary}{}{}
\label{corollary:4_11}
	The isolated primary components, the primary components \(\mfp_i\)
	corresponding to minimal prime ideals \(\mfp_i\) are uniquely determined
	by \(\mfa\).
\end{corollary}

\begin{note}
	On the other hand, the embedded primary components are in general
	not uniquely determined bt \(\mfa\).
	If \(A\) is a noetherian ring, then there are in fact
	infinitely many choices for each embedded component
	(refer to chapter 8, exercise 1).
\end{note}