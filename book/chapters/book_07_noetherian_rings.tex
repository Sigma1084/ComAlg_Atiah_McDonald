\chapter{Noetherian rings} \label{ch:noetherian_rings}

\section{Introduction} \label{sec:noetherian_rings_intro}

\begin{definition}{Noetherian Rings}{}
	A ring \(A\) is said to be \emph{Noetherian} if it satisfies
	one of the following equivalent conditions:
	\begin{enumerate}
		\item Every non-empty set of ideals in \(A\) has a maximal element.
		\item Every ascending chain of ideals in \(A\) is stationary.
		\item Every ideal in \(A\) is finitely generated.
	\end{enumerate}
\end{definition}

Noetherian Rings are by far the most important class of rings in
commutative albergra.

In this chapter, we show that Noetherian rings reproduce themselves
under various famililar operations - in particular, we prove the
famous basis theorem of Hilbert.



\section{Basic properties of Noetherian Rings}
\label{sec:noetherian_rings_basic_properties}


\begin{proposition}{}{}
	If \(A\) is Neotherian and \(\phi\) is a homomorphism of \(A\)
	onto a ring \(B\), then \(B\) is Noetherian.
	\[
		\phi \colon A \to B \quad
		\text{ is a ring homomorphism}
	\]
\end{proposition}

\begin{proof}
	Using the First Isomorphism Theorem, we have
	\[
		B \cong \bigslant{A}{\Ker(\phi)}
	\]
	We have seen that if \(A\) is Noetherian, then so is
	\(\bigslant{A}{\Ker(\phi)}\).
	Therefore, \(B\) is Noetherian.
\end{proof}

\begin{proposition}{}{}
	Let \(A\) be a subring of \(B\) and suppose \(A\) is Noetherian and
	\(B\) is \emph{finitely generated} as a \(A\)-module.
	Then \(B\) is Noetherian (as a ring).
\end{proposition}

\begin{proof}
	\(B\) is \emph{finitely generated} \(A\)-module and is hence,
	Noetherian as a \(A\)-module and hence, is Noetherian as a
	\(B\)-module.
\end{proof}

\begin{proposition}{}{}
	If \(A\) is Noetherian and \(S\) is any multiplicatively closed
	subset of \(A\), then \(S\inv A\) is Noetherian.
\end{proposition}

\begin{proof}
	TODO  % TODO
\end{proof}

\begin{corollary}{}{}
	If \(A\) is Noetheiran and \(\mfp\) is a prime ideal of \(A\),
	then \(A_{\mfp}\) is Noetherian.
\end{corollary}



\section{The Hilbert Basis Theorem}
\label{sec:noetherian_rings_hilbert_basis_theorem}

\begin{theorem}{Hilbert Basis Theorem}{}
	If \(A\) is Noetherian, then the polynomial ring
	\(A[x]\) is Noetherian.
\end{theorem}


Let \(\mfa\) be an ideal in \(A[x]\).
The leading coefficients of the polynomials in \(\mfa\) form an ideal
\(I\) in \(A\).
Now, since \(A\) is Noetherian, \(I\) is finitely generated, say
\[
	I = \ang*{a_1, \ldots, a_n}
\]
For each \(i = 1, \cdots n\), we have a polynomial \(f_i\) in \(A[x]\)
of the form and let us define \(r\)
\[
	f_i = a_i x^{r_1} + \text{ (lower terms)}
	\quad\quad \mfa' \coloneqq \ang{f_1, \ldots, f_n} \subseteq \mfa
\]
and let \(r \coloneqq \max_{i=1}^n r_i\).

Consider some polynomial \(f\) in \(\mfa\).
\[
	f = ax^m + \text{ (lower terms)} \quad \in \mfa
\]
We have \(a \in I\).
If \(m \geq r\), we write
\[
	a = u_1 a_1 + \cdots + u_n a_n \quad u_i \in A
\]
Now, we can notice that
\[
	f - \sum_{i=1}^n u_i f_i x^{m - r_1} \quad \in \mfa
\]
and has a degree \(< m\).
We can continue this process until we can represent \(f\) as a sum
\[
	f = g + h
\]
where \(g\) is a polynomial of degree \(< r\) in \(\mfa\)
and \(h\) is a polynomial in \(\mfa_1\).

Let \(M\) be the \(A\)-module generated by \(1, x, \cdots, x^{r-1}\).
Then, we have
\[
	M = \ang*{1, x, \cdots, x^{r-1}} \qandq
	g \in \mfa \cap M
\]

We just proved that
\[
	\mfa = \brak{\mfa \cap M} + \mfa'
\]
Since \(M\) is a finitely generated \(A\)-module,
\(\mfa \cap M \normsg M\) and is hence a finitely generated \(A\)-module.

If \(g_i\) generate \(\mfa \cap M\), then it is clear that
\(f_i\) and \(g_i\) generate \(\mfa\).

Thus,
\[
	A \text{ is Noetherian}
	\implies A[x] \text{ is Noetherian}
\]

\begin{corollary}{}{}
	If \(A\) is Noetherian, then so is \(A[x_1, \cdots, x_n]\)
\end{corollary}
