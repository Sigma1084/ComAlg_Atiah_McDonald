\chapter{Modules 1} \label{ch:modules1}

\section{Modules and Module Homomorphisms}

\begin{definition}{Module}{}
	Let \( A \) be a ring (commutative, with identity).
	A \emph{(left) \( A \)-module} is an abelian group \( M \)
	(written additively) on which \( A \) acts linearly
	(written multiplicatively).

	More precisely, it is a pair \brak{M, \mu_A} where \( M \)
	is an abelian group and \(\mu_A: A \times M \to M\) is a
	map satisfying the following axioms:

	\begin{enumerate}
		\item \( a(x + y) = ax + ay \)
		\item \( (a + b)x = ax + bx \)
		\item \( (ab)x = a(bx) \)
		\item \( 1x = x \)
	\end{enumerate}

	for any \( a, b \in A \AND x, y \in M \).
\end{definition}

Equivalently, a module is an abelian group \( M \) with a
ring homomorphism \( A \to \End(M) \) where \( \End(M) \)
is the ring of endomorphisms of the abelian group \( M \).

\begin{example}{Modules}{}
	The following are common examples of modules.
	\begin{enumerate}
		\item The ideals, \( \mfa \) of a ring \( A \) are \( A \)
		modules.
		In particular, \( A \) is itself an \( A \) module.
		\item If \( A \) is a field \( \kappa \), then
		\( \kappa \)-modules are \( \kappa \)-vector spaces.
		\item If \( A \) is \( \ZZ \), then abelian groups are
		\( \ZZ \)-modules.
		\item If \( A = \kappa[x] \) where \( \kappa \) is a field,
		an \( \kappa[x] \)-module is a \( \kappa \)-vector space
		with a linear transformation.
	\end{enumerate}
\end{example}


\subsection{Homomorphisms of Modules}
\begin{defn}{}{}
	Let \(M, N\) be \(A\)-modules and \(f: M \to N\) a map.
	Then, we say that \(f\) is an \emph{\(A\)-module homomorphism} or an
	\emph{\(A\)-linear map} if it satisfies:
	\begin{align*}
		f(x + y) &= f(x) + f(y) \quad\quad \forall\ x, y \in M \\
		f(ax) &= a \cdot f(x) \quad \quad \forall\ a \in A \AND x \in M
	\end{align*}
\end{defn}

Note that if \( A \) is a field, then a module homomorphism is
the same as a Linear Transformation of vector spaces.


\subsection{Module of Homomorphisms from \(M\) to \(N\)}
The set of all \(A\)-module homomorphisms from \(M\) to \(N\) can
be thought of as a module over \(A\).

Let us define \(f+g\) and \(af\) for \(f, g: M \to N\) and \(a \in A\) as
\begin{align*}
	\brak{f+g}(x) &= f(x) + g(x) \\
	\brak{af}(x) &= a \cdot f(x)
\end{align*}

This module is called the \emph{module of \(A\)-module homomorphisms
from \(M\) to \(N\)} and is denoted by \(\Hom_A(M, N)\).

For any module \(M\), there is a natural isomorphism
\[
	\Hom_A(A, M) \cong M
\]
The idea being, any \(A\)-module homomorphism from \(A\) to \(M\) is
uniquely determined by \(f(1)\) which can be any element in \(M\).



\section{Submodules and Quotient Modules}
In simple words, a submodule, \(M'\) of \(M\) is a normal subgroup of \(M\)
which is closed under multiplication by elements of \(A\).

The abelian group \(\bigslant{M}{M'}\) then inherits the \(A\)-module
structure from \(M\), defined by
\[
	a(x+M') = ax + M'
\]

Some natural examples of submodules occur during homomorphisms.

Consider the \(A\)-ring homomorphism \(f: M \to N\)
\[
	\Ker(f) = \fbrak{x \in M \mid f(x) = 0}
\]
is a submodule of \(M\).

Similarly, the image of \(f\) is the set
\[
	\Img(f) = f(M)
\]
is a submodule of \(N\) along with the cokernel, which is the quotient
submodule of \(N\).
\[
	\Coker(f) = \bigslant{N}{\Img(f)}
\]


\section{Operation on Submodules}

Let \(M\) be an \(A\)-module and \(\fbrak{M_i}_{i \in \sI}\) be a
family of submodules of \(M\). \\
Then, the \emph{sum} \(\sum M_i\) of the submodules is the set
fo all (finite) sums \(\sum x_i\) where \(x_i \in M_i\) for all \(i\).

The sum is the smallest submodule containing all the \(M_i\).

\begin{proposition}{}{}
	Suppose \(M_1, M_2 \normsg M\) be \(A\)-submodules of \(M\).
	Then,
	\[ M_1 + M_2 \normsg M \]
\end{proposition}

\begin{proof}
	Let \(x \in M_1 \AND y \in M_2\) and \(a \in A\).
	We know that \(M_1 + M_2\) is a normal subgroup of \(M\).

	\[
		a(x + y) = ax + ay \in M_1 + M_2 \quad \text{since}
		\quad ax \in M_1 \AND ay \in M_2
	\]

	Hence, \(M_1 + M_2 \normsg M\).
\end{proof}


\begin{proposition}{}{}
	Suppose \(M_1, M_2 \normsg M\) be \(A\)-submodules of \(M\).
	Then,
	\[ M_1 \cap M_2 \normsg M \]
\end{proposition}

\begin{proof}
	Clearly, \(M_1 \cap M_2\) is a normal subgroup of \(M\).

	Consider \(x \in M_1 \cap M_2\).
	Then, \(x \in M_1\) and \(x \in M_2\)
	and \(ax \in M_1 \AND ax \in M_2\) for all \(a \in A\).

	\(\implies ax \in M_1 \cap M_2 \implies M_1 \cap M_2 \normsg M\).
\end{proof}

\begin{note}
	The intersection \(\bigcap M_i\) is again a submodule of \(M\) and hence,
	the submodules of \(M\) form a complete lattice with
	respect to inclusion.
\end{note}


\subsection{Isomorphism Theorems}

\begin{theorem}{First Isomorphism Theorem}{}
	Let \(M, N\) be \(A\)-modules and
	\[
		f \colon M \to N
	\]
	be a \(A\)-module homomorphism.
	Then,
	\[
		\bigslant{M}{\Ker{f}} \cong \Img{f}
	\]
\end{theorem}

\begin{theorem}{Second Isomorphism Theorem}{}
	Let \(M\) be an \(A\)-module and let \( L \normsg N \normsg M\)
	be submodules of \(M\).
	Then,
	\[
		\bigslant{\bigslant{M}{L}}{\bigslant{N}{L}} \cong
		\bigslant{M}{N}
	\]
\end{theorem}

\begin{proof}
	Define a homomorphism
	\begin{align*}
		\theta \colon \bigslant{M}{L} &\to \bigslant{M}{N} \\
		x + L &\mapsto x + N
	\end{align*}
	The kernel of this homomorphism is
	\begin{align*}
		x \in \Ker(\theta) &\implies x + L \mapsto 0 + N \\
		&\implies x \mapsto 0 + \bigslant{N}{L} \\
		&\implies x \in \bigslant{N}{L} \\
		\therefore \Ker(\theta) &= \ \bigslant{N}{L}
	\end{align*}
	Using First Isomorphism Theorem, we get
	\[
		\bigslant{\bigslant{M}{L}}{\bigslant{N}{L}} \cong
		\bigslant{M}{N} \qedhere
	\]
\end{proof}


\begin{theorem}{Third Isomorphism Theorem}{}
	Let \(M\) be an \(A\)-module and let \( M_1, M_2 \normsg M\).
	Then,
	\[
		\bigslant{\brak{M_1 + M_2}}{M_1} \cong
		\bigslant{M_2}{\brak{M_1 \cap M_2}}
	\]
\end{theorem}

\begin{proof}
	Consider the homomorphism defined by
	\begin{align*}
		\theta \colon M_2 &\to \bigslant{(M_1 + M_2)}{M_1} \\
		x &\to x + M_1
	\end{align*}
	The homomorphism is surjective and the kernel is
	\begin{align*}
		\Ker(\theta) &= \fbrak{x \in M_2 \mid x + M_1 = 0 + M_1} \\
		&= \fbrak{x \in M_2 \mid x \in M_1} \\
		\Ker(\theta) &= M_1 \cap M_2
	\end{align*}
	Using the first isomorphism theorem, we get
	\[
		\bigslant{\brak{M_1 + M_2}}{M_1} \cong
		\bigslant{M_2}{\brak{M_1 \cap M_2}} \qedhere
	\]
\end{proof}


\subsection{Ideal Product}
We can not, in general define the product of two modules but we can define
the product of a module with an ideal.
\[
	\mfa M \coloneqq \fbrak{a x \mid a \in A \AND x \in M}
\]

\begin{claim}{}{}
	If \(N, P \normsg M\) are submodules of \(M\), then we can define
	\[
		\brak{N \colon P} \coloneqq \fbrak{a \in A \mid aP \subseteq N}
	\]
	Then \(\brak{N \colon P}\) is an ideal of \(A\).
\end{claim}

\begin{proof}
	\(a, b \in \brak{N \colon P} \implies aP, bP \subseteq N\).

	Clearly, \(\implies aP + bP \subseteq N \AND abP \subseteq N\) and hence,
	\(ab \in \brak{N \colon P}\).

	Consider \( r \in A \).
	\( arP \subseteq aP \subseteq N\) and hence, \(ar \in \brak{N, P}\).
\end{proof}

Particularly, the ideal \(\brak{0 \colon M}\)
is called the \emph{annihilator} of \(M\).

\begin{defn}{Faithful Module}{}
	An \(A\)-module \(M\) is called \emph{faithful} if
	\(\Ann(M) = 0\).
\end{defn}

\begin{proposition}{}{}
	For any \(A\)-module \(M\) and submodules \(N, P \normsg M\),
	\begin{enumerate}
		\item \(\Ann(N+P) = \Ann(N) \cap \Ann(P)\)
		\item \(\brak{N \colon P} = \Ann\brak{\bigslant{\brak{N+P}}{N}}\)
	\end{enumerate}
\end{proposition}

\begin{proof} \
	\begin{enumerate}
		\item
		\(a \in \Ann(N+P) \implies an = 0\ \forall\ n \in N \AND ap = 0\
		\forall\ p \in P\)
		and hence, \(a \in \Ann(N) \cap \Ann(P)\).

		Conversely, \(a \in \Ann(N) \cap \Ann(P) \implies
		a(n+p) = an + ap = 0\).
		and this implies \(a \in \Ann(N+P)\).

		Hence proved.

		\item
		\(a \in \brak{N \colon P} \implies aP \subseteq N
		\implies a(P+N) \subseteq (0+N) \implies a \in
		\Ann\brak{\bigslant{\brak{N+P}}{N}}\).

		Conversely, \(a \in \Ann\brak{\bigslant{\brak{N+P}}{N}} \implies
		a(p + N) = (0 + N) \ \forall\ p \in P \implies aP \subseteq N
		\implies a \in \brak{N \colon P}\).

		Hence proved.
	\end{enumerate}
\end{proof}

\subsection{Generators}

The set
\[
	Ax \coloneqq \brak{x} \coloneqq \fbrak{ax \mid a \in A}
\]
for some \(x \in M\) is a submodule of \(M\) and is called the submodule
generated by \(x\). \\

If
\[
	M = \sum_{i \in \sI} A x_i
\]
then \(\fbrak{x_i}_{i \in \sI}\) are called the \emph{generators} of \(M\).

This means every element of \(M\) can be expressed (not necessarily uniquely)
as a finite linear combination of the \(x_i\) with coefficients in \(A\).

An \(A\)-module \(M\) is said to be \emph{finitely generated} if it has a
finite set of generators.


\section{Direct Sum and Product}

If \(M, N\) are \(A\)-modules, then the \emph{direct sum} of \(M\) and \(N\)
is defined as
\[
	M \oplus N \coloneqq \fbrak{\brak{m, n} \mid m \in M \AND n \in N}
\]
is an \(A\)-module with the obvious coordinate wise
addition and scalar multiplication.

More generally, if \(\fbrak{M_i}_{i \in \sI}\) is a family of \(A\)-modules,
then the \emph{direct sum} of the modules is defined as
\[
	\bigoplus_{i \in \sI} M_i \coloneqq \fbrak{\brak{m_i}_{i \in \sI} \mid
	m_i \in M_i \text{ and finitely many } m_i \neq 0}
\]

If we drop the condition that the \(m_i\) are finitely many, then we get the
\emph{direct product} of the modules.
\[
	\prod_{i \in \sI} M_i \coloneqq \fbrak{\brak{m_i}_{i \in \sI} \mid
	m_i \in M_i}
\]
If the indexing set is finite, the direct product is the same as the direct
sum, but need not be in general.

\subsection{Ring}
Suppose the ring \(A\) is a direct product \(\prod_{i=1}^n A_i\).

Then the set of all elements of the form
\[
	\brak{0, 0, \cdots, a_i, \cdots, 0}
\]
with \(a_i \in A_i\) is an ideal \(\mfa_i\) of \(A\).

(It is not a subring (except in trivial cases) since the identity element
\(1\) is not in the ideal.)

If we consider the ring \(A\) as a module over itself, then
\(A\) is a direct sum of the ideals \(\mfa_i\).
\[
	A = \mfa_1 \oplus \mfa_2 \oplus \cdots \oplus \mfa_n
\]

The identity element of \(\mfa_i\), say \(e_i\) is an idempotent element
in \(A\) and we can also observe that
\[
	\mfa_i = \brak{e_i}
\]
(Here, we are looking at \(e_i\) being an element of the module generating
a submodule by the action of the ring where the submodule is the ideal)


\pagebreak
\section{Finitely Generated Modules}

\begin{defn}{Free \(A\)-module}{}
	An \(A\)-module \(M\) is called \emph{free} if it is isomorphic to
	an \(A\)-module of the form \(\bigoplus_{i \in \sI} M_i\) where
	\[
		M_i \cong A \brak{\text{as an } A\text{-module}}
	\]
	The notation \(A^{\brak{\sI}}\) is sometimes used.
\end{defn}


\begin{proposition}{}{}
	\(M\) is a finitely generated \(A\)-module \(\iff M\) is
	isomorphic to a quotient of \(A^n\) for some \(n \in \NN\).
\end{proposition}

\begin{proof} \brak{\implies}
	Suppose \(x_1, \cdots, x_n\) generate \(M\).
	Let us define an \(A\)-linear map onto \(M\) by
	\begin{align*}
		\phi \colon A^n &\to M \\
		(a_1, \cdots, a_n) &\mapsto a_1 x_1 + \cdots + a_n x_n
	\end{align*}
	\(\phi\), being surjective, using the first isomorphism theorem, we
	can say
	\[
		M \cong \bigslant{A^n}{\Ker(\phi)}
	\]

	\brak{\Longleftarrow}
	Now, we have a surjective \(A\)-linear map \(A^n\) onto \(M\). \\
	If \(e_i = \brak{0, \cdots, 1, \cdots, 0}\), then clearly,
	\(\fbrak{\phi(e_i)}_{\brak{1 \leq i \leq n}}\) generate \(M\).
\end{proof}


\begin{proposition}{}{}
	Let \(M\) be a finitely generated \(A\)-module and \(\mfa\) be an ideal
	of \(A\), and let \(\phi\) be an \(A\)-module endomorphism of \(M\)
	such that \(\phi(M) \subseteq \mfa M\).
	Then \(\phi\) satisfies an equation of the form
	\[
		\phi^n + a_1 \phi^{n-1} + \cdots + a_n = 0
	\]
	where \(a_i \in \mfa\).
\end{proposition}

\begin{proof}
	Let \(x_1, \cdots, x_n\) be a set of generators of \(M\).
	Then, each \(\phi(x_i) \in \mfa M\), and so we can write
	\begin{align*}
		\phi(x_i) &= \sum_{j=1}^n a_{ij} x_j \quad
		\brak{1 \leq i \leq n ; a_{ij} \in \mfa} \\
		\implies \sum_{j=1}^n \brak{\delta_{ij}\phi - a_{ij}} x_j &= 0
		\quad \text{where } \delta_{ij} = 1 \text{ if } i = j \text{ and }
		0 \text{ otherwise}
	\end{align*}
	Suppose \(T = \fbrak{\delta_{ij}\phi - a_{ij}}_{ij}\) be a matrix.
	We have \(Tx = 0\).

	Multiplying by \(\text{Adj}(T)\) on the left, we get
	\begin{align*}
		\text{Adj}(T) \ T x &= 0 \\
		\implies \det(T) &= 0 \\
	\end{align*}
	Expanding \(\det(T)\), we get the desired equation.
\end{proof}


\begin{corollary}{}{}
	Let \(M\) be a finitely generated \(A\)-module and let \(\mfa\) be
	an ideal of \(A\) such that \(\mfa M = M\).
	Then, there exists an \(x \equiv 1 \pmod{\mfa}\) such that
	\(xM = 0\).
\end{corollary}
\begin{proof}
	Putting \(\phi\) as the identity map, we get
	\[
		x = 1 + a_1 + \cdots + a_n
	\]
	where \(a_i \in \mfa\).
\end{proof}


\subsection{Nakayama's Lemma}

\begin{theorem}{Nakayama's Lemma}{}
	Let \(M\) be a finitely generated \(A\)-module and let \(\mfa\) be
	an ideal of \(A\) contained in the jacobson radical \(\mfR\) of \(A\).
	Then,
	\[
		\mfa M = M \implies M = 0
	\]
\end{theorem}


\begin{proof}
	We have \(xM = 0\) for some \(x \equiv 1 \pmod{\mfa}\) using the
	corollary above.
	Now, since \(\mfa\) is contained in the jacobson radical, we have
	\(\mfa\) is a unit since \(x \equiv 1 \pmod{\mfa}\) and hence
	\[
		M = x^{-1} x M = 0 \qedhere
	\]
\end{proof}

\begin{proof}[Alternate proof]
	Suppose \(M \neq 0\) and let \(u_1, \ldots, u_n\) be a minimal set of
	generators of \(M\).

	Then, we have \(u_n \in \mfa M\) and hence, we have the equation
	\begin{align*}
		u_n &= a_1 u_1 + \cdots + a_n u_n \quad\quad a_i \in \mfa \\
		\implies \brak{1 - a_n} u_n &= a_1 u_1 + \cdots + a_{n-1} u_{n-1}
	\end{align*}

	Since \(a_n \in \mfR\), we get \(1-a_n\) to be a unit and can show that
	\(a_n\) belongs to the submodule generated by \(u_1, \ldots, u_{n-1}\)
	which is a contradiction.
\end{proof}

\begin{corollary}{}{}
	Let \(M\) be a finitely generated \(A\)-module, \(N \normsg M\),
	and let \(\mfa \subseteq \mfR\) be an ideal of \(A\).
	Then,
	\[
		M = \mfa M + N \implies M = N
	\]
\end{corollary}

\begin{proof}
	Taking modulo \(N\) on both sides, we get
	\[
		\bigslant{M}{N} = \bigslant{(\mfa M + N)}{N} = \bigslant{\mfa M}{N}
		= \mfa \bigslant{M}{N}
	\]
	Now, applying Nakayama's lemma, to \(\bigslant{M}{N}\), we get
	\[
		\bigslant{M}{N} = 0 \implies M = N \qedhere
	\]
\end{proof}



\begin{proposition}{}{}
	Let \(A\) be a local ring, \(\mfm\) be its maximal ideal and
	\(k = \bigslant{A}{\mfm}\) be the residue field.

	Let \(M\) be a finitely generated \(A\)-module.
	Since \(\bigslant{M}{\mfm M}\) is annihilated by \(\mfm\), it is
	naturally a \(\bigslant{A}{\mfm}\)-module, that is, a \(k\)-vector space
	and is hence, finite-dimensional. \\

	Let \(x_i(1 \leq i \leq n)\) be elements of \(M\) whose images in
	\(\bigslant{M}{\mfm M}\) form a basis in this vector space.

	Then, the \(x_i\) generate \(M\).
\end{proposition}

\begin{proof}
	Let \(N\) be a submodule of \(M\) generated by \(x_i\).

	Then, the composition map
	\[
		N \to M \to \bigslant{M}{\mfm M}
	\]
	maps \(N\) onto \(\bigslant{M}{\mfm M}\).

	This implies
	\[
		N + \mfm M = M \implies N = M
	\]
	using the corollary above.
\end{proof}

